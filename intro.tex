\chapter{Introduction\label{ch:intro}}

This chapter will provide some information on why the neutron electric
dipole moment (nEDM) is interesting to measure.  Finding a nonzero
nEDM would answer questions regarding the matter-antimatter or baryon
asymmetry of the universe (BAU).  The measurement principle of nEDM
experiment and the origin of systematic effects causing false EDM's
will be explained.  This will lead to a discussion on the importance
of magnetometry on nEDM experiment.

\section{CP violation and the Standard Model}

Charge conjugation (C), Parity (P) and Time-reversal (T) symmetry are
discrete symmetries in physics.  C-symmetry describes the symmetry of
physical laws under a particle-antiparticle transformation.
P-symmetry describes the inversion of spatial coordinates and
T-symmetry the direction of time.

CPT-symmetry is believed to be a good symmetry of nature because all
relativistic quantum field theories are invariant under successive
application of the three discrete symmetries.  This is known as the
CPT theorem.

Parity symmetry was discovered to be violated in the weak interaction
through observations of spin correlations in beta-decay.  This is due
to neutrinos having a particular handedness.  This problem can be
fixed proposing that CP-symmetry is a good symmetry of nature.
However, CP-symmetry has also been found to be violated in weak decays
of kaons and B-mesons.

The nEDM is an observable that if found to be non-zero, would indicate
a violation of T-symmetry.  Because of the CPT theorem this is
equivalent to CP-violation.  The current best measurement of the nEDM
gives the upper bound $|d_n|<3.0\times
10^{-26}~e\cdot$cm~\cite{bib:baker,bib:pendlebury-revised}.
We now discuss how this result impacts physics within and beyond the
standard model.

CP violations in the standard model are arising from two separate
sources.  The first source is found in the strong interaction,
described by the $\theta$ term in the quantum chromodynamics (QCD)
Lagrangian.  This CP violating term in the QCD Lagrangian violates
both parity and time symmetry and induces a nEDM of
$|d_n|\sim-(0.9-1.2)\times
10^{−16}\theta~e\cdot$cm~\cite{bib:chuppetal} with $\theta$ being a
dimensionless parameter of the standard model.  A combination of nEDM
and Hg-EDM measurements limit the parameter to be very small
$\theta\lesssim 10^{-10}$.  The reason for the smallness of $\theta$
is currently unknown and this is sometimes called the strong CP
problem.  The second source of CP violation in the SM arises from a
complex phase in the Cabibbo-Kobayashi-Maskawa (CKM)
matrix\cite{PhysRevLett.10.531}.  The phase is responsible for CP
violation in K and B meson decays.  the CP violation within the CKM
matrix predicts $|d_n| = 10^{-31} - 10^{-32}~e\cdot$cm, well below the
current best experimental limit stated above.
%This is known as the
%weak CP-problem since it arises through W boson
%exchange~\cite{PhysRevLett.82.904}.


\section{New Physics and the Baryon asymmetry}

The T-violating nEDM is considered to be a promising probe for physics
beyond the Standard Model, a broad variety of
scenarios~\cite{bib:pospelov}.  Some theories try to make a consistent
description of the nEDM and baryogenesis, the origin of the baryon
asymmetry in the universe~\cite{bib:chuppetal}.

According to the Big Bang theory, matter and antimatter have been
created in equal amounts in the early universe. But the present
universe is overwhelmingly made up of matter rather than
anti-matter.
% The ratio of baryonic matter to photons can be expressed
%as
%\begin{equation}
%  \frac{n_B}{n_\gamma}=10^{-10}
%\end{equation}
%where $n_B$ is the difference between baryons and anti-baryons number
%and $n_\gamma$ is the number of photons in the Cosmic Backgorund.
The amount of baryon asymmetry generated in the standard model is much
smaller than current observations, in the scenario of electroweak
baryogenesis~\cite{bib:morrissey}.

The Sakharov conditions describe a way to explain the evolution of a
baryon asymmetry from an initial symmetric
condition~\cite{budker2013optical,PhysRevLett.10.531}.
\begin{itemize}
    \item Baryon number violation.
    \item Violation of C-symmetry and therefore CP.
    \item Interactions away from thermal equilibrium
\end{itemize}
Although all of these ingredients are available in the standard model
in principle, the underprediction of electroweak baryogenesis
motivates extensions to the standard model to increase the amount of
CP violation in an attempt to fix electroweak baryogenesis.  These
same new physics scenarios often predict a non-zero nEDM because of
the increased CP violation.


\section{Neutron electric dipole moment and CP violation }  

Neutron has an intrinsic electric dipole moment (EDM).  The neutron
EDM is a measure for the distribution of positive and negative charges
inside the neutron~\cite{bib:chuppetal}.  The intrinsic EDM of neutron
interacting with external magnetic and electric fields can be
described by the following Hamiltonian:
\begin{equation}\label{my_first_eqn}  
  H=-\vec{\mu}_n\cdot\vec{B}-\vec{d}_n\cdot\vec{E}
\end{equation}
where $\mu_n$ is the magnetic moment of the neutron interacting with
the magnetic field $B$, and $d_n$ is the electric dipole moment of the
neutron interacting with the electric field $E$.  Since the neutron
spin is an axial vector and the electric dipole moment vector is a
scalar vector, both vectors behave differently under P- or
T-transformations.  The orientation of the electric dipole moment
changes under P operation but leaves the magnetic moment unchanged. On
the other hand, T-operation affects the spin vector but leaves the
electric dipole moment unchanged.  Because of the conservation of CPT
symmetry, a non-zero electric dipole moment of the neutron would be a
violation of parity (P) and time-reversal (T) symmetry.

\section{nEDM measurement principle}

Ramsey's method of separated oscillatory fields \cite{bib:ramsey} is
used to extract the nEDM.  In the nEDM experiment, ultracold neutrons
(UCN) whose spins are oriented along a uniform magnetic field are
stored in a chamber.  An electric field is applied parallel to the
magnetic field.
%After applying the magnetic field $B_0$ along the
%$z$-axis of the storage cell, the UCN spins start to precess about
%magnetic field $B_0$ at their Larmor frequency.  Then a radio
%frequency pulse at the Larmor frequency of the neutron is applied to
%flip the spin by $90\degree$. The applied electric field is in a
%collinear orientation to $B$.  For parallel orientation of $E$ and
%$B$, the rotational frequency becomes
Using a series of loadings of the cell with neutrons, and application
of magnetic pulses followed by polarized UCN detection for each cell
loading, the spin-precession (Larmor) frequency of the neutrons is
determined.
\begin{equation}\label{my_first_eqn}  
    h\nu^{\uparrow\uparrow}=2\mu_nB+2d_nE
\end{equation}
where the arrows are meant to indicate the parallel orientation of the $B$ and $E$ fields.  The electric field direction is then reversed and the spin-precession frequency measured again
\begin{equation}\label{my_first_eqn}  
    h\nu^{\uparrow\downarrow}=2\mu_nB-2d_nE.
\end{equation}
By taking the difference of the two frequency measurements, the nEDM
$d_n$ may be deduced.
%When the value of nEDM is zero, the
%neutron spin orientation will be anti-parallel to the initial spin
%orientation but the spin orientation will no longer antiparallel for
%non-zero nEDM. A fully magnetized Fe-foil is used to detect the
%neutron spin while transmitting the neutron from neutron chamber which
%is used to get information about spin orientation. The probability of
%neutron transmission is proportional to the neutron spin projection on
%the preferred direction of the Fe-foil. A neutron counter is used to
%detect the transmitted UCNs. A spin flipper is used to flip the
%remaining UCNs in the storage chamber in the preferred direction and
%then they are passed through the neutron counter.  An incident for the
%nEDM is determined from the ratio of the counting rates.
A serious concern is that if the magnetic field drifts during the
frequency measurement, the difference of the two measurements will
suffer from a systematic error.  Even in the best magnetic
environment, field drifts of 1-10~pT over the $\sim$100~s (per UCN
fill) measurement period are likely.  In order to make a correction
that is more precise than neutron counting statistics, precision
magnetometers must be used to measure the magnetic field, correcting
these drifts at the level of 10-20~fT level over the same
period~\cite{doe:website2}.

\section{ Magnetometry Impact on nEDM}

%The precise measurement and control of magnetic fields and magnetic
%field fuctuations is important for experiments searching for a
%permanent electric dipole moment (EDM) of the neutron, hence it is one
%of the main factors limiting the accuracy.

Our collaboration is developing comagnetometers based on $^{199}$Hg
and $^{129}$Xe.  My work is on the development of alkali atom (Rb or
Cs) magnetometers that will be placed around the nEDM measurement
cell.  In this section I discuss the important properties of the
magnetometry strategy for the nEDM experiment which motivate my work
on the alkali atom magnetometers.

\subsection{Comagnetometer}

To correct for magnetic field drifts, the previous best experiment
(done at ILL) developed a $^{199}$Hg
comagnetometer~\cite{bib:green,bib:baker}.  The ILL EDM spectrometer
was moved to PSI and several improvements to the comagnetometer system
were made~\cite{bib:hgbetter}.

In the comagnetometer, optical pumping is used to polarize a vapor of
mercury atoms which are then leaked into the nEDM measurement cell
with the neutrons.  After the application of a $\pi/2$ pulse, the
atoms start to precess freely around $B$.  The polarized mercury atoms
precess in the same volume as the neutrons, hence probing the same
space and time-averaged magnetic field.  Circularly polarized probe
light interacts with precessing atoms and the modulation of light
transmission occurs at the Larmor frequency of the atoms.  The
statistical precision can correct 1-10 pT drifts in $B$ to the
required level of precision~\cite{bib:hgbetter}.


\subsection{Systematic errors and False EDM's}

While the comagnetometer is excellent at normalizing slow magnetic
field drifts, it was discovered relatively recently that it introduces
a potentially devastating systematic error which arises in part due to
magnetic inhomogeneities~\cite{bib:gp1}.  When precessing particles
(neutrons or $^{199}Hg$ atoms) are confined in the measurement cell in
the presence of a magnetic inhomogeneity, there is an electric-field
dependent shift on their measured resonant frequency.  Since it is
dependent on $E$, this gives rise to a false EDM signal.

The effect is easiest to understand by considering transverse fields
originating from the gradient of a slightly non-uniform $\vec{B}$
field in the axial direction ($\partial B_z/\partial z$).  Because
$\vec{\nabla}\cdot\vec{B}=0$, this gives rise to a nonzero radial
component to the field $B_r$.  In the presence of $E$, the species
experience an additional magnetic field their rest frame
\begin{equation}
  \vec{B}_v=\frac{\vec{v}\times\vec{E}}{c^2}
  \label{equation:phase effect}
\end{equation}
where $\vec{v}$ is the velocity vector.  If we consider a particular
neutron or atom that bounces around the edge of the cell, circulating
in the horizontal plane, $\vec{B}_v$ will also be oriented in the
horizontal direction.  In the rest frame of the particle, this will
give rise to a circulating magnetic field.  The rotation frequency of
radial field is the same frequency as particles move in the EDM cell.
The rotating field, although generally far off-resonance with the NMR
frequency of the particle, nonetheless induces a tiny shift on the
measured resonant frequency.  The shift induced by the far
non-resonant field is known as a Ramsey-Bloch-Siegert
shift~\cite{bib:ramsey,bib:bloch-siegert}, and its strength goes as
the square of the rotating field to leading order.  The cross-term in
the square leads to the problematic frequency shift that reverses sign
with $E$, hence giving a false EDM.

Both the $^{199}$Hg atoms and the neutrons experience this effect.  In
the case of the UCN, they are moving so slowly that the process is
adiabatic, and is therefore related to geometric phases in the same
way as Berry's phase.

But the false EDM is generally larger for the $^{199}$Hg atoms because
of their larger (thermal) velocity.  The method developed at ILL to
correct for this problem uses the height difference between the UCN
and comagnetometer atoms to measure the vertical gradient by the
difference of their frequencies compared to the ratio of gyromagnetic
ratios (which has been measured very
precisely~\cite{bib:gyro})~\cite{bib:baker,bib:pendlebury-revised}.

Our collaboration is pursuing multiple alternate ways of coping with
the false EDM.  One of these is the dual comagnetometer concept based
on $^{199}$Hg and $^{129}$Xe.  By taking a particular combination of
the $^{199}$Hg and $^{129}$Xe precession frequencies, the false EDM
can be cancelled out.  Another method we are pursuing is to use two
separate EDM measurement cells that are stacked vertically.  By
comparing precession frequencies in the top and bottom cells, the
vertical gradient can also be accessed.

Finally, we plan to measure gradients by placing a number of precision
alkali magnetometers around the nEDM cell.  It is this application
that relates most to this thesis.  This method of gradient
determination has been shown by the PSI group to correctly sense the
false EDM of Hg~\cite{bib:afach2015}.

\subsection{Internal alkali atom magnetometers}

Based on the considerations above, the alkali atom magnetometers
surrounding the nEDM measurement cells have two principal goals:
\begin{enumerate}
\item Determine the magnetic field to a statistical uncertainty
  competitive with the Hg comagnometer over the timescale of the
  frequency cycle measurement,  i.e.~measure to the statistical
  precision of 10-20~fT in 100~s.  The field measurement should also
  be systematically robust, i.e. there should be no drift of the
  magnetometer reading uncorrelated with changes in $B$.
\item Determine the homogeneity of the magnetic field, with a
  particular focus on the largest gradient term in the false EDM,
  which arises due to $\partial B_z/\partial z$.  The typical scale of
  the gradient is expected to by 0.1-1~nT/m in the experiment.  Drifts
  in the gradient should also be considered.
\end{enumerate}
This thesis relates primarily to the first goal of developing a
magnetometer system and characterizing carefully its precision and
reducing the potential for any possible systematic errors or drifts.
This is based on a prototype system previously reported in
Ref.~\cite{MARTIN201561}, but with several improvements.  The
technology relies on nonlinear magneto-optical rotation (NMOR) of the
plane of polarization of incident light in an atomic vapour, and it
will be discussed in more detail in the next chapter.

%In order to avoid any systematic effects, it becomes necessary to
%carefully investigate the magnetic field in the UCN storage chamber.
%Toward this end, all optical atomic magnetometers will be used in the
%nEDM experiment at TRIUMF.  These magnetometers, like the mercury
%co-magnetometer, are scalar so they only measure the magnitude of the
%magnetic field.  The atomic magnetometers will be placed in an array
%surrounding the precession chamber. This arrangement can be used to
%resolve the multipolarity of field perturbances.



% Superconducting
%Quantum Interference Device (SQUID) is one type of highly sensitive
%commercially available magnetometer. These magnetometers rely on the
%superconducting property of trapped magnetic flux. Because of this,
%they must be operated at cryogenic temperatures. SQUID magnetometers
%are sensitive to the 0.01 nG level \cite{doi:10.1063/1.3491215}, but
%keeping a device at extremely low temperatures would be impractical
%for the nEDM experiment. Spin Exchange Relaxation-Free (SERF)
%magnetometers boast greater sensitivity than SQUID magnetometers with
%SERF magnetic sensitivity reaching the 10 fG
%level\cite{doi:10.1063/1.3491215}. Their limited dynamic range is
%problematic for non-zero measurements.

The technology is very similar to devices developed for the planned
Munich-ILL nEDM experiment~\cite{mythesis}.  It is somewhat
technically different from the PSI nEDM experiment, which uses radio
frequency (RF) atomic magnetometers, or Mx
magnetometers~\cite{Groeger2006}.

An advantage of our strategy is that our NMOR magnetometers require
only light and atoms to operate.  In this sense our technique is
``all-optical'' i.e.~requiring no RF.  A concern with our technology
choice is whether it can reach the required statistical precision and
a principal result of this thesis is that it can.  Aside from this, we
also pursued a strategy being developed by the PSI and Munich group
which uses free-induction decay, where all pumping (either light or
RF) is switched off during the precision frequency measurement phase.

%The basic principle and sensitivity
%of these magnetometer is same as all-optical atomic magnetometer. A
%drawback is that the alignment of the atoms crucial to atomic
%magnetometry are achieved using radio frequency magnetic fields, which
%again perturb the magnetic environment, however only at high
%frequency.  All-optical atomic magnetometers provide a viable solution
%in combining the perks of each of the previous devices. The most
%important advantage of using such all-optical atomic magnetometers is
%that only a polarized laser beam is used to interact with the
%magnetometer instead of using electrical cables which reduces the risk
%of additional sources of magnetic fields to the UCN storage chamber.


%Furthermore, Cs based magnetometers are low maintenance and
%they don't have to be cooled with liquid He down to 4 K such as SQUID
%magnetometer.  The sensitivity of Cs magnetometers can even surpass
%the sensitivity of SQUID.


%In this Master's thesis, an introduction into the concept of full
%optical magnetometry will be presented. This will be continued with a
%status update, the presentation of measurements with our highly
%sensitive Rb magnetometer at The University of Winnipeg.

The principal contribution of this thesis work is therefore studies of
the precision and long-term stability of all-optical NMOR-based
magnetometers, operated using a free-induction-decay technique.

