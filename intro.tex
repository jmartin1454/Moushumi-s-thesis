\chapter{Introduction\label{ch:intro}}

This chapter will provide some information on why the neutron electric
dipole moment (nEDM) is interesting to measure.  Finding a nonzero
nEDM would answer questions regarding the matter-antimatter or baryon
asymmetry of the universe (BAU).  The measuremet principle of nEDM
experiment and the origin of systematic effects causing false EDM's
will be explained.  This will lead to a discussion on the importance
of magnetometry on nEDM experiment.

\section{CP violation and the Standard Model}

Charge conjugation (C), Parity (P) and Time-reversal (T) symmetry are
discrete symmetries in physics.  C-symmetry describes the symmetry of
physical laws under a particle-antiparticle transformation.
P-symmetry describes the inversion of spatial coordinates and
T-symmetry the direction of time.

CPT-symmetry is believed to be a good symmetry of nature because all
relativistic quantum field theories are invariant under successive
application of the three discrete symmetries.  This is known as the
CPT theorem.

Parity symmetry was discovered to be violated in the weak interaction
through observations of spin correlations in beta-decay.  This is due
to neutrinos having a particular handedness.  This problem can be
fixed proposing that CP-symmetry is a good symmetry of nature.
However, CP-symmetry has also been found to be violated in weak decays
of kaons and B-mesons.

The nEDM is an observable that if found to be non-zero, would indicate
a violation of T-symmetry.  Because of the CPT theorem this is
equivalent to CP-violation.  The current best measurement of the nEDM
gives the upper bound $|d_n|<3.0\times
10^{-26}~e\cdot$cm~\cite{PhysRevLett.97.131801,PhysRevD.92.092003}.
We now discuss how this result impacts physics within and beyond the
standard model.

CP violations in the standard model are arising from two separate
sources.  The first source is found in the strong interaction,
described by the $\theta$ term in the quantum chromodynamics (QCD)
Lagrangian.  This CP violating term in the QCD Lagrangian violates
both parity and time symmetry and induces a nEDM of
$|d_n|\sim-(0.9-1.2)\times
10^{−16}\theta~e\cdot$cm~\cite{bib:chuppetal} with $\theta$ being a
dimensionless parameter of the standard model.  A combination of nEDM
and Hg-EDM measurements limit the parameter to be very small
$\theta\lesssim 10^{-10}$.  The reason for the smallness of $\theta$
is currently unknown and this is sometimes called the strong CP
problem.  The second source of CP violation in the SM arises from a
complex phase in the Cabibbo-Kobayashi-Maskawa (CKM)
matrix\cite{PhysRevLett.10.531}.  The phase is responsible for CP
violation in K and B meson decays.  the CP violation within the CKM
matrix predicts $|d_n| = 10^{-31} - 10^{-32}~e\cdot$cm, well below the
current best experimental limit stated above.
%This is known as the
%weak CP-problem since it arises through W boson
%exchange~\cite{PhysRevLett.82.904}.


\section{New Physics and the Baryon asymmetry}

The T-violating nEDM is considered to be a promising probe for physics
beyond the Standard Model, a broad variety of
scenarios~\cite{bib:pospelov}.  Some theories try to make a consistent
description of the nEDM and baryogenesis, the origin of the baryon
asymmetry in the universe~\cite{bib:chuppetal}.

According to the Big Bang theory, matter and antimatter have been
created in equal amounts in the early universe. But the present
universe is overwhelmingly made up of matter rather than
anti-matter.
% The ratio of baryonic matter to photons can be expressed
%as
%\begin{equation}
%  \frac{n_B}{n_\gamma}=10^{-10}
%\end{equation}
%where $n_B$ is the difference between baryons and anti-baryons number
%and $n_\gamma$ is the number of photons in the Cosmic Backgorund.
The amount of baryon asymmetry generated in the standard model is much
smaller than current observations, in the scenario of electroweak
baryogenesis~\cite{bib:morrissey}.

The Sakharov conditions describe a way to explain the evolution of a
baryon asymmetry from an initial symmetric
condition~\cite{budker2013optical,PhysRevLett.10.531}.
\begin{itemize}
    \item Baryon number violation.
    \item Violation of C-symmetry and therefore CP.
    \item Interactions away from thermal equilibrium
\end{itemize}
Although all of these ingredients are available in the standard model
in principle, the underprediction of electroweak baryogenesis
motivates extensions to the standard model to increase the amount of
CP violation in an attempt to fix electroweak baryogenesis.  These
same new phyisics scenarios often predict a non-zero nEDM because of
the increased CP violation.


\section{Neutron electric dipole moment and CP violation }  

Neutron has an intrinsic electric dipole moment (EDM).  The neutron
EDM is a measure for the distribution of positive and negative charges
inside the neutron~\cite{bib:chuppetal}.  The intrinsic EDM of neutron
interacting with external magnetic and electric fields can be
described by the following Hamiltonian:
\begin{equation}\label{my_first_eqn}  
  H=-\vec{\mu}_n\cdot\vec{B}-\vec{d}_n\cdot\vec{E}
\end{equation}
where $\mu_n$ is the magnetic moment of the neutron interacting with
the magnetic field $B$, and $d_n$ is the electric dipole moment of the
neutron interacting with the electric field $E$.  Since the neutron
spin is an axial vector and the electric dipole moment vector is a
scalar vector, both vectors behave differently under P- or
T-transformations.  The orientation of the electric dipole moment
changes under P operation but leaves the magnetic moment unchanged. On
the other hand, T-operation affects the spin vector but leaves the
electric dipole moment unchanged.  Because of the conservation of CPT
symmetry, a non-zero electric dipole moment of the neutron would be a
violation of parity (P) and time-reversal (T) symmetry.

\section{nEDM measurement principle}

Ramsey's method of separated oscillatory fields \cite{bib:ramsey} is
used to extract the nEDM.  In the nEDM experiment, ultracold neutrons
(UCN) whose spins are oriented along a uniform magnetic field are
stored in a chamber.  An electric field is applied parallel to the
magnetic field.
%After applying the magnetic field $B_0$ along the
%$z$-axis of the storage cell, the UCN spins start to precess about
%magnetic field $B_0$ at their Larmor frequency.  Then a radio
%frequency pulse at the Larmor frequency of the neutron is applied to
%flip the spin by $90\degree$. The applied electric field is in a
%collinear orientation to $B$.  For parallel orientation of $E$ and
%$B$, the rotational frequency becomes
Using a series of loadings of the cell with neutrons, and application
of magnetic pulses followed by polarized UCN detection for each cell
loading, the spin-precession (Larmor) frequency of the neutrons is
determined.
\begin{equation}\label{my_first_eqn}  
    h\nu^{\uparrow\uparrow}=2\mu_nB+2d_nE
\end{equation}
where the arrows are meant to indicate the parallel orientation of the $B$ and $E$ fields.  The electric field direction is then reversed and the spin-precession frequency measured again
\begin{equation}\label{my_first_eqn}  
    h\nu^{\uparrow\downarrow}=2\mu_nB-2d_nE.
\end{equation}
By taking the difference of the two measurements, the nEDM $d_n$ may
be deduced.
%When the value of nEDM is zero, the
%neutron spin orientation will be anti-parallel to the initial spin
%orientation but the spin orientation will no longer antiparallel for
%non-zero nEDM. A fully magnetized Fe-foil is used to detect the
%neutron spin while transmitting the neutron from neutron chamber which
%is used to get information about spin orientation. The probability of
%neutron transmission is proportional to the neutron spin projection on
%the preferred direction of the Fe-foil. A neutron counter is used to
%detect the transmitted UCNs. A spin flipper is used to flip the
%remaining UCNs in the storage chamber in the preferred direction and
%then they are passed through the neutron counter.  An incident for the
%nEDM is determined from the ratio of the counting rates.
A serious concern is that if the magnetic field drifts during the
measurement, the formulae above will be invalid.
%

\section{Origin of False EDM}

The dominant sources of systematic error for all previous or proposed
EDM experiments come from magnetic field instability (uncorrelated
with the electric field E), and magnetic field inhomogeneity through
the geometric phase effect~\cite{PhysRevA.70.032102}.  The effect is
easiest to understand by considering transverse fields originating
from the gradient of the uniform $B_0$ field in the axial direction
($\partial B_{0z}/{\partial z}$).  Now in the presence of electric
field E, the experienced additional magnetic field by an UCN particle
moving at a speed v on it’s own rest frame can be written as,
\begin{equation}
  B_v=\frac{v\times E}{c^2}
  \label{equation:phase effect}
\end{equation}
Since the particle moves in the EDM cell, these radial fields also
rotate. The rotation frequency of radial field is the same frequency
as particles move in the EDM cell.  Therefore, the transverse rotating
field induce a phase shift on the resonant frequency from the central
value.  This phase shift is known as Ramsey-Bloch-Siegert shift which
is proportional to the electric-field strength and therefore mimics an
EDM signal.  To analyze this effect, the field rotations may be
represented in terms of a perturbation on the precession phase. The
phase is shifted in 2nd order, resulting in a geometric phase effect.
False EDM effects arise from the interplay between the radial
component of the applied $B_0$ field and $B_v$ in the 2nd order
perturbation.

\section{ Magnetometry Impact on nEDM}
The precise measurement and control of magnetic fields and magnetic field fuctuations is important for experiments searching for a permanent electric dipole moment (EDM) of the neutron, hence it is one of the main factors limiting the accuracy.  For that purpose, our collaboration
is planning to use dual species comagnetometer inside the UCN cell and
an array of highly sensitive all optical atomic magnetometer
surrounding the cell. Cesium/Rubidium is used in a set of external
magnetometers surrounding the storage chamber.

\subsection{Comagnetometer}
To date,~$^{199}Hg$ is the only atomic element that has been used as a co-magnetometer for a neutron EDM experiment\cite{PhysRevLett.97.131801}\cite{PhysRevLett.102.101601}.  polarized mercury atoms precess in the same volume as the neutrons, hence probing approximately the same space and time-averaged magnetic field.  Comgnetometry is an ideal tool to correct for field drifts.  So 1-10 pT drifts in $B_0$  may be corrected using the comagnetometer technique, setting a goal magnetic stability for the $B_0$  field generation system in a typical nEDM experiment\cite{PhysRevLett.97.131801}\cite{afach:in2p3-01062292}. In nEDM experiment, optical pumping is used to polarize the vapor of mercury atoms. Since the comagnetometer operation is synchronous with the nEDM measurement, injecting polarized mercury atoms and  neutron filling into the precession chamber occurs simultaneously. After the application of a $\pi/2$ pulse, the atoms start to precess freely around $B_0$. When a circularly polarized probe beam interact with precessing atoms the modulation of light intensity occurs and after further analysis of  this effect we get information about Larmor frequency of the atoms. A false nEDM signal may arise due to a combination of a magnetic field gradient $\partial B_{z}/{\partial  z}$ and motion in the electric field when species (neutrons and $^{199}Hg$ atoms) are confined in the measurement cells. In order to avoid the false nEDM problem, our proposed comagnetometer will be based on $^{199}Hg$ and $^{129}Xe$. Xe comagnetometer requires mTorr pressure of pure,
highly polarized Xe in the nEDM cell.

\subsection{Internal magnetometer}

In order to avoid any systematic effects, it becomes necessary to carefully investigate the magnetic field in the UCN storage chamber.  Toward this end, all optical atomic magnetometers will be used in the nEDM experiment at TRIUMF.  These magnetometers, like the mercury co-magnetometer, are scalar so they only measure the magnitude of the magnetic field.  The atomic  magnetometers will be placed in an array surrounding the precession chamber. This arrangement can be used to resolve the multipolarity of field perturbances. Superconducting Quantum Interference Device (SQUID) is one type of highly sensitive commercially available  magnetometer. These magnetometers rely on the superconducting property of trapped magnetic flux. Because of this, they must be operated at cryogenic temperatures. SQUID magnetometers are sensitive to the 0.01 nG level \cite{doi:10.1063/1.3491215}, but keeping a device at extremely low temperatures would be impractical for the nEDM experiment. Spin Exchange Relaxation-Free (SERF) magnetometers boast greater sensitivity than SQUID magnetometers with SERF magnetic sensitivity reaching the 10 fG level\cite{doi:10.1063/1.3491215}. Their limited dynamic range is problematic for non-zero measurements. The most similar in sensitivity to NMOR magnetometers are radio frequency (RF) atomic magnetometers, or MX magnetometers~\cite{Groeger2006}. The basic principle and sensitivity of these magnetometer is same as all-optical atomic magnetometer. A drawback is that the alignment of the atoms crucial to atomic magnetometry are achieved using radio frequency magnetic fields, which again perturb the magnetic
environment, however only at high frequency.  All-optical atomic magnetometers provide a viable solution in combining the perks of each of the previous devices. The most important advantage of using such all-optical atomic magnetometers is that only a polarized laser beam is used to interact with the magnetometer instead of using electrical cables which reduces the risk of additional sources of magnetic fields to the UCN storage chamber. Furthermore, Cs based magnetometers are low  maintenance and they don't have to be cooled with liquid He down to 4 K such as SQUID magnetometer.  The sensitivity of Cs magnetometers can even surpass the sensitivity of SQUID.

In this Master's thesis, an introduction into the concept of full
optical magnetometry will be presented. This will be continued with a
status update, the presentation of measurements with our highly
sensitive Rb magnetometer at The University of Winnipeg.

