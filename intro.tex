\chapter{Introduction\label{ch:intro}}
\small
 \section{Neutron electric dipole moment and CP violation } 
 Neutron has an intrinsic electric dipole moment (EDM). The neutron EDM is a measure for the distribution of positive and negative charges inside the neutron. Intrinsic EDM of neutron interacting with external magnetic and electric fields can be described by the following Hamiltonian:
\begin{equation}\label{my_first_eqn}  
  H=-\mu_n.B-d_n.E
\end{equation}
where $\mu_n$ is the magnetic moment of the neutron interacting with the magnetic field B,and $d_n$ is the electric dipole moment of the neutron interacting with the electric field E.
\begin{table}[h]
\centering
\begin{tabular}{|l| l| l |l |}
\hline

\textbf{} & \textbf{C} & \textbf{P}  & \textbf{T}\\
\hline
$\mu_n$ & - & + & - \\
\hline
$d_n$  & -  & + & -  \\
\hline
$B$    & -  & + & -  \\
\hline
$E$    & -  &  - & +  \\
\hline
\end{tabular}
\caption{Symmetry properties of different components of the EDM Hamiltonian}
\end{table}
Table 1.1 describes the behavior of different component of Hamiltonian under discrete symmetries. According to this, the first term( $\mu_n.B$) is CP-even and T-even and the second term ($d_n.E$) is CP-odd and T-odd. Because of the conservation of CPT symmetry, A non-zero electric dipole moment of the neutron would be a violation of parity (P) and time-reversal (T) symmetry.

\section{nEDM measurement principle (Ramsey's method of separated oscillatory fields)}
\bigskip
In the nEDM experiment, spin-polarized ultracold neutrons, oriented along a uniform field $B_0$ stored into a neutron storage chamber. After applying the magnetic field $B_0$ along the z-axis of the storage cell, the ultra-cold neutrons start to precess about magnetic field $B_0$ at their Larmor frequency.  Then a radio frequency pulse at the Larmor frequency of the neutron is applied to flip the spin by $90 \degree$. The applied electric field is in a collinear orientation to B.
For parallel orientation of E and B, the rotational frequency becomes
\begin{equation}\label{my_first_eqn}  
    h{\omega_0}^ {\uparrow\uparrow}=2\mu_n {B}^{\uparrow\uparrow}+ 2 d_nE
\end{equation}
            
and for the antiparallel orientation of E and B, the rotational frequency is given by
\begin{equation}\label{my_first_eqn}  
    h{\omega_0}^ {\uparrow\downarrow}=2\mu_n {B}^{\uparrow\downarrow}- 2 d_nE
\end{equation}

where $\mu_n$ denotes the neutron magnetic moment, and $d_n$ denotes the hypothetical electric dipole moment
When the value of nEDM is zero, the neutron spin orientation will be anti-parallel to the initial spin orientation but the spin orientation will no longer antiparallel for non-zero nEDM. A fully magnetized Fe-foil is used to detect the neutron spin while transmitting the neutron from neutron chamber which is used to get information about spin orientation. The probability of neutron transmission is proportional to the neutron spin projection on the preferred direction of the Fe-foil. A neutron counter is used to detect the transmitted UCNs. A spin flipper is used to flip the remaining UCNs in the storage chamber in the preferred direction and then they are passed through the neutron counter. An incident for the nEDM is determined from the ratio of the counting rates.
\section{Comagnetometer}
\section{Origin of False EDM}
\section{Internal Magnetometry Impact on nEDM}
