\chapter{Conclusion\label{ch:conclusion}}
\section{Summary of Results of the Thesis}


The study based on the adjustment of the pump and probe time discussed in Section \ref{sec:optimization} allowed us to measure the field faster. Previously  each measurement cycle was about 1 s but now the optimized cycle time which include pump and probe time is about 0.35 s which allowed us to take the field measurement about 3 times faster.
  

In order to find out any correlation between coil current drift and magnetic field drift the stability of coil current further should be studied very carefully again now
  that degaussing system seems to have improved.  Fluctuations should
  be easily observable.  Improvements to data acquisition system are easy and should be pursued.  Larger changes in coil current were
  easily observable and seemed to agree with expectation based on the
  measured coil constant.
  
  
It can be concluded for the study discussed in Section \ref{sec:temperature}  that  magnetometer drift compared with drifts of temperature is $\lesssim 5$~pT/\degree{C}.

Studies regarding degaussing (Section \ref{sec:degaussing}), which in part tell the story of our
  degaussing development and learning.  
  
From Section \ref{sec:degauss-near-zerofield} it can be concluded that we set up the degaussing system and tested mainly sample rate (related
to number of oscillations) stated to be important in Ref.~\cite{doi:10.1063/1.2713433}.  We found that if the innermost shield
was already degaussed that additional poor or rapid degaussing did not mar
the field quality as badly as we expected.

 The conclusion of the study discussed in Section \ref{sec:three-degauss} regarding initial operations at non-zero field, and Ramp field to large values, without degaussing was ba.  After degaussing  the  the direction of the field drift changed but the field was unstable about  at the same level as before degaussing. 
    
      
The conclusion of the another degaussing study, reported in Section~\ref{sec:outermosts-shield-degauss}, in which degaussing additionally
      all three outermost shields in an unoptimized system was
      done can be drawn that the field quality seemed to improve after degaussing. The field drift reduced by a factor of 4.
 
  From the studies discussed in section \ref{sec:tuning} regarding laser locking and tuning, and the requirements on
  tune stability can be found that
    lock point or laser seems to drift over time, which could be due to the polarizing beam splitter cube used in the DAVLL system.  A drift of
     about 120 pT was observed where statistical error gets worse.  We 
     demonstrated conclusively that this level of drift cannot be due to a drift in the tune of the laser by some additional studies where the tune was purposely varied. In order to understand 
   whether the drift of lock points affects measured
      field another study has done by manual locking. We found that lock drift does not affect measured field
      drift very much, but does affect statistical precision of
      magnetometer.

 Studies attempting to push below 1~pT in an individual FID
  measurement by adjusting pump and probe powers and lock-in amplifier
  settings.  This includes adjustment of the pump and probe powers,
  and lock-in amplifier settings.  This study revealed problems in the
  procedures used to determine the precession frequency at high
  precision and suggests avenues for further study.

  Future work is to finalize these studies in order to further reduce
  the errors.
 Finally, I showed my studies which revealed a way to use FID
  mode to measure transverse fields.  Further work is required to push
  to nT-scale transverse fields relevant for nEDM experiments.
%for typ.~unmeasurable
%  gradients that enter $\delta_T$ correction in Hg-n signals in nEDM
%  experiments.

\section{Recommended future work for the Rb magnetometer studies}

A finite element analysis simulation of the z-coil in the current configuration should be carried out.  The homogeneity of z-coil in this shielding configuration is not  measured with endcaps of innermost shield removed.  Coupling to second to innermost shield is also believed to be a problem and the degree of coupling could be calculated in such a simulation.

The homogeneity measurement of z-coil in shield  has been done before basic degaussing system which is slightly different than our present degaussing system. A further study could be done to measure homogeneity of this z-coil with our present degaussing system.


Study magnetometer performance with self-shielded coil inside the shield instead of our present z-coil. The idea  is to study whether decoupling the coil from the shield has any impact on long-term stability.

The data fitting procedure of field measurement shows a sensitivity to the lock-in time constant and reference frequency which has been discussed in Section \ref{sec:reference-frequency}.  In the future removing the lock-in amplifier from data  acquisition system of field measurement could be a better idea for avoiding such systematic error. The signal could be collected directly from photodiode and use digital and/or analogue filters for further data analysis  instead of demodulating the FID signal through lock-in amplifier.          

For nEDM experiment, the operated magnetic field is 1 $\mu$T and if our magnetometer could sense transverse field in the range of 1~nT, it would be a new application of such magnetometry techniques. So another important future study could be done to find out how precisely the transverse field measurement can be done with
this magnetometry system. 
 



\section{Farther future:  implementation in the nEDM experiment at TRIUMF}

The work presented in this thesis is based on measuring magnetic field stability and homogeneities which is a part of the TUCAN's future nEDM measurement at TRIUMF. Measuring nEDM within the sensitivity level of $10^{-27}~e\cdot$cm is the goal for this experiment. Study magnetic field stability is very crucial in order to avoid systematic effects (discussed in Chapter \ref{ch:intro}) in nEDM experiment. %because the best previous nEDM measurements showed that 

A fiber-coupled multiple magnetometer system will be implemented into the nEDM system, which is based on this work.  It will lead to a determination of both field stability and gradients to the required levels.



