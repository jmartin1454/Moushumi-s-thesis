\chapter{Conclusion\label{ch:conclusion}}
\section{Summary of Results of the Thesis}
\begin{itemize}
\item Measurements of magnetic fields over long timescales using FID mode.

Conclusion: fields drift over time.  Question: how much is due to
 magnetometer drift vs. other sources?

\item The study based on the adjustment of the pump and probe time Allowed us to measure faster.e the field faster.
  
Conclusion: Allowed us to measure faster.
\item drift in coil current Conclusion: should be studied again now
  that degaussing system seems to have improved.  Fluctuations should
  be easily observable.  Improvements to data acquisition system are
  easy and should be pursued.  Larger changes in coil current were
  easily observable and seemed to agree with expectation based on the
  measured coil constant.
\item Magnetometer drift compared with drifts of temperature.
  Conclusion: drift is $\lesssim 5$~pT/\degree{C}.
\item Studies of degaussing, which in part tell the story of our
  degaussing development and learning.  This includes studies of:
%with multiple goals, telling the story of
%  our degaussing
  \begin{itemize}
    \item the degaussing setup and testing it near zero field
    
Conclusion:we set up the system, we tested mainly sample rate (related
to number of oscillations) stated to be important in Thiel et
al.\cite{doi:10.1063/1.2713433} We found that if the innermost shield
was already degaussed that additional poor/rapid degauss did not mar
the field quality as badly as we expected.
    \item initial operations at non-zero field, and Ramp field to
      large values, without degaussing was bad.  After degaussing was
      ok but still not great.
    \item final degaussing procedure, in which degaussing additionally
      all three outershield shields in an unoptimized system was
      studied.  The field quality seemed to improve.
  \end{itemize}
\item Studies of laser locking and tuning, and the requirements on
  tune stability
  \end{itemize}
again multiple goals:
 \begin{itemize}
   \item Lock point or laser seems to drift over time.  See ``Drift is
     about 120 pT'' where statistical error gets worse.  Also would
     lose lock sometimes.  We think this might be due to PBS.
    \item Other concern is whether drift of lock affects measured
      field.  Conclusion: Studied by ``manual locking'' and found not
      to be important.  Lock drift does not affect measured field
      drift very much, but does affect statistical precision of
      magnetometer.
%  \end{itemize}
\item Studies attempting to push below 1~pT in an individual FID
  measurement by adjusting pump and probe powers and lock-in amplifier
  settings.  This includes adjustment of the pump and probe powers,
  and lock-in amplifier settings.  This study revealed problems in the
  procedures used to determine the precession frequency at high
  precision and suggests avenues for further study.
%But when
%  we improved the statistical precision significantly, we began to run
%  into systematic errors in frequency measurement.  This led us to
%  study additional errors related to lock-in amplifier settings.
  Future work is to finalize these studies in order to further reduce
  the errors
\item Finally, I showed my studies which revealed a way to use FID
  mode to measure transverse fields.  Further work is required to push
  to nT-scale transverse fields relevant for nEDM experiments.
%for typ.~unmeasurable
%  gradients that enter $\delta_T$ correction in Hg-n signals in nEDM
%  experiments.
\end{itemize}
\section{Future work}
\begin{itemize}
\item	FEMM simulation of z-coil 	Homogeneity of z-coil in this shielding configuration is not  measured with endcaps of innermost shield removed.  Coupling to second to innermost shield is also believed to be a problem and the degree of coupling could be calculated in such a simulation.
\begin{itemize}

\item
Can also measure homogeneity of z-coil in shield – it has been done before and I recall that it is homogeneous at $1\% $level when in shield with basic degaussing (not our present degaussing system).
\end{itemize}
\item
Study magnetometer performance with self-shielded coil (Junyao's coil) inside shield instead of our present z-coil. 
\begin{itemize}
\item	The idea here is to show whether decoupling the coil from the shield has any impact on long-term stability.
\end{itemize}
 \item Removing the lock-in amplifier from data  acquisition system of field measurement. The idea is to collect signal directly from photodiode and use digital/ analogue filter for further data analysis  instead of demodulating the FID signal through lock-in amplifier 

\item Transverse fields: find out how well they can be measured with
  this system.

\end{itemize}

\section{Farther Future}
Build nEDM experiment:

The work presented in this thesis is based on measuring magnetic field stability and homogeneities which is a part of the TUCAN's future nEDM measurement at TRIUMF. Measuring nEDM within the sensitivity level of $10^{-27}$ e.cm is the goal for this experiment. Study magnetic field stability is very crucial %because the best previous nEDM measurements showed that 





