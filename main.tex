\documentclass[12pt]{report}
\usepackage{amsmath,amssymb}
\usepackage{gensymb}
\usepackage{lineno}
\usepackage{titlesec}
\usepackage{geometry}
\usepackage{lipsum}
\usepackage{caption}
\usepackage{subcaption}
\newgeometry{
    top=1in,
    bottom=1in,
    outer=1in,
    inner=1in,
}
\usepackage{setspace} \doublespacing
%\titleformat*{\section}{\LARGE\bfseries}
%\titleformat*{\subsection}{\Large\bfseries}
%\titleformat*{\subsubsection}{\large\bfseries}
\usepackage{hyperref}
\hypersetup{%makes links, references, and table of contents hyperlinks within the text, and colours them blue
    colorlinks,
    citecolor=blue,
    filecolor=black,
    linkcolor=blue,
    urlcolor=black
}
\usepackage{graphicx}
\usepackage[english]{babel} 
\usepackage[
backend=biber,
style=numeric,
sorting=none
]{biblatex}
\addbibresource{mybibliography.bib}

\begin{document}
\begin{titlepage}
    \begin{center}
        \vspace*{0.5cm}
        
        {\large \textbf{Highly Sensitive Atomic Magnetometer for
            Neutron Electric Dipole Moment Experiment} }
        \vspace{0.5cm}
        {\large
        
        
        \vspace{1.5cm}
        by\\
        \vspace{0.9cm}
        Moushumi Das
        \vfill
        }
% * <moushumi.das.phy@gmail.com> 2018-09-10T15:56:34.828Z:
%
% ^.
       \vspace{1.5cm}
       A thesis submitted to the Faculty of Graduate Studies of\\ The University of Manitoba\\
       in partial fulfillment of the requirements of the degree of\\
        \vspace{1.2cm}
        MASTER OF SCIENCE
        {\Large
        \vspace{0.8cm}
        
%        \includegraphics[width=0.4\textwidth]{figures/university_logo_}\\
        \large
        Department of Physics and Astronomy\\
        University of Manitoba\\
        Winnipeg
  }
        
    \end{center}
\end{titlepage}



\begin{abstract}
%% Text of abstract
A non-zero neutron electric dipole moment (nEDM) would indicate time reversal and charge-parity violation (T and CP). Many experiments are currently being conducted or planned to measure the nEDM. At present the experimental upper bound on the nEDM is~$3.0 \times~10^{-26}$ e$\cdot$cm. Our collaboration is developing an experiment at TRIUMF to improve the sensitivity to the nEDM by one order of magnitude ($10^{-27}$ e$\cdot$cm). I am proposed to develop a highly sensitive nonlinear magneto-optical rotation (NMOR) based magnetometer which is a crucial tool in the nEDM apparatus, for stability and homogeneity measurements. The NMOR magnetometer is designed to serve in an array of such sensors to form an auxiliary magnetometer system monitoring the stability and uniformity of the magnetic field in nEDM experiment at the TRIUMF. A NMOR magnetometer has been developed and constructed at the University of Winnipeg. In this NMOR system an atomic vapour cell containing natural rubidium with stable isotopes $^{87}Rb$ and $^{85}Rb$ is used. An NMOR resonance occurs when the optical pumping is synchronous with Larmor precession. This causes an atomic vapor to become dichroic, so that subsequent probe light experiences polarization rotation modulated at the same frequency. 

\end{abstract} 
\pagestyle{plain}
\setcounter{page}{1}
\pagenumbering{roman} 
\tableofcontents
\newpage
\listoffigures
\newpage
%\Large
\setcounter{page}{1}
\pagenumbering{arabic}
% page 1 % 

% Chapter 1:  Introduction
\chapter{Introduction\label{ch:intro}}

This chapter will provide some information on why the neutron electric
dipole moment (nEDM) is interesting to measure.  Finding a nonzero
nEDM would answer questions regarding the matter-antimatter or baryon
asymmetry of the universe (BAU).  The measuremet principle of nEDM
experiment and the origin of systematic effects causing false EDM's
will be explained.  This will lead to a discussion on the importance
of magnetometry on nEDM experiment.

\section{CP violation and the Standard Model}

Charge conjugation (C), Parity (P) and Time-reversal (T) symmetry are
discrete symmetries in physics.  C-symmetry describes the symmetry of
physical laws under a particle-antiparticle transformation.
P-symmetry describes the inversion of spatial coordinates and
T-symmetry the direction of time.

CPT-symmetry is believed to be a good symmetry of nature because all
relativistic quantum field theories are invariant under successive
application of the three discrete symmetries.  This is known as the
CPT theorem.

Parity symmetry was discovered to be violated in the weak interaction
through observations of spin correlations in beta-decay.  This is due
to neutrinos having a particular handedness.  This problem can be
fixed proposing that CP-symmetry is a good symmetry of nature.
However, CP-symmetry has also been found to be violated in weak decays
of kaons and B-mesons.

The nEDM is an observable that if found to be non-zero, would indicate
a violation of T-symmetry.  Because of the CPT theorem this is
equivalent to CP-violation.  The current best measurement of the nEDM
gives the upper bound $|d_n|<3.0\times
10^{-26}~e\cdot$cm~\cite{bib:baker,bib:pendlebury-revised}.
We now discuss how this result impacts physics within and beyond the
standard model.

CP violations in the standard model are arising from two separate
sources.  The first source is found in the strong interaction,
described by the $\theta$ term in the quantum chromodynamics (QCD)
Lagrangian.  This CP violating term in the QCD Lagrangian violates
both parity and time symmetry and induces a nEDM of
$|d_n|\sim-(0.9-1.2)\times
10^{−16}\theta~e\cdot$cm~\cite{bib:chuppetal} with $\theta$ being a
dimensionless parameter of the standard model.  A combination of nEDM
and Hg-EDM measurements limit the parameter to be very small
$\theta\lesssim 10^{-10}$.  The reason for the smallness of $\theta$
is currently unknown and this is sometimes called the strong CP
problem.  The second source of CP violation in the SM arises from a
complex phase in the Cabibbo-Kobayashi-Maskawa (CKM)
matrix\cite{PhysRevLett.10.531}.  The phase is responsible for CP
violation in K and B meson decays.  the CP violation within the CKM
matrix predicts $|d_n| = 10^{-31} - 10^{-32}~e\cdot$cm, well below the
current best experimental limit stated above.
%This is known as the
%weak CP-problem since it arises through W boson
%exchange~\cite{PhysRevLett.82.904}.


\section{New Physics and the Baryon asymmetry}

The T-violating nEDM is considered to be a promising probe for physics
beyond the Standard Model, a broad variety of
scenarios~\cite{bib:pospelov}.  Some theories try to make a consistent
description of the nEDM and baryogenesis, the origin of the baryon
asymmetry in the universe~\cite{bib:chuppetal}.

According to the Big Bang theory, matter and antimatter have been
created in equal amounts in the early universe. But the present
universe is overwhelmingly made up of matter rather than
anti-matter.
% The ratio of baryonic matter to photons can be expressed
%as
%\begin{equation}
%  \frac{n_B}{n_\gamma}=10^{-10}
%\end{equation}
%where $n_B$ is the difference between baryons and anti-baryons number
%and $n_\gamma$ is the number of photons in the Cosmic Backgorund.
The amount of baryon asymmetry generated in the standard model is much
smaller than current observations, in the scenario of electroweak
baryogenesis~\cite{bib:morrissey}.

The Sakharov conditions describe a way to explain the evolution of a
baryon asymmetry from an initial symmetric
condition~\cite{budker2013optical,PhysRevLett.10.531}.
\begin{itemize}
    \item Baryon number violation.
    \item Violation of C-symmetry and therefore CP.
    \item Interactions away from thermal equilibrium
\end{itemize}
Although all of these ingredients are available in the standard model
in principle, the underprediction of electroweak baryogenesis
motivates extensions to the standard model to increase the amount of
CP violation in an attempt to fix electroweak baryogenesis.  These
same new phyisics scenarios often predict a non-zero nEDM because of
the increased CP violation.


\section{Neutron electric dipole moment and CP violation }  

Neutron has an intrinsic electric dipole moment (EDM).  The neutron
EDM is a measure for the distribution of positive and negative charges
inside the neutron~\cite{bib:chuppetal}.  The intrinsic EDM of neutron
interacting with external magnetic and electric fields can be
described by the following Hamiltonian:
\begin{equation}\label{my_first_eqn}  
  H=-\vec{\mu}_n\cdot\vec{B}-\vec{d}_n\cdot\vec{E}
\end{equation}
where $\mu_n$ is the magnetic moment of the neutron interacting with
the magnetic field $B$, and $d_n$ is the electric dipole moment of the
neutron interacting with the electric field $E$.  Since the neutron
spin is an axial vector and the electric dipole moment vector is a
scalar vector, both vectors behave differently under P- or
T-transformations.  The orientation of the electric dipole moment
changes under P operation but leaves the magnetic moment unchanged. On
the other hand, T-operation affects the spin vector but leaves the
electric dipole moment unchanged.  Because of the conservation of CPT
symmetry, a non-zero electric dipole moment of the neutron would be a
violation of parity (P) and time-reversal (T) symmetry.

\section{nEDM measurement principle}

Ramsey's method of separated oscillatory fields \cite{bib:ramsey} is
used to extract the nEDM.  In the nEDM experiment, ultracold neutrons
(UCN) whose spins are oriented along a uniform magnetic field are
stored in a chamber.  An electric field is applied parallel to the
magnetic field.
%After applying the magnetic field $B_0$ along the
%$z$-axis of the storage cell, the UCN spins start to precess about
%magnetic field $B_0$ at their Larmor frequency.  Then a radio
%frequency pulse at the Larmor frequency of the neutron is applied to
%flip the spin by $90\degree$. The applied electric field is in a
%collinear orientation to $B$.  For parallel orientation of $E$ and
%$B$, the rotational frequency becomes
Using a series of loadings of the cell with neutrons, and application
of magnetic pulses followed by polarized UCN detection for each cell
loading, the spin-precession (Larmor) frequency of the neutrons is
determined.
\begin{equation}\label{my_first_eqn}  
    h\nu^{\uparrow\uparrow}=2\mu_nB+2d_nE
\end{equation}
where the arrows are meant to indicate the parallel orientation of the $B$ and $E$ fields.  The electric field direction is then reversed and the spin-precession frequency measured again
\begin{equation}\label{my_first_eqn}  
    h\nu^{\uparrow\downarrow}=2\mu_nB-2d_nE.
\end{equation}
By taking the difference of the two frequency measurements, the nEDM
$d_n$ may be deduced.
%When the value of nEDM is zero, the
%neutron spin orientation will be anti-parallel to the initial spin
%orientation but the spin orientation will no longer antiparallel for
%non-zero nEDM. A fully magnetized Fe-foil is used to detect the
%neutron spin while transmitting the neutron from neutron chamber which
%is used to get information about spin orientation. The probability of
%neutron transmission is proportional to the neutron spin projection on
%the preferred direction of the Fe-foil. A neutron counter is used to
%detect the transmitted UCNs. A spin flipper is used to flip the
%remaining UCNs in the storage chamber in the preferred direction and
%then they are passed through the neutron counter.  An incident for the
%nEDM is determined from the ratio of the counting rates.
A serious concern is that if the magnetic field drifts during the
frequency measurement, the difference of the two measurements will
suffer from a systematic error.  Even in the best magnetic
environment, field drifts of 1-10~pT over the $\sim$100~s (per UCN
fill) measurement period are likely.  In order to make a correction
that is more precise than neutron counting statistics, precision
magnetometers must be used to measure the magnetic field, correcting
these drifts at the level of 10-20~fT level over the same
period~\cite{doe:website2}.

\section{ Magnetometry Impact on nEDM}

%The precise measurement and control of magnetic fields and magnetic
%field fuctuations is important for experiments searching for a
%permanent electric dipole moment (EDM) of the neutron, hence it is one
%of the main factors limiting the accuracy.

Our collaboration is developing comagnetometers based on $^{199}$Hg
and $^{129}$Xe.  My work is on the development of alkali atom (Rb or
Cs) magnetometers that will be placed around the nEDM measurement
cell.  In this section I discuss the important properties of the
magnetometry strategy for the nEDM experiment which motivate my work
on the alkali atom magnetometers.

\subsection{Comagnetometer}

To correct for magnetic field drifts, the previous best experiment
(done at ILL) developed a $^{199}$Hg
comagnetometer~\cite{bib:green,bib:baker}.  The ILL EDM spectrometer
was moved to PSI and several improvements to the comagnetometer system
were made~\cite{bib:hgbetter}.

In the comagnetometer, optical pumping is used to polarize a vapor of
mercury atoms which are then leaked into the nEDM measurement cell
with the neutrons.  After the application of a $\pi/2$ pulse, the
atoms start to precess freely around $B$.  The polarized mercury atoms
precess in the same volume as the neutrons, hence probing the same
space and time-averaged magnetic field.  Circularly polarized probe
light interacts with precessing atoms and the modulation of light
transmission occurs at the Larmor frequency of the atoms.  The
statistical precision can correct 1-10 pT drifts in $B$ to the
required level of precision~\cite{bib:hgbetter}.


\subsection{Systematic errors and False EDM's}

While the comagnetometer is excellent at normalizing slow magnetic
field drifts, it was discovered relatively recently that it introduces
a potentially devastating systematic error which arises in part due to
magnetic inhomogeneities~\cite{bib:gp1}.  When precessing particles
(neutrons or $^{199}Hg$ atoms) are confined in the measurement cell in
the presence of a magnetic inhomogeneity, there is an electric-field
dependent shift on their measured resonant frequency.  Since it is
dependent on $E$, this gives rise to a false EDM signal.

The effect is easiest to understand by considering transverse fields
originating from the gradient of a slightly non-uniform $\vec{B}$
field in the axial direction ($\partial B_z/\partial z$).  Because
$\vec{\nabla}\cdot\vec{B}=0$, this gives rise to a nonzero radial
component to the field $B_r$.  In the presence of $E$, the species
experience an additional magnetic field their rest frame
\begin{equation}
  \vec{B}_v=\frac{\vec{v}\times\vec{E}}{c^2}
  \label{equation:phase effect}
\end{equation}
where $\vec{v}$ is the velocity vector.  If we consider a particular
neutron or atom that bounces around the edge of the cell, circulating
in the horizontal plane, $\vec{B}_v$ will also be oriented in the
horizontal direction.  In the rest frame of the particle, this will
give rise to a circulating magnetic field.  The rotation frequency of
radial field is the same frequency as particles move in the EDM cell.
The rotating field, although generally far off-resonance with the NMR
frequency of the particle, nonetheless induces a tiny shift on the
measured resonant frequency.  The shift induced by the far
non-resonant field is known as a Ramsey-Bloch-Siegert
shift~\cite{bib:ramsey,bib:bloch-siegert}, and its strength goes as
the square of the rotating field to leading order.  The cross-term in
the square leads to the problematic frequency shift that reverses sign
with $E$, hence giving a false EDM.

Both the $^{199}$Hg atoms and the neutrons experience this effect.  In
the case of the UCN, they are moving so slowly that the process is
adiabatic, and is therefore related to geometric phases in the same
way as Berry's phase.

But the false EDM is generally larger for the $^{199}$Hg atoms because
of their larger (thermal) velocity.  The method developed at ILL to
correct for this problem uses the height difference between the UCN
and comagnetometer atoms to measure the vertical gradient by the
difference of their frequencies compared to the ratio of gyromagnetic
ratios (which has been measured very
precisely~\cite{bib:gyro})~\cite{bib:baker,bib:pendlebury-revised}.

Our collaboration is pursuing multiple alternate ways of coping with
the false EDM.  One of these is the dual comagnetometer concept based
on $^{199}$Hg and $^{129}$Xe.  By taking a particular combination of
the $^{199}$Hg and $^{129}$Xe precession frequencies, the false EDM
can be cancelled out.  Another method we are pursuing is to use two
separate EDM measurement cells that are stacked vertically.  By
comparing precession frequencies in the top and bottom cells, the
vertical gradient can also be accessed.

Finally, we plan to measure gradients by placing a number of precision
alkali magnetometers around the nEDM cell.  It is this application
that relates most to this thesis.  This method of gradient
determination has been shown by the PSI group to correctly sense the
false EDM of Hg~\cite{bib:afach2015}.

\subsection{Internal alkali atom magnetometers}

Based on the considerations above, the alkali atom magnetometers
surrounding the nEDM measurement cells have two principal goals:
\begin{enumerate}
\item Determine the magnetic field to a statistical uncertainty
  competitive with the Hg comagnometer over the timescale of the
  frequency cycle measurement,  i.e.~measure to the statistical
  precision of 10-20~fT in 100~s.  The field measurement should also
  be systematically robust, i.e. there should be no drift of the
  magnetometer reading uncorrelated with changes in $B$.
\item Determine the homogeneity of the magnetic field, with a
  particular focus on the largest gradient term in the false EDM,
  which arises due to $\partial B_z/\partial z$.  The typical scale of
  the gradient is expected to by 0.1-1~nT/m in the experiment.  Drifts
  in the gradient should also be considered.
\end{enumerate}
This thesis relates primarily to the first goal of developing a
magnetometer system and characterizing carefully its precision and
reducing the potential for any possible systematic errors or drifts.
This is based on a prototype system previously reported in
Ref.~\cite{MARTIN201561}, but with several improvements.  The
technology relies on nonlinear magneto-optical rotation (NMOR) of the
plane of polarization of incident light in an atomic vapour, and it
will be discussed in more detail in the next chapter.

%In order to avoid any systematic effects, it becomes necessary to
%carefully investigate the magnetic field in the UCN storage chamber.
%Toward this end, all optical atomic magnetometers will be used in the
%nEDM experiment at TRIUMF.  These magnetometers, like the mercury
%co-magnetometer, are scalar so they only measure the magnitude of the
%magnetic field.  The atomic magnetometers will be placed in an array
%surrounding the precession chamber. This arrangement can be used to
%resolve the multipolarity of field perturbances.



% Superconducting
%Quantum Interference Device (SQUID) is one type of highly sensitive
%commercially available magnetometer. These magnetometers rely on the
%superconducting property of trapped magnetic flux. Because of this,
%they must be operated at cryogenic temperatures. SQUID magnetometers
%are sensitive to the 0.01 nG level \cite{doi:10.1063/1.3491215}, but
%keeping a device at extremely low temperatures would be impractical
%for the nEDM experiment. Spin Exchange Relaxation-Free (SERF)
%magnetometers boast greater sensitivity than SQUID magnetometers with
%SERF magnetic sensitivity reaching the 10 fG
%level\cite{doi:10.1063/1.3491215}. Their limited dynamic range is
%problematic for non-zero measurements.

The technology is very similar to devices developed for the planned
Munich-ILL nEDM experiment~\cite{mythesis}.  It is somewhat
technically different from the PSI nEDM experiment, which uses radio
frequency (RF) atomic magnetometers, or MX
magnetometers~\cite{Groeger2006}.

An advantage of our strategy is that our NMOR magnetometers require
only light and atoms to operate.  In this sense our technique is
``all-optical'' i.e.~requiring no RF.  A concern with our technology
choice is whether it can reach the required statistical precision and
a principal result of this thesis is that it can.  Aside from this, we
also pursued a strategy being developed by the PSI and Munich group
which uses free-induction decay, where all pumping (either light or
RF) is switched off during the precision frequency measurement phase.

%The basic principle and sensitivity
%of these magnetometer is same as all-optical atomic magnetometer. A
%drawback is that the alignment of the atoms crucial to atomic
%magnetometry are achieved using radio frequency magnetic fields, which
%again perturb the magnetic environment, however only at high
%frequency.  All-optical atomic magnetometers provide a viable solution
%in combining the perks of each of the previous devices. The most
%important advantage of using such all-optical atomic magnetometers is
%that only a polarized laser beam is used to interact with the
%magnetometer instead of using electrical cables which reduces the risk
%of additional sources of magnetic fields to the UCN storage chamber.


%Furthermore, Cs based magnetometers are low maintenance and
%they don't have to be cooled with liquid He down to 4 K such as SQUID
%magnetometer.  The sensitivity of Cs magnetometers can even surpass
%the sensitivity of SQUID.


%In this Master's thesis, an introduction into the concept of full
%optical magnetometry will be presented. This will be continued with a
%status update, the presentation of measurements with our highly
%sensitive Rb magnetometer at The University of Winnipeg.

The principal contribution of this thesis work is therefore studies of
the precision and long-term stability of all-optical NMOR-based
magnetometers, operated using a free-induction-decay technique.



% Chapter 2:  Magnetometry Literature Review
\chapter{Atomic Magnetometry\label{ch:magnetometry}}


 In the following chapter, there will be given an introduction into the quantum mechanical
structure and interactions of a Rb atom with special focus on the hyperfine structure and the Zeeman splitting of the energy levels in an external magnetic field. In the second section of this chapter, the principle of atom-light interactions including optical pumping
will be explained. After introducing basic concepts of alkali atom quantum mechanics, the phenomenology of nonlinear magneto-optical rotation (NMOR) will be discussed, which is the basis for the operation of an atomic magnetometer as it will be used for the nEDM setup

A possible outline of this Chapter.
\begin{itemize}
\item Atomic structure of Rb

  \begin{itemize}
  \item g.s.and e.s ($L$, $S$)
  
  
  \item fine structure ($J$)
  
  \item hyperfine structure ($F$)
  
  
  \item Selection rules ($\Delta l$, $\Delta J$, $\Delta F$, $\Delta
    M_l$), D1 and D2 lines of Rb
    
  
  \item Doppler broadening and Rb absorption spectrum (of D1 line)
  \end{itemize}
\item Linear and nonlinear magneto-optical rotation (mostly from
  Budker, Orlando, Yaschuk Am J Phys)
  \begin{itemize}
  \item Hypothetical transition and linear magneto-optical rotation
   \item Nonlinear effects in atomic gases: hole-burning and coherent
    population trapping
 
  \item Nonlinear magneto-optical rotation; also show result from
    Budker PRL 1998.
  \end{itemize}
\item Magnetometry using NMOR


  \begin{itemize}
    \item Near zero field.
    \item Nonzero field.
      \begin{itemize}
      \item FM NMOR 
      
     
      \item AM NMOR
       

      
      \item Michi's thesis

 
      \end{itemize}
    \item Sensitivity to tilted fields
    
    
    \item Vector magnetometer
    
    
  \end{itemize}
\item Reminder of what we do, leading into next chapter
\end{itemize}


\section{Atomic structure of Rb }
\label{sec:Rb structure} 
Alkali-metal atoms have single electron in their outer shell. For this reason they are often used for optical pumping. Alkali-metal atoms have a nuclear spin.
 Because of the nuclear spin the energy level of these alkali atom becomes complicated. The ground state structure of the Rb atom is n = 5, L =0~(s-state) and j = 1/2, i.e, $5^2S_{1/2}$. The lowest excited states (L~=1) are the 5p states. According to quantum theory of angular momentum these 5p states have total electron angular momentum J = L + S where L is the orbital angular momentum and S is the electron spin angular momentum. Fine structure splitting occurs due to spin orbit coupling.  The lowest excited states 5P (L~=1, S=1/2) split into  $5^2P_{1/2}$ and $5^2P_{3/2}$  due to the fine structure splitting. The hyperfine structure is a result of the coupling between J and the total nuclear angular momentum I. The total atomic angular momentum F is then given by F~=~J~+I  and the magnitude of F must lie in the range
$J - I \leq F \leq J + I$.
\begin{figure}[h]
\centering
\includegraphics[width=0.7\linewidth]{figures/Rb_structure.JPG}
\caption{ Energy level splitting of the ground state and first excited state of $^{85}$Rb.
The fine structure splits the first excited state into levels with J=1/2 and J=3/2, and the hyperfine
structure further splits the energy levels due to the nonzero nuclear spin.  The splittings are not drawn to scale. \label{fig:Rb_structure}}
\end{figure}

For the ground state of~$^{85}{Rb}$, J = 1/2 and I = 3/2, so F = 1 or F = 2. And  for the excited state $5^2P_{1/2}$ ~  which has nuclear angular momentum I = 3/2 and J = 1/2  the allowed values of F are  1 ~and~ 2. The atomic energy levels are shifted according to the value of F. In quantum mechanics, a set of rules tell us the allowed transition between states which are known as selection rules. The rules are that the total orbital angular momentum change should be $\Delta L= \pm 1$ and the change in magnetic quantum number should be $\Delta M_L= 0,\pm 1$. According to selection rules the transitions from the state L=0 to L=1 are possible. Transmission corresponding to the energy difference between the ~$5^2S_{1/2}$~ and~ $5^2P_{1/2}$ levels of rubidium is termed the D1 line; its wavelength is roughly 795 nm\cite{doe:website}.  The ~$5^2P_{3/2}$~state is separated from the ~$5^2S_{1/2}$~ state by an energy corresponding to 780 nm wavelength, it is called the D2 line. Our Rb atomic magnetometer is based on exciting the D1-line transition by optical pumping.  Linearly polarized laser beam is used to induce transitions of electrons from one energy level to another via optical pumping. In this case, the laser beam is tuned to the transition frequency and of sufficient power to perturb the equilibrium distribution of the ground state energy levels. The gyromagnetic ratio of  $^{85}Rb$ is 4.667415~Hz/nT~\cite{bib:rb-gyro-reference}.
\section{Linear and nonlinear magneto-optical rotation }
In the presence of an axial magnetic field, when a linearly polarized light passes through an atomic medium the plane of polarization rotates. This kind of effect is known as the Faraday effect. Afterwards an resonant enhancement of the Faraday rotation has been discovered by D. Macaluso and O.M Corbino in 1898 which is known as nonlinear magneto-optical rotation (NMOR) or nonlinear Faraday rotation \cite{budker2013optical}. Consider an example system of $ F=~1$ to $F'=~0$ transition (Fig~\ref{fig:Zeemansplitting}).  Linearly polarized light can decompose into two counter-rotating circularly polarized components $\sigma^\pm$. The transferable angular momentum is +1 for left circular polarization, -1 for right circular polarization. 
 
 In the absence of a magnetic field, the $M=\pm 1$ sublevels are
degenerate and the optical resonance frequencies for  the two circular
polarizations coincide. But in the presence of magnetic field the Zeeman sublevels $M=\pm 1$ shift in energy  by an amount $g\mu B/\hbar$ where g is the Lande factor and $\mu$~is the Bohr magneton. This Zeeman splitting therefore leads to a difference between the  resonance frequencies for for $\sigma^+$ and
$\sigma^-$ light. The left circularly polarized light experiences refractive index $n_+$ and the refractive index for right circularly polarized light is $n_-$. In this case, optical rotation arises due to the difference in the refractive index 
\begin{equation}
\label{eq:emc}
\phi = \pi(n_+-n_{-})\frac{l}{\lambda} \\
\end{equation}
where l is the length of the medium traversed and $\lambda$ is the wavelength of light. The difference between $n_+$ and $n_-$ is shown in  Fig.\ref{fig:Faraday} . On resonance, the rotation is related to the Zeeman shift in the $M=\pm 1$ sub levels. For light with spectral
width of the absorption line much smaller than the transition width, and for zero
frequency detuning from the resonance, the optical rotation can be estimated as
\begin{equation}
\phi \approx \frac{(2g\mu B)/ \hbar\tau)}{(1+((2g\mu B)/(\hbar\tau))^2 )}\frac{l}{l_0}
\end{equation}
where $2g\mu/\hbar$ is correlated to Zeeman splitting, $\tau$ is the doppler width of the
absorption line (of order GHz), and $l_0$ is the absorption length in the medium. 
\begin{figure}[h]
\centering
\includegraphics[width=0.75\linewidth]{figures/optical_rotation}
\caption{Illustrative example of F = 1 to $F' = 0$ atomic transition with Zeeman
splitting in the presence of a magnetic field. Image\cite{bib:Budker2002}\label{fig:Zeemansplitting}}
\end{figure}
\begin{figure}[h]
\centering
\includegraphics[width=0.65\linewidth]{figures/farday.jpg}
\caption{The dependence of the refractive index on light frequency detuning
D in the absence (n) and in the presence ($n_\pm$) of a magnetic field. Shown is
the case of $2g\mu B=\hbar\tau$ and a Lorentzian model for line broadening. The
lower curve shows the difference in refractive index for the two circular
polarization components. This is the characteristic spectral profile of Macaluso–Corbino optical rotation. Image\cite{bib:NMOR1998}\label{fig:Faraday}}
\end{figure}
 
In NMOR, the optical properties of the medium are modified by the laser light, resulting in nonlinear effects such as hole-burning and the creation of a coherent dark state.     
Hole burning is the  nonlinear
effect leading to enhanced Faraday rotation. Spectral holes, or Bennett-structures, are dips in the velocity distribution of a group of atoms produced by pump laser beam. The Faraday rotation produced by atoms with such velocity distribution can be thought of as rotation produced by the Doppler distributed atoms without the hole minus the rotation that would have been produced by the pumped out atoms.

Optical pumping is a process by which light modifies the quantum state of the medium it is transmitting through. Coherent population trapping can be described as the pumping of the atomic system in a particular state, the coherent superposition of the atomic states, which is a nonabsorbing state. The exciting radiation creates an atomic coherence, such that the atom's evolution is prepared exactly out of phase with the incoming radiation and no absorption takes place. In the presence of a magnetic field, the atomic alignment
axis created by the coherent population trapping precesses
around the direction of the field with the Larmor frequency.

\section{Magnetometry using NMOR\label{sec:magnetometry literature review}}

nonlinear magneto-optical rotation (NMOR), is a promising technique for a new generation of ultrasensitive atomic magnetometers. NMOR
magnetometers have the advantages of operating near room temperature and of being all optical
(i.e., they do not require magnetic field compensation or excitation). Magnetometers based on nonlinear
magneto-optical rotation (NMOR) feature parallel pump and probe beams and measure
the magnetic field along the direction of beam propagation \cite{bib:Budker2002} , and they can achieve sensitivity on the order of 1 fT/$\sqrt{\text{Hz}}$ \cite{bib:sensitivemagnetometry}. 
 
Several dispersion-like features in the magnetic field dependence of the nonlinear magneto-optic effect
were observed \cite{bib:UltranarrowWidths} in an experiment performed on rubidium atoms contained in a vapor cell with anti-relaxation
coating. The narrowest feature has effective resonance width $\Delta B \sim$ 2.8 mG is the
peak-to-peak separation. The observed nontrivial dependence of the magneto-optic effect on transverse
magnetic fields is discussed.

Budker et al. \cite{bib:FMNMOR} demonstrated that the frequency modulated (FM) NMOR technique is useful for increasing the dynamic range of NMOR-based magnetometers.  It is possible to achieve the  sensitivity of the device in the range of  $10^{-11} $ G/$\sqrt{\text{Hz}}$ \cite{doi:10.1063/1.3225917} (comparable to the most sensitive superconducting quantum interference (SQUID) sensors).

 Studies have been reported by  Pustelny et al.\cite{bib:AMNMOR,bib:amNMOR} on all-optical magnetometric technique based on nonlinear magneto-optical rotation with amplitude-modulated(AM) light. To extent the magnetometer sensitivity to magnetic fields where Larmor
precession is much faster than the ground state relaxation rate, it is
necessary to synchronize the optical pumping rate with Larmor
precession which can be achieved by modulating the light
. In AM NMOR, the atoms need to be pumped repeatedly. When this modulation frequency is the same as the harmonic
frequency of the atoms we observe NMOR signal. Optimization of modulation waveform is possible in AM NMOR which offer better control of the atomic dynamics and observed signals. The method enables sensitive magnetic-field measurements in a broad dynamic range. The sensitivity of  $4.3\times10^{-9}$ G/$\sqrt{\text{Hz}}$ at 10 mG and the magnetic field tracking in a range of 40 mG has been achieved \cite{bib:AMNMOR}. 

Patton et al. \cite{bib:vectormagnetometer} showed that an all-optical magnetometer is also capable of measuring the direction of a magnetic field along with field magnitude .  This study has been conducted using nonlinear magneto-optical rotation in cesium vapor.  Vector capability is added by effective modulation of the field along orthogonal axes and subsequent demodulation of the magnetic-resonance frequency.  The sensor exhibits a demonstrated rms noise floor of   65 fT/$\sqrt {Hz}$ in measurement of the field magnitude and $ 0.5$  mrad/$\sqrt {Hz}$ in the field direction.  Applications for this all-optical vector magnetometer would include magnetically sensitive fundamental physics experiments, such as the search for a permanent electric dipole moment of the neutron.

NMOR based atomic magnetometer can also be operated in self-oscillation mode \cite{PhysRevA.62.043403}\cite{bib:Budker_2006} originally proposed by Bloom \cite{bib:Bloom_62}. It is possible to acheive a sensitivity of at least 5 pT/$\sqrt {Hz}$ by operating NMOR based magnetometer in this mode.
 Being a quite fast process and having a high bandwidth are
the main features of this self-oscillation scan. 

Pustelny et al. \cite{PhysRevA.74.063420}  showed that when the magnetic field is along the light propagation direction, the main resonance occurs at $\Omega_m = 2\Omega_L$. This resonance appear because of the symmetry of the optically pumped state.  In this case, the amplitude of resonance signal decreases with increasing tilt angle . However, If we tilt the magnetic field direction towards the light polarization axis a new resonance appears at $\Omega_L$ along with the main resonance at $2\Omega_L$ if linearly polarized light is used. The amplitude of the new resonance signal at $\Omega_L$  increases as the angle between B and the light propagation direction increases while main resonance amplitude at $2\Omega_L$  decrease with increasing tilt angle. However,when the tilt angle is larger than some certain angle the resonance amplitude measured at $\Omega_L$ also start to decrease and reaches zero when the magnetic field is directed along the y axis.   It could be possible to evaluate the magnitude of the magnetic field from the ratio of the resonance amplitudes at $\Omega_L$ and $2\Omega_L$. The ratio of the resonance amplitudes at $\Omega_L$ and $2\Omega_L$ can be used to
evaluate the magnitude of the B-field at the measuring point.


Along with FM NMOR and AM NMOR in order to achieve further sensitivity another way of receiving a signal from the magnetometer
has been studied by Michael Sturm \cite{mythesis}. In this study an all optical magnetometer has been operated in Free Induction Decay (FID) mode where the Cs atoms have to be excited
once and afterwards the decaying processes (damped oscillation) of the coherence state is
observed (similar to NMR). The sensitivity depends on the $T_2$ time, which is the decay time
of the macroscopic polarization moment.


%Disadvantages: Since feedback loop self-oscillate in the case of
%constructive interference, it will work only for the signal having a
%phase shift of integer multiple of $2\pi$. This additional phase shift
%might be resulting into slightly off-centered resonance. As a result,
%self-oscillation mode is more susceptible to systematic errors in
%field measurement.
\section{Rb magnetometry based on NMOR}

A scalar magnetometer requires coherent precession of the spin ensemble, so a resonant excitation must be applied in order to force some large fraction of the atoms to precess together with a common phase. Otherwise the phase of individual atoms is random, and the total transverse spin of the ensemble averages to zero . The sensitivity of NMOR based atomic magnetometer depends on the lifetime of the polarization state. So in this case, it is important to use ground-state polarization since ground-state polarization has a longer lifetime than the excited states. 
\begin{figure}[h]
\centering
\includegraphics[width=0.55\linewidth]{figures/optical_pumping}
\caption{The schematic diagram of the optical magnetometry technique.
  A linearly polarized pump beam is sent through the vapor cell
  containing natural Rubidium placed in a homogeneous magnetic field
  B.  The polarization rotation of a linearly polarized probe beam is
  used to measure magnetic field,\label{Rb magnetometry}}
\end{figure}
The working principle of a Rb magnetometer can be described as three step process
\begin{itemize}
\item
Resonant light polarizes Rb atoms via optical pumping. Magnetic
moments of the atoms are oriented with respect to the axis of
alignment
\end{itemize}
\begin{itemize}
\item Aligned magnetic dipole moments experience a torque and precess around the axis of the field at the Larmor frequency and medium becomes birefringent
\end{itemize}
\begin{itemize}
\item A linearly polarized probe beam propagating parallel to the
pump beam is passed through the alkali vapor, the plane of polarization of the probe beam
rotates by an angle proportional to the spin component along that direction, and we detect
this rotation in order to observe the spin behavior. optical polarization rotation of a probe beam is used to measure magnetic field.
\end{itemize}

In the next chapter we will discuss about each components and their uses  for our Rb NMOR magnetometry setup.


% Chapter 3
\chapter{Overview of Rb magnetometer and apparatus}

%\begin{itemize}
%\item Purpose: Describe your experiment, focusing on the overall
%  function, and providing some description of the individual
%  components and their function.
%\item Outline:
%  \begin{itemize}
%  \item Overview of experimental apparatus.  I think this is done by
%    describing Figs.~\ref{fig:zerofield} and \ref{fig:pumpprobe}.
%  \item Group Fig.~\ref{fig:pumpprobe} into major components and
%    describe each one.  For example:
%    \begin{itemize}
%      \item ECDL and DAVLL
%      \item Pump beam and AOM
%      \item Probe beam and balanced polarimeter
%      \item Cell
%      \item Magnetic field system
%        \begin{itemize}
%        \item Shield arrangement
%        \item Degaussing system
%        \item Internal coil\underline{s}
%        \end{itemize}
%      \item DAQ
%    \end{itemize}
%  \item General operation is discussed in the next chapter.
%  \end{itemize}
%\end{itemize}

The purpose of this Chapter is to describe the experimental apparatus
used at UW.
The system is composed of three main parts:
\begin{itemize}
\item A tunable laser system, operating near the Rb D1 line.  Pump,
  probe, and laser characterization beams all come from this light
  source.  The beams are analyzed using various sensitive photodiode
  sensors.
\item A paraffin-coated natural Rb cell.  The cell provides a vapour
  pressure of Rb atoms whose spin state can remain coherent after many
  wall bounces.  The long coherence time is important for the best
  magnetometer sensitivity.
\item A magnetic shielding and field generation system.  The
  magnetometer is being operated in magnetic fields that are
  considerably smaller than Earth's field.  An aspect of the research
  also is to use the magnetometer to learn about magnetic field
  stability.
\end{itemize}
In this Chapter, I start with a more detailed overview of the overall
experiment.  I then discuss each of the major components of the
apparatus in more detail.

In Chapter~\ref{ch:characterization}, I will cover the initial
characterization of a few different modes of operation of the
magnetometer.  (As discussed in Chapter~\ref{ch:magnetometry}, there
are several different ways in which the magnetometer can be operated
dependent with different advantages and disadvantages to each.)

In Chapter~\ref{ch:results}, I discussed the advances made in the
understanding of the magnetometer performance, mostly in relation to
the development of the Free-Induction-Decay mode of operation.

%this section will be completed with the comparison of different
%possible operation modes of an atomic magnetometer and an introduction
%into the concepts of magnetometer uncertainties and the noise level of
%a measurement

\section{Overview}

\begin{figure}%[h]
\centering
\includegraphics[width=0.8\textwidth]{figures/experimental_setup_zero_field}
\caption{Schematic diagram of experimental setup for zero field NMOR measurement (discussed in the text).\label{fig:zerofield}.}
\end{figure}

Fig.\ref{fig:zerofield} schematically depicts a top view of the Rb
magnetometer, as it may be used for measurements of the magnetic field
within a few hundred pT of zero.  This mode was used, for example, in
Ref.~\cite{bib:nmor} to measure the axial magnetic shielding factor of
the passive magnetic shielding system.  In this thesis, this operation
mode was not normally used because we desired to develop the system
for larger magnetic fields.  But it is nonetheless instructive to
demonstrate the starting point of my thesis research.  We also used
this mode to study our process of degaussing of the magnetic shields,
which was also being developed.

An external cavity diode laser produces light near the Rb D1
transition.  The laser is tuned near the $^{85}$Rb F=3,2$\rightarrow$2
absorption minimum.  A microscope slide is used as a beamsplitter in
order to divert a reduced power into the experiment.  The power is
further reduced using neutral density filters (indicated by N.D.F. in
Fig.~\ref{fig:zerofield}).  A linear polarizing sheet is oriented at a
45$^\circ$ angle relative to the vertical direction.  Since the laser
beam is polarized, this further reduces the power by a factor of
two. The laser power after this point is normally in the range of
2-20~$\mu$W.  Small adjustments to the incident plane of polarization
may be made by a $\lambda/2$ plate.

The beam then passes through a paraffin-coated cell containing Rb
vapour.  The cell is within a magnetic shielding and field generation
system that produces a uniform field along the beam axis.  The plane
of polarization of the laser light will rotate slightly as it passes
through the cell because of non-linear magneto-optical effects.  A
balanced polarimeter is used to analyze the optical rotation of the
laser light resulting from the interaction of the laser light with the
atoms in the cell.  The polarimeter consists of a polarizing beam
splitter (a Wollaston prism) which splits the beam into its vertical
and horizontal polarization components.  Each component is sensed by a
balanced photodiode which outputs the difference in light intensities.
If the polarizer is set at a 45$^\circ$ angle and aligned with the
axis of the $\lambda/2$-plate, and in the absence of any optical
rotation, the balanced photodiode would output zero volts.  If the
plane of polarization is rotated by passage through the cell, it will
be sensed by the differential photodiode signal.  This is discussed
further in Section~\ref{sec:Signal Processing}.

Results of the operation of the zero-field magnetometer are discussed
further in Sections~\ref{sec:something-in-ch4}
and~\ref{sec:degaussing-section-in-ch5}.  In order to measure fields
farther from zero, a pump-probe technique is used, where the amplitude
of the pump beam is modulated.

\begin{figure}%[h]
\centering
\includegraphics[width=0.95\linewidth]{figures/experimental_setup}
\caption{Schematic diagram of the experimental setup for measuring
  rotation of the polarization plane with amplitude modulated (AM)
  light. AOM stands for acousto optic modulator, $\lambda/2$ - half
  wave plate, N.D.F- neutral density filter, PBS- polarizing beam
  splitter.\label{fig:pumpprobe}}
\end{figure}

Fig.~\ref{fig:pumpprobe} shows the schematic layout of the apparatus
used for studying the non-linear magneto-optical effects with the
amplitude modulated light. This experimental setup is used in Forced
oscillation scan and FID NMOR discussed in
Section~\ref{sec:Forced-Oscillation Mode} and~\ref{sec:FID}.

% I think you said all this already, or can leave it for the more
% detailed systems which follow.

%In this work, a paraffin-coated vapor cell (about 5~cm long)
%containing natural rubidium with stable isotopes $^{87}$Rb and
%$^{85}$Rb, is used as the magneto-optical sample. A semiconductor
%diode laser is the light source. The laser wavelength
%($\lambda=795$~nm) is precisely controlled by an external locking
%system.

Pump and probe beams are created from the main beam using microscope
slides as beam splitters. The typical light power of the pump beam is
40 $\mu$W (time-averaged) and the typical light power of the probe
beam is 20 $\mu$W.  An acousto-optic modulator (AOM) is used to
modulate the amplitude of the pump beam, normally as a square wave
with low duty cycle.  The details of the working principle of AOM is
discussed in Section~\ref{sec:AOM}.  The pump beam is focused into the
AOM and re-parallelized afterward using two lenses of the same focal
length.  Before interact with the Rb cell pump beam then passes
through an N.D.F to adjust the beam power, and then a linear
polarizer.  After that the linearly polarized light beam interacts with
Rb atoms.

The probe beam follows essentially the same path as for the zero-field
magnetometer.  The nonlinear Faraday rotation signals are analyzed by
a balanced polarimeter.  A Wollaston prism (labelled as PBS in
Fig.~\ref{fig:pumpprobe}) is used to split the beam into its
perpendicular polarization axes which are then analyzed individually
by a differential photodiode.

In this case, the optical rotation is also modulated near the pump
modulation frequency.  A lock-in amplifier is used to demodulate the
signal. Table~\ref{table:laser power} shows the typical beam power at
several positions in the experimental setup.



\section{External Cavity Diode Laser}

In our NMOR based optical magnetometry setup, laser light is provided
by an external cavity diode laser (ECDL).  
%These kind of diodes are
%semiconductor diodes and thus vibration and shock resistant.
%They are
%also wavelength tunable.
ECDLs emit a single mode laser light with a very narrow linewidth
($\sim 100$~kHz).  A critical aspect of an ECDL is temperature control
of the cavity, since the laser frequency depends on the cavity length
and hence on the thermal expansion coefficient of the cavity
material. Micrometer screws enable manual coarse tuning, while precise
scans without mode hops are performed by a piezo actuator. This kind
of grating stabilized diode lasers is especially advantageous for
spectroscopy with the alkalines. Our Toptica DL-100 outputs a tunable
wavelength near 795 nm with an output power $<100$ mW. The laser spot
size is elliptical, and approximately 3 mm x 5 mm = $15 mm^2$.  The
laser was typically tuned to the D1 (F = 3,2$\rightarrow$2) absorption
minimum and then adjusted to maximize optical rotation.

\section{Dichroic Atomic Vapor Laser Lock (DAVLL)}

In order to control the laser frequency to a fraction of the Doppler-broadened linewidth of the relevant hyperfine atomic transition of the Rb D1-line, a frequency error signal is generated by taking usage of the Zeeman effect combined with circular dichroism of an atomic vapor which is exposed to a magnetic field \cite{doi:10.1063/1.3568824}. The generated error signal passes through zero crossings as the laser frequency coincident with the lock frequency. 
\begin{figure}[h]
\centering
\includegraphics[width=0.8\linewidth]{figures/DAVLL}
\caption{Schematic diagram of experimental setup for characterizing the DAVLL . PBS- polarizing beam splitter, PD- photodiode.\label{fig:DAVLL}}
\end{figure}
The schematic diagram of the apparatus used in  the DAVLL system has shown in Fig \ref{fig:DAVLL}. A linearly polarized probe beam is incident on a glass cell filled with Rb vapor surrounded by a set of permanent magnets. In this magnet arrangement a layer of ceramic magnets (also known as ferrite magnets) and a layer of flexible magnetic stripping are placed one after another and a customized holder is used to hold the pieces together. Then the magnet holder is placed on the tp of Newport 271 Lab Jack which is uniquely designed to provide smooth, stable height adjustment and high load capacity.  The wave vector of the light is parallel to the axis of the generated magnetic field by permanent magnets. After the interaction of the laser beam with the Rb cell, the beam passes through a quarter-wave plate before impinging on a polarizing beam splitter (PBS) or a Wollaston prism. The linearly polarized beam incident can be split into two orthogonal circularly polarized beams. A set of photodiodes are used to detect the intensity of the right and left circularly polarized beams. Both of the photodiodes are attached to a polarimeter board which is used to measure the difference in signals and amplifies it. A Tektronix TDS2024 oscilloscope is used for monitoring each photodiode signal as well as the difference output. The signal is then fed into a PID controller which finally controls the laser diode current corresponding to a certain laser frequency.

\begin{figure}[h]
\centering
\includegraphics[width=0.7\linewidth]{figures/magnet.png}
\caption{Picture of cell-magnet arrangement in the DAVLL system.\label{fig:magnet}}
\end{figure}

\section{Laser tuning and locking}
In order to polarize atoms by optical pumping, it's important to tune the laser properly to expected transition line.
Setting the laser temperature correctly is one of the important parts to achieve good tuning . First we need to turn on the temperature control panel and then it’s easy to adjust laser temperature manually by tuning the knob of temperature control panel. For our tune the laser temperature is set to 20.1 $\degree$c. For good tuning it is also important to set the laser current which corresponds to the emission of laser light with a wavelength matching the absorption line of the Rb atom. This is usually done by a DAVLL scan so it's necessary to make sure the DAVLL optical setup is done correctly. An oscilloscope is used to observe the output of the differential amplifier. Trigger the scope on the trigger output on the scan control (SC) panel.
Laser current can be controlled by adjusting the current control knob until the spectral structure of $^{85}Rb$ appears (see Fig \ref{fig:laser tune}). we need to keep adjusting the current control knob until a maximum symmetry between the upsweep and downsweep portions is achieved. After that by adjusting the knob of scan control panel we can zoom in the structure and set the trigger to steep of the absorption minima. 
\begin{figure}[h]
\centering
\includegraphics[width=0.8\linewidth]{figures/laser_control_}
\caption{Diode laser controller unit consist of analog current control module (CC), scan control module (SC),low-noise temperature control module (TC), feedback controlyzer (FC) and Diode Laser Controller display .\label{fig:laser controller}}
\end{figure}

Digilock 110 feedback controlyzer (FC) is used for laser locking. The DigiLock 110 is an up-to-date digital hardware allows to implement the scan generator, PID controllers and optional frequency modulation techniques all into one plug in module. It offers graphical user interface (see Fig \ref{fig:digilock}) running on a PC makes the procedure of laser locking enormously easy. Different important features are also available in this module for analyzing and optimizing the control system.
 \begin{figure}[h]
\centering
\includegraphics[width=0.7\linewidth]{figures/laser tune.png}
\caption{Oscilloscope trace of differential photodiode signal \label{fig:laser tune}}
\end{figure}
The output of the laser feedback controlyzer panel is connected to the computer where Digilock software was already installed. The output signal of the differential photodiode is fed into the controlyser panel main input. After connecting the DigiLock 110 we can turn on scan and navigate to the autolock screen at the bottom. Then the the portion of the spectrum that we tuned earlier will appear (see Fig \ref{fig:digilock}). Then we need to select the crosshairs tool which allow us to drag the crosshairs on the part of the spectrum that we want to tune to. Finally for successful laser locking we need to click and select "PID lock to slope". For most of the studies reported in this thesis we use the auto locking features of the DigiLock software.

\begin{table}[h]
\centering
\begin{tabular}{|l | l|}
\hline

\textbf{ Position}    & \textbf{Laser Power($\mu$W)} \\
\hline

AOM ~&  ~ ~ ~ ~ ~4000  \\

Pump beam ~  &  ~ ~ ~ ~ ~ 60  \\

Probe beam ~  &  ~ ~ ~ ~ ~ 22  \\
After Cell ~ &  ~ ~ ~ ~ ~ 18   \\

\hline
\end{tabular}
\caption{Adjusted laser power at several positions in the experiment.\label{table:laser power}}
\end{table}
\begin{figure}[h]
\centering
\includegraphics[width=0.7\linewidth]{figures/digilock.png}
\caption{digilock user interface\label{fig:digilock}}
\end{figure}
\section{Acousto-optic Modulator (AOM)\label{sec:AOM}}

The acousto-optic modulator (AOM) offers a method of modulating the
amplitude of the laser light.
The AOM is used to modulate the amplitude of the pump beam. The pump beam is linearly polarized light, and so the amplitude
  is modulated at $2\omega_L$. This form of pumping generates a coherent precession of an axis of alignment in the atoms in the Rb cell. In FID mode, once the coherence has been sufficiently established, the AOM may be switched off in order to measure the precession frequency of the state using the probe beam.

 In principle, it contains an optically transparent medium (e.g glass, quartz) which has a piezoelectric transducer attached at the end that propagates acoustic waves within the medium. An RF signal needs to apply to the transducer to generate the acoustic wave. The compression and refraction of the sound waves result in periodic variations of the refractive index of the medium which then form a diffraction grating. Any incident laser beam will be diffracted by this grating generally provides a number of diffracted beams. The strength of the sound wave is directly related to the intensity of the defracted light. Depending on the interaction length L, laser wavelength $\lambda$ in the medium and the sound wavelength $\Lambda$ it is possible to operate AOM in two different modes, Raman-Nath regime and Bragg regime. For our experiment we need to operate AOM in Bragg regime.

\begin{itemize}
\item In the Bragg regime~($L > \Lambda^2/\lambda$) the light beam enters the medium at one particular Bragg angle 
\begin{equation}
\theta_B=\frac{\lambda}{2\Lambda}
\end{equation}                                 
and only one diffraction order produce. This observed diffraction pattern generally consists of two diffraction maxima; these are the zeroth and the first orders. In this case the possible maximum intensity of the first order diffracted light is 100\%. Due to this relatively high efficiency, this first order can be used for amplitude modulation. For our atomic magnetometry setup we operated the AOM in the Bragg regime.
\end{itemize}
 For this experimental setup, an Isomet 1205C-1 AOM  is used with an RF center frequency of 80 MHz. The AOM uses crystal Lead Molybdate (PbMo$0_4$) for the optical interaction medium and Lithium Niobate as the piezoelectric transducer. This AOM has a bandwidth of 80 MHz.  The amplitude modulating pulses are driven with an Isomet 532C-2 AO driver with video input range 0.0 V to 1.0 V. An Agilent 33522A function generator is used to regulate the  AOM driver. This amplitude modulation is done by a square wave modulation with a duty cycle of 1-10\%. For this specific model of AOM driver the RF rise/fall time is smaller than 6 nsec. Since the active aperture of the modulator is tiny (1mm) its very hard to focus the laser beam into it. In order to reach maximum deflected light intensity we need to adjust Bragg angle. In this setup, the achieved maximum intensity of the first order deflected light is 86$\%$. For  1~$\mu$T field the AOM operation frequency is 9335 Hz and the AOM driving frequency is 2034 Hz when the magnetometer operate at $0.2~ \mu$T .
\begin{figure}[h]
\centering
\includegraphics[width=0.7\linewidth]{figures/AOM}
\caption{Acousto-optic modulator}
\end{figure}
\section{Rb Cell}
When glass cell is used to store alkali atoms, the atomic mean free paths increase and alkali spins depolarize immediately after making non-elastic collision with the glass walls. Prolonging the atomic alignment is crucial to achieving ultra-narrow NMOR resonance widths. So it is necessary to prevent these non-elastic collisions for achieving the longer lifetimes of atomic ground state coherences \cite{PhysRevA.72.023401,Balabas:10,doi:10.1063/1.3236649}. Two methods are currently in use to extent the atomic coherence lifetime. One of them is to add buffer gas to the cell containing alkali sample and the other method is to coat the inner walls of the cell with anti-relaxation materials. The advantage of using buffer gas is that it reduces the resonance width by extending the lifetime of coherence state. 
\begin{figure}[h]
\centering
\includegraphics[width=0.5\linewidth]{figures/cell}
\caption{Paraffin coated Rb cell}
\end{figure}
A paraffin-coated vapor cell containing  natural rubidium with stable isotopes $^{85}$Rb and $^{87}$Rb is used for this work. The reason behind using paraffin as a anti-relaxation coating is that it allow polarized atoms to bounce off the walls of a paraffin-coated cell $\sim 1000$ times before depolarizing \cite{PhysRev.147.41,PhysRevA.72.023401}. Coated cells have the advantages of providing larger optical rotation
signals, reducing the effect of magnetic field gradients on the spin polarization lifetime,
and lowering the power requirements of the lasers used for pumping and probing. The vapor cell is cylindrical, 5 cm long and 5 cm in diameter with optical
flats on the ends. The cell was provided by D. Budker, having been prepared in
a fashion similar to the cells described in Ref.\cite{PhysRevA.72.023401}. The cell was characterized
using a method similar to Ref. \cite{PhysRevA.72.023401}, by measuring the relaxation of longitudinal
polarization using optical rotation as a probe. The long time component of the relaxation was thereby found to decay with a time scale of 60 ms.  Longer relaxation time indicates good quality of cell. The temperature of the vapour cell was controlled by the ambient temperature of the surrounding room (∼ 21 $\degree$). 

\section{Magnetic field system}
The magnetic field system consist of a four-layer $\mu$ metal magnetic shielding, degaussing system and Internal coils.
\subsection{Magnetic Shielding}
\begin{figure}%[h]
\centering
   \includegraphics[width=0.8\linewidth]{figures/magnetic_shielding}
 \caption{Four layer $\mu$ -metal magnetic shielding. The diameter of each endcap is larger by 0.1 cm to fit over its corresponding cylinder. The hole diameter and stovepipe length for each endcap are the same. High density polyethylene spacers and nylon thread rods/nuts are used to hold the shields and end caps together.}
\end{figure}
\begin{figure}%[h]
\centering
\includegraphics[width=0.8\linewidth]{figures/shield.JPG}
 \caption{ Schematic diagram of the 4-layer magnetic shield (dimensions in cm).\label{fig:shield}}
\end{figure}
In precision magnetometry, magnetic shielding is required to achieve well characterized, stable magnetic field conditions independent of the earth’s magnetic fields and environmental perturbations. Shielding ratio T is a parameter to evaluate the efficiency of a shield which can be expressed as
\begin{equation}
 T = \frac{B_{in} }{B_0} 
\end{equation}

where $B_0$ is the homogeneous magnetic field in free space, and $B_{in}$ is the field induced in the presence of shield due to  $B_0$. Using multi-layer shielding is an efficient way to enhance shielding efficiency \cite{doe:website2} . The effectiveness of a multilayer shield with thin shells having wide gaps between them is same as the much heavier and more expensive thick, single layer shield.
The optimization of the air gaps between the shells is important to minimize the total size of a multilayer shield. When a coil is placed inside the innermost layer of passive shield the shield turns into a return yoke. As a result, the magnitude of field $B_0$ increase. A four layer $\mu$-metal magnetic shielding with nearly spherical geometry is used for this highly sensitive Rb magnetometer. $\mu$-metal is a very high permeability nickel based alloy, which is used for shielding low-frequency magnetic field. When the shape is close to a sphere we will get zero magnetic field inside independent of the outside field. The inner radius of the innermost shield is 18.44 cm, equal to its half-length. The radii and half-lengths of the progressively larger outer shields increase geometrically by a factor of 1.27. Each cylinder has two endcaps which possess a 7.5 cm diameter central hole \cite{Andalib:2016ahj}. A stove-pipe of length 5.5 cm is placed on each hole was designed to minimize leakage of external fields into the progressively shielded inner volumes. The magnetic shielding factors of each of the four cylindrical shells, and of various combinations of them, were measured and found to be consistent with $\mu_r \sim 20, 000$ \cite{Martin:2014foa}


\subsection{Degaussing system\label{sec:Degaussing}}
Our four layer $\mu$ metal magnetic shield is designed to minimize the magnetic field at the cell, but
magnetic hysteresis limits the magnetic field in any region to be exactly zero. Degaussing (demagnetizing) process is used to reduce the background magnetic field inside the shield. Although Rb cell is securely placed inside the four layer mu-metal shielding it is necessary to demagnetize the shield before every measurement session  to eliminate accidentally created local magnetizations.
\begin{figure}%[h]
\centering
\includegraphics[width=0.6\linewidth]{figures/degaussing_coil.png}
\caption{Photograph of degaussing coil.\label{fig:degaussing_ccoil.}}
\end{figure}
 For degaussing a special coil(like the white wire  shown in Fig \ref{fig:degaussing_coil}) is wind around inner most layer of shielding  in toroidal configuration
and  since the endcap of innermost shield is open it might important to degauss the outer layers of shield. Toward this end, a single turn wire is wind around the outer 3-layers of shielding.
In practice, demagnetizing is achieved by winding by supplying the coil with oscillating current. The current amplitude decreases depending on the chosen envelope function starting from a current that yields magnetic saturation inside the ferromagnetic material, down to zero. 
\begin{figure}[h]
\centering
\includegraphics[width=1.0\linewidth]{figures/degaussing_system}
\caption{Schematic diagram of degaussing system consist of function generator, power amplifier, transformer, variable resistor(R), 1 $\ohm$ sense resistor (A), oscilloscope (V) and degaussing coil. \label{fig:schematic-of-degaussing_system.}}
\end{figure}

In our degaussing system, an Agilent 3522A function generator  provides a ramped sinusoidal that controls a current supply driving the degaussing coil \cite{Martin:2014foa}. The wires going to the degaussing loop should be twisted together to avoid picking up or causing noise.  A DCP 260 Servowatt is used as the amplifier for degaussing system which outputs $\pm$ 60~V at idle and nominal output current is  $\pm$ 5 A. The used transformer for this setup is a  HS1B250 hevi-duty transformer which provide unity gain (wire connection is set to in 240:240 mode). The frequency of the transformer is  60 Hz.  A double-pole, double-throw (DPDT) switch is used to electrically connect and disconnect the degaussing coil from the circuit. The function generator settings has discussed in Table \ref{table:degaussing setting}. For our degaussing system we are using an envelope function which has $1\times 10^6$ points. Degaussing is complete after $5\times 10^6$ points. Sample rate affects how quickly we move through this waveform. If sample rate is set to 10000 samples/sec and the frequency of the carrier wave is set to 10 Hz then it means 500 cycles were completed. 
 \begin{table}%[h]
\centering
\begin{tabular}{|l|l|}
\hline
\textbf{ SETTING}    & \textbf{ VALUE} \\
\hline
Function generator &   \\
\hline
Frequency &  10 Hz   \\
Sample rate    &  10000 sample/sec  \\
Amplitude   &   10 V \\
Offset  &       0 V  \\
\hline
\end{tabular}
\caption{Setting for degaussing system \label{table:degaussing setting}}
\end{table}


\subsection{Internal coil\label{sec:Internal coil}}
 For the Rb atomic magnetometer an internal coil referred as z-coil is used to provide the magnetic field along the axis of light propagation direction (z-direction). This coil produces uniform field in the ROI  of magnetic shielding. The z-coil was wound on a 7.62 cm diameter, 20.32 cm long ABS plastic pipe. Seven turns of 26 AWG magnet wire were wound at 2.54 cm spacing, with 1.27 cm spacing from the magnetic faces of the endcaps of the innermost magnetic shield. The spacings were chosen so that, in the infinite permeability limit, and in the limit where the axial aperture holes in the endcaps are small, the boundary conditions would produce image currents forming an infinitely long solenoid. Two saddle coils were wound on the same cylinder in order to control transverse fields( along x and y direction) internally; these were normally disconnected during precision measurements. The internal coil system was calibrated using a three-axis fluxgate magnetometer at fields of ∼100 nT.  The calibration constant for z coil is 48 nT/mA and the calibration constant for x and y coil is 25 nT/mA. The calibration of the z-coil was verified using the NMOR magnetometer with an AM pump beam, and the known gyromagnetic ratios of~ $^{85}Rb$ and $^{87}Rb$. Homogeneity of the residual field and magnetic field generated by the coil system was measured by scanning a fluxgate magnetometer along the axis of the system with and without the coil energized. At a field of 1 $\mu$T, the axial field generated by the coil was uniform within the ROI to the $1\%$  level.


\section{Lock in Amplifier}
SR830 DSP Lock in Amplifier is a elementary part in the DAQ system of Rb magnetometry. It can measure very  small voltages. The most attractive feature of a lock in amplifier is that it is able to suppress all noise contributions which differ from the reference frequency. A reference signal is applied to the lock-in amplifier which is usually done by an external oscillator which passes a discriminator. In this magnetometry setup sync output of a function generator is used as external reference signal for force oscillation scan. The external reference signal is phase locked to an internal reference frequency, provided internal oscillator of the lock-in.Since a lock-in amplifier has two phase sensitive detectors we obtain two output signals. During the process of phase sensitive detection (PSD), the reference signal is first multiplied with the real input signal  and in a second stage, the real input signal is multiplied by
the lock-in reference signal with a phase shift of $90\degree$. Then a lowpass filter is used to filter both signals. The first one is referred to as X output and the 2nd output signal is referred to as Y output. X output is knows as in phase component and the Y output is called out of phase component.


\section{Signal Processing\label{sec:Signal Processing}}
Sensitive magnetometry requires detection of extremely small optical rotation angles. There are numerous methods for detecting the optical rotation of the probe beam.  we instead use the balanced polarimetry technique.  Transmitted light was analyzed for optical rotation by a balanced polarimeter system containing a Wollaston prism and a Newport model 2307 balanced photo receiver. The power delivered to the vapour cell was typically 15 $\mu$W, measured periodically using a Newport model 818-SL power meter inserted into the
beamline. After the cell the probe beam passes through a polarizing beamsplitter set at 45\degree~to the initial polarization direction resulting in two separate beams with individual intensities given by 
\begin{equation}
I_1 = I_0\sin^2(\theta-\frac{\pi}{4})
\end{equation}

\begin{equation}
I_2 = I_0 \cos^2(\theta-\frac{\pi}{4})   
\end{equation}

such that $I_1 + I_2 = I_0$, and the intensities are balanced when $\theta = 0$ and unbalanced otherwise. 
For small rotations $\theta << 1$ in terms of power we can write,
\begin{equation}
  \theta \approx \frac{P_1-P_2}{2(P_1-P_2)}  
\end{equation}

where $P_1$ and $P_2$ represent signals of the two photodiodes in the polarimeter. The differential value is acquired with a lock-in amplifier (LIA) referenced to the pump beam modulation provided by a function generator. The demodulated signal from the LIA is acquired by a personal computer for analysis. 



% Chapter 4
\chapter{Magnetometer Operation\label{ch:characterization}}

In this Chapter I describe the different ways in which the
magnetometer was operated.  Some of these modes of operation were
applied to furhter experiments, which are reported in
Chapter~\ref{ch:results}.  Some of the modes were simply used to
perform a basic characterization of the magnetometer and to compare
with the results of other groups which were discussed in the
Literature Review in Chapter~\ref{ch:magnetometry}.

The main purpose of this Chapter is to give an overview of the many
parameters which can affect the performance of the magnetometer.
Chapter~\ref{ch:results} goes into further detail on studies of
specific performance metrics under modification of additional
parameters.

The main operation modes of the magnetometer reported in this Chapter
are:
\begin{itemize}
\item Near-zero-field operation.  In this mode, the magnetic field
  must be swept in order to calibrate the magnetometer.  The dynamic
  range of the magnetometer in this case is $|B_z|\lesssim 0.2$~nT.
\item Amplitude modulated NMOR, which itself was used in two distinct modes:
\begin{itemize}
\item Continuously pumped (a.k.a.~forced oscillation) mode.  In this
  mode, the amplitude of the pump beam was modulated continuously and
  the optical rotation signal was demodulated resonantly at the same
  frequency.
\item Free-induction decay (FID) mode.  In this mode, the amplitude of
  the pump beam is modulated for a time and then switched off.  The
  oscillation frequency of the optical rotation signal is then
  measured non-resonantly.
\end{itemize}
In these modes, the magnetometer was generally operated at 0.2~$\mu$T
or 1.0~$\mu$T, which is of considerably more relevance to the nEDM
experiment.
\end{itemize}
In Chapter~\ref{ch:results}, most of the measurements will relate to
our studies using FID mode.  The exception is that some degaussing
studies will be done near zero field and hence will use that mode.


\section{NMOR near zero field and degaussing studies}

During this measurement the pump beam was switched off and the probe
beam is used as its own pump (see
Section~\ref{sec:something-in-chapter-3}).

I used the magnetometer in this mode to study the function of the
degaussing system described in
Section~\ref{sec:Degaussing}.

The experiment was carried out as follows:
\begin{enumerate}
\item The laser beam was tuned for maximum optical rotation and
  stabilized using the DAVLL system.  The beam power was $\sim
  20~\mu$W.
\item Optical rotation of the probe beam was monitored throughout the
  experiment via an oscilloscope monitoring the differential
  photodiode signal.
\item The innermost magnetic shield was degaussed with various
  parameters for the sequence.  The operation of the degaussing
  circuit was described in Section~\ref{sec:Degaussing}.  After
  completing the degaussing sequence a switch is opened to
  electrically isolate the degaussing coil.
\item The magnetic field along the $z$-direction (defined in
  Section~\ref{sec:Internal coil}) is swept in order to calibrate the
  differential photodiode signal as a function of applied $B_z$.  In
  this way the initial magnetic field after degaussing may be deduced.
\end{enumerate}
%In this case,  at first degaussing the shields that surround the vapour cell is done in order to cancel background B-field.A function generator is used to drive the degaussing coil. After completing a degaussing sequence switch is opened to electrically disconnect the degaussing coil from experimental setup. Then B-field ramp is started and NMOR signal is observed through a oscilloscope which  connected to balance photodiode output. The magnetic field sweeping is done in a triangle wave. In 
Fig.~\ref{fig:TUNE} displays the sequence of measurement events in
time, along with the differential photodiode signal (in Volts), which
is proportional to optical rotation.  Also shown is a voltage applied
to the $z$-coil with a 10~k$\Omega$ resistor in series which dominates
the resistance of the circuit.  Recall that the coil constant of the
$z$-coil is $\sim 48$~nT/mA (see Section~\ref{sec:Internal coil}).
The sweep range of this trace is therefore
% (0.8~V)/(10000~Ohm)/*48~nT/mA = 3.84 nT ???  It would be good to
% write the correct calibration constant in the part of the thesis
% where this coil is describe (probably somewhere in Chapter 3).
1.9~nT peak to peak.

In the first section of the oscilloscope trace, the impact of the
degaussing procedure inducing noise in the optical rotation signal can
be seen.  The next section involves the ramping down of a variable
resistor, followed by opening the switch to isolate the degaussing
coil.  Then the magnetic field $B_z$ is swept and the characteristic
dispersive Lorentzian shape of NMOR is observed.

\begin{figure}%[h]
\centering\includegraphics[width=0.7\linewidth]{figures/scope_trace_of_field_sweeping}
\caption{Oscilloscope trace of measurement OR near zero field. The
  purple curve shows the voltage across the 10~k$\Omega$ resistor in
  series with the $z$-coil, from which the magnetic field ramp of
  1.9~nT peak to peak can be deduced.  The blue trace indicates the
  differential photodiode signal which is proportional to the optical
  rotation.  The left side of the traces show the impact of the
  degaussing procedure, while the right side shows the calibration
  procedure.\label{fig:TUNE} }
\end{figure}


In this measurement NMOR signal is used to determine effectiveness of
degaussing procedure. After the degaussing procedure the observed OR
signal is non-zero while the applied field $B_z$ was zero which
indicates the existence of a remnant field.  The effect of degaussing
parameters sensed by the magnetometer is discussed further in
Section~\ref{sec:degaussing}.


% We need to have this table somewhere, but I think it needs to be in
% the section where you talk about degaussing.

%\begin{table}%[h]
%\centering
%\begin{tabular}{|l|l|}
%\hline
%\textbf{SETTING}    & \textbf{VALUE} \\
%\hline
%Function generator &   \\
%\hline
%Frequency &  10 Hz   \\
%Sample rate    &  10000 sample/sec  \\
%Amplitude   &   10 V \\
%Offset  &       0 V  \\
%\hline
%\end{tabular}
%\caption{Setting for degaussing \label{table:degaussing setting}}
%\end{table}

Fig.~\ref{fig:near zero field} shows an example of the calibration of
the differential photodiode signal to the field applied by the
$z$-coil, for data where the degaussing part of the sequence have been
removed.  The field is calibrated to the voltage signal as described
above.  In Fig.~\ref{fig:near zero field}, the data have been fitted
to a dispersive Lorentzian shape given by the function
\begin{equation}
\mathrm{Signal~(V)}=\frac{a(B-B_0)}{1+a(B-B_0)^2}\cdot l+C
\end{equation}
where $a$, $B_0$, $C$, and $l$ are fit parameters.  A key measure of
magnetometer performance is the valley-to-peak distance, which in this
case is about $\Delta B=\frac{2}{a}=0.49$~nT.  The other key measure
in this case is the deduced field at zero crossing given by the fit
parameter $B_0=0.019$~nT.  Thus, after degaussing, a remanent field of
19~pT is found.  This is only the on-axis field.  It is possible that
transverse fields can be larger, and these tend to make the width
$\Delta B$ of the zero-field curve larger.

\begin{figure}[h]
\centering\includegraphics[width=0.7\linewidth]{figures/near_zero_field}
\caption{Optical rotation as a function of magnetic field applied
  along the direction of the laser beam. The signal looks like a pure
  dispersive Lorentzian curve. The measured resonance width is 0.49~nT
  and the measured remanent field is 0.019~nT.\label{fig:near zero
    field}}
\end{figure}

In order to determine the sensitivity, a magnetometer can be operated
in different modes. Most of them found its application during the time
this Master's thesis was prepared. Although main focus of this work
was to study magnetometer performance in Free Induction Decay (FID)
mode.

\section{Amplitude Modulated NMOR:  Forced Oscillation Mode}

In this forced-oscillation measurement scheme the pump beam amplitude
is modulated and the differential photodiode signal is demodulated at
the same frequency $\Omega_m$ using a lock-in amplifier.  The
modulation/demodulation frequency is near twice the Larmor frequency
of the atoms in the magnetic field.

In the first instance, we tried keeping the frequency constant and
sweeping the magnetic field.  We can then determine the resonant field
by fitting the resonance lineshape.  The disadvantage of this method
is that the field must be changed in order to measure it.  {\bf An
  example of this might be shown in Fig.~\ref{fig:AMORmaybeidontknow},
  or it might not be.  Really, it is impossible to tell.}

\begin{figure}[h]
\centering\includegraphics[width=0.7\linewidth]{figures/AM_NMOR}
\caption{\bf The rest of this figure caption might be totally
  false... AMOR resonance signal with a 5 cm cell containing natural
  Rb.  Data was acquired by using a balanced photodiode which
  demodulated through lock-in amplifier as the frequency is swept
  slowly near 9.37~kHz.  The parameters of the sweep are shown in
  Table~\ref{tab:freqsweep} and are discussed further in the text. The
  observed resonance width, when translated from frequency into field,
  is 2.5~nT.\label{fig:AMORmaybeidontknow}}
\end{figure} 

A better way to measure the field without having to change it is to
sweep the modulation/demodulation frequency instead.  The resonant
frequency then gives a determination of the field (when
$\Omega_m=2\Omega_L$)

\begin{figure}[h]
\centering\includegraphics[width=0.7\linewidth]{figures/AM_NMOR}
\caption{\bf I do not know what this figure is, and so here is what
  I'm guessing that it might be: AMOR resonance signal with a 5 cm
  cell containing natural Rb.  Data was acquired by using a balanced
  photodiode which demodulated through lock-in amplifier as the
  frequency is swept slowly near 9.37~kHz.  The parameters of the
  sweep are shown in Table~\ref{tab:freqsweep} and are discussed
  further in the text. The observed resonance width, when translated
  from frequency into field, is 2.5~nT.\label{fig:AMOR}}
\end{figure} 

Fig.~\ref{fig:AMOR} {\bf potentially, maybe, and if it doesn't we
  should find the figure that does show this, since you included the
  table below which relates to the data that may or may not be
  presented in the figure but should be} shows a measurement of the
in-phase and quadrature signals measured while the frequency is swept
very slowly.  The parameters of this mode of operation, particularly
of the slow frequency modulation, are shown in
Table~\ref{tab:freqsweep}.

\begin{table}%[h]
\centering
\begin{tabular}{|l |l|}
\hline

\textbf{ SETTING}    & \textbf{VALUE} \\
\hline
Function generator &   \\
\hline
Frequency & 9.37kHz   \\

Waveform    &  Square  \\

Amplitude   &  $1V_{pp}$  \\
Offset  &       500 mV  \\
Duty cycle       &    $1\%$ \\
Frequency Deviation     &   40 Hz  \\
FM Frequency     &   100 mHz  \\
modulation waveform      &    Triangle \\
Amplitude modulation & On \\
\hline
Lock-in amplifier &     \\
\hline
Lock in frequency     & 9.37 KHz \\
Time constant     &  $300\mu s$ \\
Sensitivity      &  100mV  \\
\hline
\end{tabular}
\caption{Setting for AM NMOR at $1\mu$T field {\bf which may or may
    not correspond to any other data showed in any other figure of
    these thesis.  All other parameters of the measurement are
    apparently unknown.}.\label{tab:freqsweep}}
\end{table}

Fig.~\ref{fig:AMOR} shows the characteristic dispersive and absorptive
shapes in the in-phase and quadrature signals, respectively.  This is
in good agreement with expectation.  The quality of the magnetometer
is again characterized by the peak-to-valley distance in frequency,
which, when translated to field corresponds to a width of 2.5~nT.

A drawback of this mode of operation is that the drive frequency is
constantly changing, so that each data point for optical rotation
represents a range of drive frequencies that were sampled within the
lock-in time constant.  A more robust method involves selecting
particular frequencies in series and then fitting to determine the
resonant frequency, as done in Ref.~\cite{mythesis}.  I tried
this method also, which I call a forced-oscillation scan.

For this forced-oscillation scan a frequency range and frequency
increments are entered into a custom python code. Via a USB
connection, the function generator is set to the appropriate
frequencies, driving the pump modulation of the magnetometer. The
frequencies are taken by the lock-in amplifier as a reference signal
as well as the differential output of the polarimeter board which is
demodulated at the reference frequency. The resulting in-phase (X) and
quadrature (Y) outputs are collected by the python code via GPIB.  A
settling time of at least 5 lock-in time constants ensures that no
memory of the previous frequency is retained by the lock-in amplifier.
\
% Refer to appropriate Section of Chapter 2, if you want to make this
% statement.

%The quadrature components arise because exactly on resonance, the
%aligned atoms produce maximum optical rotation when the alignment
%axis is at an angle of $ \pi/4$ to the direction of the light
%polarization.

%Study NMOR signal by sweeping resonance frequency with amplitude
%modulated light.

%The magnetometer can be run as a forced oscillator, where a frequency
%generator is used to sweep the frequency of the laser amplitude
%modulator through the NMOR resonance. In this case the applied
%magnetic field~($B_z$) remains fixed during a measurement cycle
%(pumping and probing).
%The experimental setup for this measurement
%scheme remains almost same as for AM NMOR.

{\bf Fig.~\ref{fig:FMOR} shows something, but I do not know what.
  Below is my best guess as to what is displayed in
  Fig.~\ref{fig:FMOR}.}

\begin{figure}[h]
\centering\includegraphics[width=0.5\linewidth]{figures/FM_modulation}
\caption{Demodulated components of the optical rotation, as a function
  of the forced-oscillation frequency.  NMOR resonance recorded with
  the 5 cm natural Rubidium cell, square-wave $100\%$ modulation of 1
  duty cycle. The $X$ component {\bf does not quite} reaches its
  maximum and the y component {\bf does not quite} have its zero
  crossing at $\Omega_m=2\Omega_L$ {\bf because of $\phi$}.  In this
  case the measured resonance width is 1.67~nT {\bf when translated
    from a frequency into a magnetic field???  It says a frequency, in
    the figure?  Lines drawn on the figure apparently do not line up
    with anything, and are meant to distract the
    reader?  I recommend to fix this figure.}.\label{fig:FMOR}}
\end{figure} 

The applied magnetic field was about 1~$\mu$T directed along light
propagation direction.  {\bf I think that this sentence is true.}

The measurement was taken by driving an Agilent 33522A function
generator to different modulation frequencies in order to find the
resonance frequency of the given Rb oscillator. The modulation
waveform is a square wave with a duty cycle of 1\%. Since our pump
beam is linearly polarized, a modulation at 2$\Omega_{L}$ is necessary
because of the two fold symmetry of alignment state. The typical range
of drive frequency is 9.31 KHz to 9.409 kHz for 1~$\mu$T field. An
unmodulated linearly polarized probe beam is used to analyze the spin
response. While the magnetometer response was recorded with a lock-in
amplifier, which is connected to balanced photodiode output,
demodulating the signal at the drive frequency of the atomic
oscillator.

{\bf The new things that I think I learned from this paragraph:}
\begin{itemize}
\item The duty cycle of the pump beam was 1\%.
\item The range of drive frequencies used in Fig.~\ref{fig:FMOR} was
  9.31 kHz to 9.409 kHz.
\end{itemize}


The center of the resonance is used to determine the Larmor frequency
and hence the magnetic field.  Fig.~\ref{fig:FMOR} shows the resonance
scan where data was taken by setting a function generator to different
drive frequencies for the given atomic oscillator. During the scan
drive frequencies was working as reference frequencies of the lock-in
amplifier. The red and blue data points indicate x and y output
respectively. When the modulation frequency is twice the Larmor
frequency the x component reaches its maximum and the y component has
its zero crossing. In this force oscillation scan the measured
resonance width is 1.67 nT.

{\bf The new things that I think I learned from this paragraph:}
\begin{itemize}
\item The red and blue data points indicate x and y output
respectively.
\item Something in the width corresponds to 1.67 nT, but it is very
  unclear how.
  
  \bf The width ($\Delta f$) is 15.66 Hz which then translated into magnetic field using the relation $\Delta B= \Delta f/2 \gamma$ where $\gamma$ is the gyromagnetic ratio of Rb.
\end{itemize}

The probe beam then analyzed by photodiode whose output was connected
to a digital-signal-processing lock-in amplifier. The lock-in is
connected to a computer via GPIB. The DAQ computer saves the lock-in
data in a Python program where the data can be analyzed. The idea of
the fitting algorithm is to generate a concatenated function, having
the absorptive part as the first part and the dispersive as the second
(directly connected to each other). The corresponding frequency values
are simply the frequency scan range for the absorptive part while the
frequency increments are added to the maximum frequency point of the
absorptive in order to obtain the frequency parts for the dispersive
curve. This results in the overall data for the frequency, which is
inserted into the fitting function.

{\bf The new things that I think I learned from this paragraph.}
\begin{itemize}
\item There is a fit being done.
\end{itemize}


The function of absorptive part can be expressed as:
\begin{equation}
\phi_y= \frac{A_0 (X-X_0 )\omega_0}{2(X-X_0 )^2+(\omega_0^2)/4}\cos\theta-\frac{\omega_0^2A_0}{(X-X_0 )^2+(\omega_0^2/4)}\sin\theta
\end{equation}
while the dispersive part of complex lorentzian is given by
\begin{equation}
\phi_x= \frac{A_0 (X-X_0 )\omega_0}{2(X-X_0 )^2+(\omega_0^2)/4}\sin\theta+\frac{\omega_0^2A_0}{(X-X_0 )^2+(\omega_0^2/4)}\cos\theta
\end{equation}
In this conjunction, $A_0$ represents the maximum of the purely
absorptive curve, $\omega_0$ the width of resonance (FWHM) and $x_0$
the center resonance frequency. Furthermore, x represents the
frequency values.  The overall fitting function in order to really fit
a complex lorentzian is given by (the phases in R(x) and D(x) are not
considered in the following equations, since they are varied
(sin/cos)).

{\bf What I think I am supposed to get from this discussion:}
\begin{itemize}
\item The fit functions being used.
\end{itemize}


\begin{equation}
F = R(x)^{'}~\theta(x \leq f_{max}) + D(x-f_{max})^{'}~\theta(x > f_{max})
\label{equation:fit funtion}
\end{equation}
Where 
\begin{equation}
R(x)^{'}=R(x) . \cos(\phi_0)~ + D(x) .\sin(\phi_0)
\end{equation}\\
and
\begin{equation}
R(x)^{'}=D(x) . \cos(\phi_0) - R(x) .\sin(\phi_0)
\end{equation}

Here, $\phi_0 $ represents the phase shift between the absorptive and dispersive parts. This phase shift $\phi_0$ appears due to the wrong settings of lock-in amplifier. The actual fitting function in~ Eq~ \ref{equation:fit funtion} includes a case
structure, as it is given by the Heaviside-like function $\phi(x)$. If $ (x \leq f_{max}) $ ,$\phi(x)$ becomes zero,
if $(x > f_{max})$, it is equal to one. The initial guesses for the least-square parameters are given
as:\\
\begin{itemize}
\item
Amplitude $A_0$: In order to get guess amplitude, Averaging the maximum of the absorptive and dispersive curve is done.
\item
Width $\omega_0$: A width of the resonance curve of 15 Hz is assumed.
value.
\item
Center frequency $x_0$: The maximum frequency value of the
absorptive curve is considered as a guess for the center frequency.
\item
Phase $\phi_0$: In order to get a guess for the phase, $ R =\sqrt (
X^2 + Y ^2)$ is calculated for each
X and Y data pair. The corresponding X and Y data pair which gives the maximum
R value is taken and the phase guess $\phi _0$  is calculated by $\phi _0 = arctan(Y_{max}=X_{max})$.
\end{itemize}
Advantages: In the force oscillation scan technique entire resonance curve is scanned which allow us to see if there is a pure resonance or the real resonance curve get destroyed by any other external influences. By mapping out the entire resonance curve, we can easily debug the problem because in this process we can repeatedly adjust the power of the pump and probe beam immediately after each scan. Since the output signal of the balanced photodiode is demodulated stepwise by the lock-in amplifier at the modulation frequency at each increment, small amounts of noise induce in this process which is the most advantageous point of this field measurement technique. \\
Disadvantages: The force oscillation scan is a quite slow process which is the main drawback of this measurement scheme.
For a scan range of some hundred Hz usually takes a few minutes with  waiting time of couple second. As a result the force oscillation scans are not directly sensitive to magnetic field drifts, occurring at time intervals which are shorter than the actual scan. Field drifts would cause a degradation of the precision of a swept oscillation scan.

%\subsection{Self oscillation mode}

% This should be in Chapter 2.

%In self-oscillation mode\cite{PhysRevA.62.043403}, the output signal
%of photodiode is fed back to AOM for amplitude modulation. In this
%case the output signal of polarimeter is modified to act as a square
%wave which drives the AOM directly. After setting the phase and gain
%of the feedback system properly, the system start to oscillates
%spontaneously at the Larmor frequency. In order to measure the
%oscillation frequency a frequency counter can be used in this mode. An
%online tracking of the oscillating signal is also possible.

%A customized circuit board, consisting of an analog voltage amplifier,
%a Schmitt trigger, and two metastable circuits, is used to process
%optical rotation.  {\bf No, it isn't.}

%Advantages: Being a quite fast process and having a high bandwidth are
%the main features of this self-oscillation scan. In order to get rapid
%update of the magnetic field the magnetometer can be operated in this
%mode.

%Disadvantages: Since feedback loop self-oscillate in the case of
%constructive interference, it will work only for the signal having a
%phase shift of integer multiple of $2\pi$. This additional phase shift
%might be resulting into slightly off-centered resonance. As a result,
%self-oscillation mode is more susceptible to systematic errors in
%field measurement.



\subsection{Free Indution Deecay(FID) }
\bigskip
\begin{itemize}
\item The magnetometer can also
be run in free induction decay mode,where Rb atoms inside the cell is excited once and afterwards the decaying processes of the excited atoms is observed. A function generator is used to delivered the pump pulses which are necessary for pumping during a FID measurement. The output frequency of the function generator is set to the resonance frequency
\item Pumping is done for a very short time interval and the coherence decay takes place fast. The pumping process is instantaneously stopped by AOM (by applying a digital TTL signal an acousto-optic modulator can be used to shutter a laser beam on and off), which can also be used to trigger the  data acquisition (DAQ). 
\item The change in the optical rotation of probe beam is measured by balanced  photodiode and the output signal of the photodiode is then fed into the lock-in amplifier.
\item The reference signal on the lock-in amplifier has further to be set slightly off resonance ($\sim 100$ Hz) in internal frequency mode in order to properly record the FID.
The X and Y channels output of the lock-in amplifier are further transfered to a Tektronix oscilloscope. A python script is used to transfer the data presented on the oscilloscope screen to computer for further analysis. 
\end{itemize}
The entire process of pumpimg and probing during a measurement cycle in FID mode  has shown in Figure 4.2. Rb atoms are pumped using amplitude modulated laser beam for 0.1 sec and then observed the spontaneous decay of excited atoms for another 0.2 sec while the pump beam was off. \\
Table 4.1 shows the function generator and lock-in amplifier settings for FID measurement.
\begin{table}[h]
\centering
\begin{tabular}{|l |l|}
\hline

\textbf{ SETTING}    & \textbf{VALUE} \\
\hline
Function generator &   \\
\hline
Frequency & 9.4kHz   \\

Waveform    &  Square  \\

Amplitude   &  $1V_{pp}$  \\
Offset  &       500 mV  \\
Phase       &    $0\degree$ \\
Trigger     &   Manual  \\
Burst       &    1000 cycle \\
Amplitude modulation & On \\
\hline
Lock-in amplifier &     \\
\hline
Lock in frequency     & 9.297 KHz \\
Time constant     &  $300\mu s$ \\
Sensitivity      &  500mV  \\
\hline
\end{tabular}
\caption{Setting for FID at $1\mu T$ field\label{table:FID setting}}
\end{table}
\begin{figure}[h]
\centering\includegraphics[width=0.55\linewidth]{figures/Capture2}
\caption{ An example of Free Induction Decay (FID) signal, when pumping the Rb atoms  was done for 0.1 s with linearly polarized light and probing was done for 0.2 s. The applied magnetic field during the measurement is $0.2~\mu T$. }
\end{figure}
The signal recorded  from X channel will be a sinusoid with an exponentially damping, according to   
  \begin{equation}
 y(t) = y_0 + A   e^{(t-t_0)/\tau}\sin(\omega t + \phi_0)\label{eq:decaying sinwave}
\end{equation}  

And the signal recorded  from Y channel will be a decaying cosine wave, according to
                                       
  \begin{equation}
 y(t) = y_0 + A   e^{(t-t_0)/\tau}\cos(\omega t + \phi_0)\label{eq:decaying coswave}
\end{equation}
where $y_0$ describes a possible offset, A is the maximum amplitude of the sinusoidal oscillation,
t the present time, $t_0$ the starting point of the measurement, $\omega$ the oscillation frequency and $\phi_0$  some possible phase shift. The data fitting procedure were done in two ways in order to study a variety of systematic effects that were encountered. One method is to take a least square fit of x and y data separately to a decaying sin and cosine wave respectively. Another way of data fitting is to fit X and Y data simultaneously. A least square-fit of the recorded data set to equation \ref{eq:decaying sinwave} or \ref{eq:decaying coswave} gives an estimate on frequency and  therefore the magnetic field.
\begin{figure}[h]
\centering\includegraphics[width=0.55\linewidth]{figures/fid_simultaneous}
\caption{ Simultaneous fit of X and Y signal(green and red dots are represents x and y data respectively whereas green and red curve represents fit curves\label{Fig:FID fit}}
\end{figure}
  

The initial guesses for the least-square parameters are given
as:
\begin{itemize}
\item
Beat frequency $\omega$: In order to guess the beat frequency, the Fast Fourier Transform (FFT) of FID signal is done. The extracted FFT frequency is used as guess frequency.
\item
Amplitude A: In order to get guess amplitude, Averaging the maximum of the X and Y data is done.
\item
Offset $\phi_0$: The mean of FID signal is calculated. This mean value is used as guess for the offset. 

\end{itemize}
The data fitting procedure provides us the actual value for those parameters. The oscillation frequency, one of the extracted fit parameters, is then is used to estimate the magnetic field according to the following equation
\begin{equation}
 B= \frac{\nu_{fit}~ +~\nu_0}{2\gamma}\label{eq:field}
\end{equation}
 where $\nu_{fit}$ denotes the oscilation  frequency, $\nu_{0}$ is lock-in frequency and $\gamma$ is the gyromagnetic ratio of Rb vapor. Fig \ref{Fig:FID fit} shows an example least square fit of FID signal where both x and y output of lock-in has displayed.
 
Advantages: FID mode is free of  pump light induced frequency shifts or instabilities because the optical pumping is done for a very short time period and the frequency measurement takes place quickly. This frequency is later translate into magnetic field.\\ 

Disadvantages: As a finite duty cycle is used (pump modulation only happens during a short period of time and the main idea is to watch the coherence decay, a decreasing of the maximum achievable sensitivity with this method.



\subsection{FID in tilted magnetic field}
For the nEDM experiment it is very important to do tilted field measurement in order to have better understanding of  geometric phase effects which are the leading sources for systematic uncertainties during nEDM measurement . 
Since our Rb magnetometer is a scalar magnetometer, it is not possible to measure vector field components directly.In general,  The same study with frequency modulated light has been reported by Pustelny et al.\cite{PhysRevA.74.063420}. In this study the NMOR Signal was observed by connecting the balanced photodiode to the oscilloscope directly. A python script is used to transfer the data presented on the oscilloscope screen to computer for further analysis. 

\begin{figure}
    \centering
 
    \begin{subfigure}[b]{0.45\textwidth}
        \centering
        \includegraphics[width=\textwidth]{figures/transverse_field}
        \caption{}
        \label{fig:transverse}
    \end{subfigure}
    \hfill
    \begin{subfigure}[b]{0.45\textwidth}
        \centering
        \includegraphics[width=\textwidth]{figures/transverse_field_2}
        \caption{}
        \label{fig:transverse2}
    \end{subfigure}
    \caption{(a) Optical rotation as a function of time at $\Omega_L$ in the yz plane at tilt angle $15\degree$ with light propagation direction. (b) Optical rotation as a function of time at $2\Omega_L$ for same tilt angle}
    \label{fig:Tilted field}
\end{figure}



% Chapter 5
\chapter{Experiments using the magnetometer\label{ch:results}}

In this Chapter, I present the main results on measurements of the
properties of the magnetometer, optimization studies, and applications
of the magnetometer to characterize magnetic fields.  The main studies
that are presented are:
\begin{itemize}
\item Measurements of magnetic fields over long timescales using FID mode.
%  Conclusion: fields drift over time.  Question: how much is due to
%  magnetometer drift vs. other sources?
\item Adjustment of the pump and probe timescales in order to measure
  the field faster.
%  Allowed us to measure faster.
\item Magnetometer drift compared with drifts in the coil current and
  room temperature.  This study aimed at finding sources of drifts.
%Conclusion: neither drift seems
%  particularly correlated with field drift over long time periods.
%  Temperature stability seems quite good, consistent with Michi's
%  thesis.  Side conclusion: when field {\bf only} is changed, the
%  magnetometer responds as expected.  Implies that magnetometer works
%  well at sensing small changes in field, at least on shorter
%  timescales.
\item Studies of degaussing, which in part tell the story of our
  degaussing development and learning.  This includes studies of:
%with multiple goals, telling the story of
%  our degaussing
  \begin{itemize}
    \item the degaussing setup and testing it near zero field
%we set up the
%      system, we tested mainly sample rate (related to number of
%      oscillations) stated to be important in Thiel et al.  We found
%      that if the innermost shield was already degaussed that
%      additional poor/rapid degauss did not screw it up as badly as we
%      expected, until degaussing was very rapid.
    \item initial operations at non-zero field, and
%Ramp field to large values, without
%      degaussing was bad.  After degaussing was better.
    \item final degaussing procedure, in which degaussing the next to
      innermost shield was studied.
  \end{itemize}
\item Studies of laser locking and tuning, and the requirements on
  tune stability
%again multiple goals:
%  \begin{itemize}
%    \item Lock point or laser seems to drift over time.  See ``Drift
%      is about 120 pT'' where statistical error gets worse.  Also
%      would lose lock sometimes.  We think this may be due to PBS.
%    \item Other concern is whether drift of lock affects measured
%      field.  Studied by ``manual locking'' and found not to be
%      important.  Lock drift does not affect measured field drift very
%      much, but does affect statistical precision of magnetometer.
%  \end{itemize}
\item Studies pushing below 1~pT in an individual FID measurement.
  This includes adjustment of the pump and probe powers, and lock-in
  amplifier settings.  As will be shown, this study revealed problems
  in the procedures used to determine the precession frequency at such
  high precision and suggests avenues for further study.
%But when
%  we improved the statistical precision significantly, we began to run
%  into systematic errors in frequency measurement.  This led us to
%  study additional errors related to lock-in amplifier settings.
%  Future work is to finalize these studies in order to further reduce
%  the errors
\item Finally, I show my studies which revealed a way to use FID mode
  to measure transverse fields.
%Further work is
%  required to push to nT-scale transverse fields relevant for
%  typ.~unmeasurable gradients that enter $\delta_T$ correction in Hg-n
%  signals in nEDM experiments.
\end{itemize}
Each study will now be presented in turn and conclusions will be
summarized in Chapter~\ref{ch:conclusion}.

% Belongs in Magnetometer Literature Review Chapter 2?

%. In the case of AM NMOR, high-field
%resonance occur in addition to the regular zero-field resonance. The
%optical properties of the medium are being modulated at twice the
%Larmor frequency. In the case of strong external magnetic field, the
%dynamic Stark effect limits the sensitivity of NMOR based atomic
%magnetometry by reducing the accuracy of the field measurement. The
%advantage of using AM NMOR method is that it can reduce the Stark
%effect because light frequency is not affected by amplitude
%modulation\cite{gawlikoptical}. In AM NMOR, It is easily possible to
%control the number and amplitudes of the high field resonances by
%using the square wave modulation of light intensity.




\section{Long-term FID measurements\label{sec:long-term}}

A forced-oscillation scan or an acquisition of a single FID give a
measurement of the magnetic field within a relatively short time
period ($<1$~s).  For nEDM experiments, it is important to get
information about the change in the average magnetic field over time
periods of 100~s and longer.

In order to study the fluctuations and drifts in the magnetic field
over time, and to search for possible drifts in the magnetometer
itself, repeated measurements of FID's were made and recorded.

%long
%term data was taken in the FID mode.
During this long term process the laser frequency was tuned for
maximal FID amplitude, and locked using the DigiLock 110 module.  In
order to observe the FID signal, a Tektronix DPO 2014 oscilloscope was
connected to X and Y output of output of lock-in amplifier. A Python
script was used to set up the function generator and to trigger data
acquisition using the oscilloscope. The same script is also used to
transfer data continuously from oscilloscope to the computer.  For
further data analysis another Python script was used to process each
FID.  A least-squares fit was done for each FID for each X and Y pair.
The measured oscillation frequency of the decaying oscillating signal
was then convert to magnetic field using Equation~(\ref{eq:field}).

The magnetometer settings for these runs were: Pump power $\sim
40~\mu$W, pump time 0.49~s, probe power $\sim 20~\mu$W, probe time
0.5~s, lock-in frequency is 1929.5~Hz, AOM frequency 2038~Hz, and
lock-in time constant 300$\mu$s.

\begin{figure}%[h]
\centering\includegraphics[width=0.85\linewidth]{figures/field_3_day}
\caption{Magnetic field recorded over 4 hours on three different
  days. The observed field drift is similar ($\sim 15$~pT) for each
  day.\label{fig:long-term-field}}
\end{figure}

Fig.~\ref{fig:long-term-field} shows the magnetic field recorded over
4 hours on three different days. Each data points in the graph
corresponds to a single FID measurement.  The observed field drift is
similar ($\sim 15$~pT) for each day.  The measurement was conducted at
0.2 $\mu$T magnetic field.

\begin{figure}%[h]
\centering\includegraphics[width=0.8\linewidth]{figures/field_3_day_allan.png}
\caption{Allan deviation of recorded magnetic field vs.~averaging time
  for the time-series data presented in in
  Fig.~\ref{fig:long-term-field}.\label{fig:allan_deviation}}
\end{figure}

The Allan deviation was used to further quantify the long-term
stability~\cite{doe:website2} (see also Appendix~\ref{cite:appendix}).
Allan deviations characterize changes in the measured quantity when
the data are averaged on different timescales.  When the data behave
statistically on short timescales, the Allan deviation is equal to the
standard deviation.  If drifts occur, normally on longer timescales,
the Allan deviation grows linearly with a slope that $1/\sqrt{2}$
times the slope of the drift in time.

Fig.~\ref{fig:allan_deviation} shows the Allan deviations of the
measurements field presented in Fig.~\ref{fig:long-term-field}.  The
minimum in the Allan deviation occurs when statistical behavior is
overtaken by drift.  Fig.~\ref{fig:allan_deviation} shows that this
transition generally occurs after 10-60~s of averaging, corresponding
to a precision in magnetic field of 300-500~fT at the Allan minimum.

For an nEDM experiment, the goal precision is $\sim 20$~fT for the
average field over as measured over the 100~s neutron free-precession
measurement cycle.  This is not likely to be equivalent to the Allan
deviation minimum of our one magnetometer, because the long-term drift
is driven in part by the drift of the magnetic field within the
shield.  The goal of subsequent work was:
\begin{itemize}
\item to attempt to identify some of the sources of drift.  This
  included searching any sources that might be caused by the
  magnetometer itself, but also included the effects of
\item to improve the single FID performance so that fields could be
  measured faster.
\end{itemize}
In terms of the Allan deviation, it means trying to move the Allan
minimum lower and to the right.

\section{Optimization of cycle time} 

The goal of this optimization study was to reduce the cycle time
without sacrificing too much precision in the single-FID frequency
measurement.  If more measurements can be made more quickly, the
precision of the magnetometer over time would be improved.

\begin{figure}
\centering
\begin{subfigure}[b]{0.46\textwidth}
  \centering
  \includegraphics[width=\textwidth]{figures/Capture}
  \caption{}
  \label{fig:pump-long}
\end{subfigure}
\hfill
\begin{subfigure}[b]{0.45\textwidth}
  \centering
  \includegraphics[width=\textwidth]{figures/FID_optimized.png}
  \caption{}
  \label{fig:pump-short}
\end{subfigure}
\caption{(a) FID signal for pump time 0.49~s and probe time 0.4~s. (b)
  FID signal for pump time 0.1~s and probe time 0.2~s.  Both
  measurements were conducted at $0.2~\mu$T magnetic field.}
    \label{fig:pump-time}
\end{figure} 


During this study the magnetometer was operated in FID mode at
0.2~$\mu$T field.
%A complete cycle of FID measurement consist of a
%pump time followed by a probe time.  Pump time represents the time
%atoms take to generate a polarized ground state. In this study we were
%trying to study how long the optical pumping of Rb atom should have
%continued in order to generate an alignment and optical pumping for
%long time does make any difference in measuring magnetic field
%precisely or not.
An example of our initial settings is shown in
Fig.~\ref{fig:pump-long}.  The pump time is 0.49~s and the probe time
is 0.4~s.  The amplitude of the differential photodiode signal is seen
to saturate well within 0.1~s.  The coherence time, indicated by the
decay time of the oscillating signal, is about 0.06~s.
Fig.~\ref{fig:pump-short} shows a more optimized the FID signal for
0.1~s pump time and 0.2~s probe time.
\begin{figure}%[h]
  \centering\includegraphics[width=\linewidth]{figures/pump_time_}
  \caption{Histograms of measured B-field for different pump times,
    for a measurement time of 100~s.  (a) is a pump time of 0.49~s (b)
    is a pump time of 0.39~s (c) is a pump time of 0.2~s (d) is a pump
    time of about 0.1~s.  The probe time in each case is about 0.25~s.
    The number of measurements taken in each case is estimated to be
    (a) 137 (b) 158 (c) 224 (d) 287.  The standard deviation of each
    set of measurements is indicated in the respective
    figure.}\label{fig:different-pump-time}
\end{figure}

In Fig.~\ref{fig:different-pump-time} a histogram of measured magnetic
fields by making subsequent measurements over 100~s is shown for
different pump times.  The longer pump time is 0.49~s
(\ref{fig:different-pump-time}(a)) and the shorter one is 0.1~s
(\ref{fig:different-pump-time}(d)).

From Fig.~\ref{fig:different-pump-time}, the pump time, when varied
over this limited range, does not strongly affect the precision of the
individual FID measurements.  However, it allows us to take
measurements faster because previously the duration for one FID
measurement was 1~s while after optimization it only take 0.35~s.

It is not surprising that the precision does not change much because
the amplitude of the differential photodiode signal during the pump
phase has saturated.

%In Fig.~\ref{fig:pump-long} the optical pumping is done for 0.49 s
%while signal amplitude reach its maximum over 0.1 s and after that
%signal amplitude remains unchanged till 0.49 s. In this case there is
%no point to pump more than 0.1 s. Same situation happens about probe
%time. If signal amplitude decays so quickly there is no point to set
%longer probe time. By optimizing pump and probe time we can speed up
%data acquisition system. After optimization for a single FID scan it
%only takes 0.35 s without losing any information. By using this faster
%data acquisition system it is possible to record multiple FID run with
%in a short period of time which is helpful to gather more information
%about magnetic field environment and helpful to achieve statistical
%precession.


\section{Magnet field compared with current in the $z$-coil}

\subsection{Current drift\label{sec:current-drift}}

\begin{figure}%[h]
\centering
\includegraphics[width=0.7\linewidth]{figures/current}
\caption{Recorded coil current (by measuring voltage across a
  100~$\ohm$ resistor in series with the $z$-coil) as a function of
  time when the current source has been set to
  4.50000~mA.\label{fig:current}}
\end{figure}

The idea was to determine the impact of possible current drifts.  An
Agilent B2962A power supply was used to supply DC electrical current
to the $z$-coil.  Fig.~\ref{fig:current} shows the current supplied to
the $z$-coil over a measurement period of 10000~s.  The current in the
power supply was set to 4.50000~mA, which was operated in a
current-controlled mode.  The current has been measured by measuring
the voltage drop across a Vishay metal-foild 100~$\ohm$ resistor in
series with the $z$-coil using a Keithley 2002 8-digit multimeter.  A
photograph of this system is shown in
Fig.~\ref{fig:Current_study_setup}, although with an alternate
resistor which we tried.  The Vishay resistor is 1\% absolute
precision, but the values reported in Fig.~\ref{fig:current} have been
converted to currents using an exact value for the resistance of
100.000~$\ohm$.  The main feature of this resistor is that it
possesses a low temperature coefficient of 2~ppm/$\degree$C.
Measurements recorded by this system are expected to be reliable on
the $\sim\pm 2$~ppm level given the resistor and multimeter being used
and the typical temperature fluctuations in the room.


\begin{figure}%[h]
\centering
\includegraphics[width=\linewidth]{figures/current_study_setup.png}
\caption{Photograph of equipment arrangement used to control and
  measure the current in the $z$-coil.\label{fig:Current_study_setup}}
\end{figure}

The current of 4.5~mA achieves a magnetic field of 0.2~$\mu$T.  From
Fig.~\ref{fig:current}, the maximum current excursion during the
measurement period was 300~nA.  Likely these fluctuations are given by
the stability of the power supply. Converted to magnetic field, it
would represent a $\sim 10$~pT fluctuation in field.  Unfortunately,
during this measurement time, the magnetometer and degaussing system
could not be operated sufficiently well in concert with the current
measurement system to determine whether or not the current
fluctuations were correlated to field changes.  Principally this was a
problem of the data acquisition system.  Clearly, field changes as
large as 10~pT should be observable, given the statistical precision
of the magnetometer.  Future work would be on improving the data
acquisition system.

%\begin{figure}
%  \centering
%  \begin{subfigure}[b]{0.67\textwidth}
%    \centering
%    \includegraphics[width=\textwidth]{figures/field_coil_current.png}
%    \caption{}
%    \label{fig:field_measure_and_produced}
%  \end{subfigure}
%  \begin{subfigure}[b]{0.65\textwidth}
%    \centering
%    \includegraphics[width=\textwidth]{figures/field_current_allan_plot.png}
%    \caption{}
%    \label{fig:allan_plot}
%  \end{subfigure}
%  \caption{(a) (Blue) Measured current in the $z$-coil converted to
%    field, using the normalization constant measured at 0~s.  (Red)
%    Measured magnetometer frequency converted to field using the
%    gyromagnetic ratio of $^{85}$Rb.  (b) Allan deviation of of the
%    time series in (a).}
%  \label{fig:current_vs_field_allan_deviation}
%\end{figure} 

%Fig.~\ref{fig:field_measure_and_produced} displays the magnetic field
%measured by the magnetometer over the same period.  The magnetometer
%was operated in FID mode as usual and the magnetometer settings were
%similar to those used in Section~\ref{sec:long-term}.
%%{\bf list all settings, lock-in time constant, laser power, lock-in
%%  frequency, magnetometer measured frequency, etc.}
%%produced by z-coil(green line) and measured field (red line) by
%%running the magnetometer in FID mode over 8000 s.
%The field produced by the $z$-coil is calculated by converting
%measured current to field with the coil constant measured at $t=0$~s.
%At $t=0$ both fields are therefore forced to coincide and over time a
%linear drift can be seen in the field measured by the magnetometer,
%while the current in the $z$-coil would indicate that the field should
%not be changing.  Interestingly, aside from the overall drift, there
%might be some correlation of excursions in the current with excursions
%in the magnetometer reading.  The interpretation is somewhat obscured
%by the drift.  Also the exact time synchronization of the current data
%and the field measurement data was not tested precisely, and may have
%some relative time drift.

%For a better understanding of the stability of the power supply the
%Allan deviation of the field produced by the $z$-coil (deduced by its
%current) and the field measured by the magnetometer are shown in
%Fig.~\ref{fig:allan_plot}.  It can be seen from the figure that for
%long times, the current stability is better than typical measured
%field changes.  At short times, the current stability is considerably
%worse than the field stability.  This might be due to additional noise
%in the current measurement system, or a different bandwidth being used
%for the current measurement system.


%{
%\bf How is the coil current measured? 
%%Coil current was measured by using a Keithley 2002 multimeter.
%Why was the coil current
%  measured for longer than the field?  Was the coil current averaged
%  over the same time as each frequency measurement? 
%  % yes averaging time is same for current and frequency.
%  Why is the
%  current noisier than the magnetometer for short averaging times in
%  the Allan deviation? 
%  
%  How were the current and field data acquired?
%  How were the data synchronized to one another in time?  How is the
%  measurement time defined for the coil current?  Is it at the start
%  of the FID measurement or the end of the FID measurement, or
%  somewhere in the middle, or asynchronously?  For the FID
%  measurements, which time is graphed?  The start, end, or half-way
%  time, or at the 1/e time of the FID?  Please show the measured coil
%  current in the next section as well for
%  Fig.~\ref{fig:field-change}.}



\subsection{Change in magnetic field driven by larger changes in coil current}

A study was conducted to determine the field change by changing the
coil current by a known amount. The main objective of this study is to
confirm the Rb magnetometer performance on magnetic field measurement.
For this measurement coil current was changed periodically by $\pm
0.001$~mA steps.  Based on the coil constant, field changes of 48~pT
were expected to be observed in the magnetometer, corresponding to
frequency shifts of 0.5~Hz.

The laser frequency was tuned for maximum FID amplitude, then locked
using the DAVLL system.  All other magnetometer settings were similar
to the previous studies in Section~\ref{sec:current-drift}.  During
this study the magnetometer has been operated in FID mode.  The
lock-in frequency was 1950 Hz and time constant was 300~$\mu$s.  The
AOM frequency was tuned initially to maximal optical rotation and
found to be 2049~Hz.  The probe beam power was 20~$\mu$W and pump
power $\sim 40~\mu$W.  The pump and probe times were both 0.5~s.  Only
X data was fitted to determine the magnetometer frequency.

\begin{figure}%[h]
\centering
\includegraphics[width=0.7\linewidth]{figures/field_change_with_current}  
\caption{Change in magnetic field measured by the magnetometer by step
  changes in the $z$-coil current.  In Regions 1, 3, and 5, the
  current was set to 4.50000~mA on the power supply.  In Region 2, the
  current was changed to 4.50100~mA.  In Region 4, the current was
  4.44900~mA.\label{fig:field-change}}
\end{figure} 
 
Fig.~\ref{fig:field-change} presents the magnetic field change over
time for different coil current. For better understanding
Fig.~\ref{fig:field-change} has been divide into five different
regions where a number of different coil currents were set near
4.50000~mA in steps of 0.00100~mA.  It is clear from the figure that
the field has been changed by $\sim 50$~pT on each transition.  This
measurement proves rather conclusively that changes in the magnetic
field with large current changes are indeed reproduced correctly.  It
should be noted that up to 30~s of data of been omitted near each
transition.


\section{Study the effect of room temperature in magnetic field} 

The goal of this study was to search for any correlation between the
magnetic field measured by the magnetometer and ambient temperature
flucturations near the magnetometer.

\begin{figure}%[h]
\centering\includegraphics[width=0.8\linewidth]{figures/temp_.png}
\caption{Temperature measurements as a function of time.  The upper
  graph (blue) shows the temperature measured on the outermost
  magnetic shield.  The lower graph (red) shows the temperature in the
  room outside the laser hut, near the power supply and function
  generator system.\label{fig:temperature-measurement}}
\end{figure}

Fig.~\ref{fig:temperature-measurement} displays the measured
temperature at various locations vs.~time.  It can be seen that room
temperature fluctuations are 1\degree{C} while temperature measured on
the outside of the magnetic shielding is about 0.8\degree{C}.  The
optics table is covered by a clean enclosure, which makes the
temperature measured on the top of the magnetic shield slightly warmer
than the rest of the room.  The temperature on top of the shield is
measured by a T-type thermocouple that was held to the shield using
tape and thermally connected by thermal grease.  We often use T-type
thermocouples because they are non-magnetic, as opposed to K-type
which are magnetic.

For room temperature measurement a precision thermometer was connected
to a Agilent 34410A 6$\frac{1}{2}$ Digit Multimeter.  A Python script
configured the multimeter for a 2-wire RTD measurement, triggered the
meter, and transferred the reading to the computer.  The T-type
thermocouple on the shield was connected to an Arduino Uno.  For
configuring this device and then transferring the reading to the
computer a different part of the same Python script was used.  After
an acquisition of the two temperatures, the script paused for 10~s, so
that measurements came approximately every 12.3~s.

%The laser diode temperature was read by the output of the temperature
%control unit for the Toptica laser using another oscilloscope and
%converted to temperature using a known coefficient.  Since it did not
%vary with time it was not considered further.

\begin{figure}%[h]
\centering\includegraphics[width=0.8\linewidth]{figures/field_.png}
\caption{Field measurement\label{fig:field}}
\end{figure}

Fig.~\ref{fig:field} shows the magnetic field measured by the
magnetometer in FID mode vs.~time in similar times as the temperature
measurement was made.  The magnetometer settings were the same as in
the previous section, with the caveat that the frequency of the AOM
was adjusted for best FID amplitude (2036~Hz) and lock-in adjusted to
keep the FID demodulated frequency near 90~Hz (1929.5~Hz).

The FID measurements were taken using an oscilloscope.  The
oscilloscope would be read occasionally.  No data is shown twice in
Fig.~\ref{fig:field}, but time jumps in the data did occur at random
intervals.  Thus any correlation with temperature is qualitative.  The
meausrements were taken over approximately the same time as the
temperature measurements.

The asynchronous data acquisition systems mean that time was defined
with factor of 2 level accuracy.  Nonetheless, a the correlation of
the field measurement is ruled out at approximately the
5~pT/\degree{C} level.


%Temperature change and field change, the measured field
%vs.~temperature has shown in Fig.~\ref{fig:field_vs_temp}. {\bf What
%  was the data acquisition system for each device?  How were the data
%  synchronized?%   }  It is clear from the graph
%that there is no obvious of field drift dependency on temperature.

%\begin{figure}%[h]
%\centering\includegraphics[width=0.6\linewidth]{figures/field_vs_temp.png}
%\caption{Field vs. temperature\label{fig:field_vs_temp}}
%\end{figure}
 
   
\section{Degaussing studies\label{sec:degaussing}}

\subsection{Initial tests using the magnetometer near zero field}

%different degaussing schemes(changing sample rate)
   
  
%Degaussing process is done in order to avoid the environmental
%perturbation and reduce any remnant field inside the four layer $\mu$
%metal magnetic shield \cite{doi:10.1063/1.2713433}. The setting for
%degaussing has shown in Table \ref{table:degaussing-setting}.  A
%detailed analysis of NMOR signal by changing the degaussing parameter
%could give us some information about the goodness of degaussing
%procedure. Toward this end, a study has been carried out to determine
%the dependency of the resonance width on the sample rate (see
%discussion on section \ref{sec:Degaussing}).

We used the procedure described in Section~\ref{sec:near zero field}.
To remind the reader, the procedure is:
\begin{itemize}
\item Initiate the degaussing sequence in one channel of the function
  generator.
\item Once complete, ramp the rheostat to zero and open the switch.
\item Pressing a button on the computer initiates the ramp of the
  $z$-coil on the second channel of the function generator, which
  calibrates the optical rotation to field.
\item Both the current in the $z$-coil and the differential photodiode
  signal are monitored at all times using an oscillscope.
\end{itemize}

%In this study, data was acquired by sweeping the magnetic field near zero field. Before each measurement degaussing the innermost layer of shield has done. During this measurement only one degaussing parameter, sample rate, was varied in order to study the effect of degaussing parameter on magnetic field measurement.  

\begin{figure}%[h]
  \centering\includegraphics[width=\linewidth]{figures/sample_rate}
  \caption{ Optical rotation vs. measured B-field for different sample
    rate.  The data were taken in order from smaller sample rate to
    higher sample rate.  The numbers written in the red circles
    indicate the distance in magnetic field from the optical rotation
    minimum to the optical rotation maximum $\Delta B$, as deduced
    from the fit parameter $a$.  The fit parameter $B_0$ indicates the
    remnant field sensed by the horizontal offset of the dispersive
    shape, reported in nT.\label{fig:different-sample-rate}}
\end{figure}

Optical rotation as a function of magnetic field for different sample
rate is shown in Fig.~\ref{fig:different-sample-rate}.  Recall that
the sample rate determines the rapidity with which the $5\times 10^5$
individual samples of the linear degaussing envelope function are
stepped through.  A sample rate of 10,000/s therefore represents a
degaussing time of 50~s.  Since the carrier wave in all cases is
10~Hz, it means that 500~cycles were used.  The sample rate of
80,000/s represents a degaussing time of 6.25~s and 62.5 cycles.

It should also be noted that the data were acquired in order of
increasing sample rate, and that many additional degaussing sequences
were conducted which are not shown in
Fig.~\ref{fig:different-sample-rate}.

We had expected that larger sample rate would manifest itself as a
worse degaussing resulting possibly in worse magnetometer performance.
Paradoxically, the resonance width $\Delta B$, the difference between
two peak of the dispersive curve, reduced very slightly as the sample
rate was increased.  It can be seen from
Fig.~\ref{fig:different-sample-rate} that, the resonance width is
about 0.49~nT for sample rate 10,000/s and the resonance width is
0.38~nT for sample rate 80,000/s.  This is likely an indication of a
small reduction in the transverse fields or generally an improvement
the homogeneity of the field.  This is consistent with the observation
that the amplitude grows slightly, which is another indication of the
improved field quality, all other magnetometer settings being equal.

The remnant field $B_0$ is an indication of the average longitudinal
field (along the laser beam axis) and is within 20~pT of zero.  It
increased with the sample rate from a starting negative value to a
positive value.

The conclusion of this study was that additional degaussing, as long
as it has a reasonable number of cycles, tends not to affect the field
or it might improve slightly the homogeneity of the field.  Generally
the field is reduced below 20~pT.

%\begin{figure}%[h]
%\centering\includegraphics[width=0.6\linewidth]{figures/field_vs_sample_rate}
%\caption{Resonance width vs.~sample rate. Resonance width decreases
%  with increasing sample rate.  When the sample rate is 5000~samples/s
%  the observed resonance width is 0.55~nT. On the other hand the
%  resonance width is 0.38~nT for sample rate 80000
%  sample/s. \label{fig:resonance width vs. sample rate} }
%\end{figure}

%Fig~\ref{fig:resonance width vs. sample rate} shows the resonance
%width as a function of sample rate. Resonance width decreases with
%increasing sample rate.  It is obvious from the graph that the
%resonance width becomes narrower for larger sample rate.

\subsection{Measurements at 0.2~$\mu$T and initial studies of the effect of degaussing on field stability\label{sec:three-degauss}}
%Ramp field up/down 

The previous results implied that additional (even poor) degaussing
had little impact if the previous degaussing was done adequately.  To
address this, we began to do studies where the internal $z$-coil field
was purposely ramped to a large value, then reset to a low value for
measurements at non-zero field.  The idea here was to use the $z$-coil
itself to magnetize the shields, and to measure drifts at non-zero
field, which is the chief interest for nEDM experiments.

%A study has conducted by making the magnetic field environment bad
%inside the shielding intentionally by ramping the field from a lower
%value to higher value and observe any change in field drift. This
%study has done in 3-steps:
%  \begin{itemize}
%      \item take field measurement for about 2000 s without performing any degaussing.
%      \item Intentionally make the field environment bad and take field measurement for another 2000 s. No degaussing has done in this step also.
%      \item perform degaussing of the innermost shield and again measured magnetic field for 2000 s.
%  \end{itemize}

\begin{figure}
  \centering
  \begin{subfigure}[b]{0.49\textwidth}
    \centering
    \includegraphics[width=\textwidth]{figures/ramp_1}
    \caption{}
    \label{fig:ramp-up}
  \end{subfigure}
  \hfill
  \begin{subfigure}[b]{0.49\textwidth}
    \centering
    \includegraphics[width=\textwidth]{figures/ramp_2}
    \caption{}
    \label{fig:ramp-down}
  \end{subfigure}
  \begin{subfigure}[b]{0.49\textwidth}
    \centering
    \includegraphics[width=\textwidth]{figures/ramp3}
    \caption{}
    \label{fig:degauss}
  \end{subfigure}
  \caption{Measurements conducted in time order from (a) to (b) to (c)
    under different degaussing conditions at $B_z=0.2~\mu$T magnetic
    field: (a) no degaussing, (b) ramp $B_z$ from 0.2~$\mu$T to
    10~$\mu$T then again set it to 0.2~$\mu$T and collect FID signal
    before degaussing, and (c) after degaussing.}
  \label{fig:ramp-updown}
\end{figure}

Fig.~\ref{fig:ramp-updown} shows an example of such a study.  In all
three cases shown in Fig.~\ref{fig:ramp-updown}, the magnetometer
settings are similar in each case.  The AOM frequency was adjusted
slightly to optimize the FID amplitude for each measurement, but the
lock-in frequency and most other settings were left the same.
%Since the field changed in each measurement, the magnetometer was
%retuned each time, principally the pump frequency and the lock-in
%amplifier internal reference frequency.  {\bf What are the
%  magnetometer settings?}
 % AOM frequency 2064 Hz and lock-in frequency 1950.2 Hz (fig:5.12(a))
 % AOM frequency 2059 Hz and lock-in frequency 1950.2 Hz (fig:5.12(b))
 % AOM frequency 2078 Hz and lock-in frequency 1950.2 Hz (fig:5.12(c))
  
In the 1st part of this study the magnetometer was operated in FID
mode at 0.2~$\mu$T field, {\bf at a point in time when the
  magnetometer had been operated in a quiet environment for several
  days?}. The recorded magnetic field measurement over a 2200~s period
in this condition is shown in Fig.~\ref{fig:ramp-up}.  It can be seen
from Fig.~\ref{fig:ramp-up} that the magnetic field increased linearly
for first 300~s then showed a decrease in field. The field was pretty
stable between 600~s and 1800~s and after that the field showed an
increase. The overall field change is about 15 pT during the
measurement.  In Fig.~\ref{fig:ramp-down}, data was acquired by
ramping $B_z$ from 0.2~$\mu$T to 10~$\mu$T then down again to
0.2~$\mu$T and collect FID signal.  Since the $z$-coil is coupled to
the inner shield system, this should magnetize the shields.  By doing
this field ramping we intentionally perturb the magnetic field
environment inside the shield.  After this field ramping the long term
field measurement has been conducted for another 2200 s. In this case,
the field showed a downward drift of about 35~pT.  Thus field ramping
did seem to change the magnetic environment inside the magnetic
shielding.

Finally, for Fig.~\ref{fig:degauss}, the innermost
magnetic shield was degaussed  with degaussing parameters stated in Table \ref{table:degaussing-setting}
Before starting field measurement via FID mode degaussing was done multiple times (4-5 times) repeatedly but during the measurement no degaussing was done. 
  
  While this changed the direction of the
drift, it did not change the magnitude of the drift which was again
35~pT over 2200~s.  {\bf Is it true that no amount of degaussing of
  the innermost shield changed this significantly?}

During this time we began to develop a hypothesis about magnetic
couplings inside the magnetic shielding.  We realized that the
$z$-coil, which had been designed to be coupled to the innermost
shield, was now more strongly coupled to the second innermost shield.
This is because most of the flux in the solenoidal winding exits the
end of the innermost shield, whose endcaps have been removed in order
to accommodate its degaussing coil.


\subsection{Effect of degaussing the three outermost shields}
 
If the system had been kept at low field for several days, the usual
field drift was about 15 pT over 3000~s (as discussed in
Sections~\ref{sec:three-degauss} and~\ref{sec:long-term}).  After
doing a transverse field study (applying current to $x$- and $y$-coils
along with $z$-coil, like that reported in
Sections~\ref{sec:ch4_tilted_field} and~\ref{sec:tilted-results}) the
magnetic field environment became even more unstable.

\begin{figure}
  \centering
  \begin{subfigure}[b]{0.5\textwidth}
    \centering
    \includegraphics[width=\textwidth]{figures/before_degaussing}
    \caption{}
    \label{fig:with DG_innermost}
  \end{subfigure}
  \hfill
  \begin{subfigure}[b]{0.48\textwidth}
    \centering
    \includegraphics[width=\textwidth]{figures/after_degaussing}
    \caption{}
    \label{fig:with DG}
  \end{subfigure}
  \caption{Magnetic field vs.~time (a) after degaussing the innermost
    shield, and (b) after additionally degaussing the remaining layers
    of magnetic shielding with a single loop of wire wound through all
    three layers. During this study the signal amplitude was
    relatively low ($\sim 3$~V) likely due to poor laser alignment
    through the AOM or other settings. As a result the fluctuations in
    magnetic field were larger than the usual 1-5~pT~\label{fig:effect
      of DG}}
\end{figure}

Fig.~\ref{fig:with DG} shows about 50~pT drift in magnetic field in
4000~s. Although before start taking field measurement degaussing
innermost layer of mu-metal is done multiple times in order to cancel
background field inside the shield but still the field drift problem
can be seen. Then degaussing the other three shielding layers is done
with a single loop of wire wound through all three.  The maximum
current achieved in this coil was equivalent to the maximum current in
the innermost shield deguassing sequence.  However since a single loop
is wound through all three outermost shields, it is unlikely that any
of the three shielding layers are ever saturated.  Nonetheless, this
seemed to reduce the field drift after degaussing the outer three
shielding layers, as shown in Fig.~\ref{fig:with DG}). During those
measurements the lock-in reference frequency was set to 1938.5~Hz and
lock-in time constant is 300~$\mu$s.

Since the end cap of the innermost shield layer was removed, the
$z$-coil may magnetize the second-to-innermost shield.  We expect this 

\section{Laser tuning} 

In the course of tuning the laser for maximum FID amplitude, some
studies were conducted where the laser was purposely mistuned in order
to check the effect on the measured FID fit parameters.  In general,
for the particular magnetometer settings otherwise in use, it was
found that detuning from the maximum amplitude tended to increase the
coherence time (1/e decay time of the oscillating FID signal),
although on occasion shorter coherence times could be observed.  By
taking measurements in rapid succession, it was also found that the
FID frequency did not change considerably at the 0.01~Hz (sub-pT)
level.

We started doing studies like this because there was some (mistaken)
belief that the laser tune significantly altered the FID measured
frequency.  This was based on data like that shown in
Fig.~\ref{fig:digilock-drift}.

\begin{figure}%[h]
  \centering\includegraphics[width=0.6\linewidth]{figures/field_drift}
  \caption{Magnetic field recorded over 12~hours.  In this
    measurement the observed field drift is
    about~120~pT.\label{fig:digilock-drift}}
\end{figure}

Fig.~\ref{fig:digilock-drift} indicates a 120~pT drift in magnetic
field over 12~hours.  The laser frequency is believed to drift because
the amplitude of the FID decreased by a factor of 2.5 over the course
of the measurement, while the resonant frequency only changed by
1.5~Hz.  The fact that the amplitude has decreased is indicated by the
relevant fit parameter which is not shown, but can be seen from the
additional statistical noise on the field measurements at long times
compared to short times.  The question was if this possible drift in
tune was causing any frequency drift or not.

We suspected the tune might be drifting because we use a polarizing
beamsplitter cube in our DAVLL system which may possess poor stability
over long time periods~\cite{bib:Philip2008}.

In order to test the hypothesis that the frequency of the laser might
be drifting, I did a study where I manually adjusted the frequency of
laser light to maximize optical rotation.  The DAVLL and Digilock
system were not used to lock the laser during this study.  Rather,
periodically, the data acquisition was paused, and the laser frequency
was adjusted manually to maintain the same (largest) differential
photodiode amplitude. The data acquisition was then restarted.

\begin{figure}
  \centering
  \begin{subfigure}[b]{0.48\textwidth}
    \centering
    \includegraphics[width=\textwidth]{figures/manual_tuning}
    \caption{}
    \label{fig:field-manual-tuning}
  \end{subfigure}
  \hfill
  \begin{subfigure}[b]{0.48\textwidth}
    \centering
    \includegraphics[width=\textwidth]{figures/amplitude_manual_tuning}
    \caption{}
    \label{fig:amplitude-manual-tuning}
  \end{subfigure}
  \caption{(a) magnetic field vs. time. (b) {\bf The vertical scale
      appears to be labeled incorrectly (units)} amplitude of recorded
    FID NMOR signal over 20000~s. During this study laser tuning is
    maintained manually, as can be seen from the jumps in the
    amplitude which occur upon retuning.}
  \label{fig:manual-tuning}
\end{figure}
   
The results of the measurement are shown in
Fig.~\ref{fig:manual-tuning}.  Fig.~\ref{fig:amplitude-manual-tuning}
shows the fitted initial amplitude (after the pump beam has been
switched off) as determined by the fit to the decaying oscillating
function.  The spikes in Fig.~\ref{fig:amplitude-manual-tuning}
indicate times when the retuning of the laser was conducted, as
described above.  A time jump (not shown) of a random by relatively
short interval also occurs for each retuning, which is not shown.

Fig.~\ref{fig:field-manual-tuning} shows the result of the frequency
measurement, when translated into magnetic field.  It can be seen that
as the amplitude of the fit decreases (presumably due to a drift in
the laser frequency), the statistical fluctuations in the field
measurement increase.  This would be expected because a small fit
amplitude will make the frequency statistically more difficult to
measure.

Clearly the field measurement in Fig.~\ref{fig:field-manual-tuning}
does not drift in the same way as the FID amplitude in
Fig.~\ref{fig:amplitude-manual-tuning}.  Thus we conclude that the
field measurement in FID mode is relatively insensitive to the
frequency tuning of the laser, except that if the tune drifts, the
frequency measurement becomes statistically less precise.

%we can say that during this measurement the amplitude
%of FID NMOR was pretty stable except small fluctuations over 20000 s
%. The observed small fluctuation is due to the manual adjustment while
%keeping the laser frequency tuned to atomic transition over long
%period of time. Obtaining stable signal amplitude is a indication that
%laser is not drifting to much. Although laser frequency was not
%drifting a lot during the measurement, still there is a drift in
%magnetic field (Fig.~\ref{fig:manual-tuning}). The overall drift is
%about 20~pT over 20000~s. So it can be conclude that the drift in
%laser frequency is not the main reason behind this induced field
%drift.


\section{Improving the precision of individual FIDs, and problems encountered in doing so\label{sec:reference-frequency}}

\subsection{Optimization of pump and probe beam power} 
A study has been performed to determine how optimization of the pump
and probe beam power effect on precise field measurement.  During this
study, the magnetometer has operated in FID mode.
\begin{figure}
  \centering
  \begin{subfigure}[b]{0.7\textwidth}
    \centering
    \includegraphics[width=\textwidth]{figures/beam_power_less}
    \caption{}
    \label{fig:power less}
  \end{subfigure}
  \begin{subfigure}[b]{0.7\textwidth}
    \centering
    \includegraphics[width=\textwidth]{figures/beam_power_double}
    \caption{}
    \label{fig:power double}
  \end{subfigure}
  \caption{ Magnetic field as a function of time for different power
    of probe beam: (a) 15~$\mu$W, the standard deviation of the data
    is 3.01~pT (b) 30~$\mu$W, the standard deviation is
    7.2~pT.\label{fig:different probe power}}
\end{figure}
%data colllected 13th August 2018
 
In Fig.~\ref{fig:different probe power} the measured magnetic field
has been displayed for two different probe beam
power. Fig.~\ref{fig:power less} presents the measured field over
100~s with 15~$\mu$W probe power while Fig.~\ref{fig:power double}
display the measured field over 100~s for probe power 30~$\mu$W. In
both cases, the pump beam power the same ($\sim 40~\mu$W). Each blue
points in this graph represent a single FID scan. As can be seen from
Fig.~\ref{fig:different probe power} the magnetic field were more
scattered for high probe beam power.  The standard deviation of the
data was 7.2~pT for probe beam power 30~$\mu$W while for low beam
power of 15~$\mu$W the standard deviation was 3.01~pT.

\begin{table}%[h]
\centering
\begin{tabular}{|c|c|c|}\hline
\textbf{Probe power ($\mu$W)}    & \textbf{Coherence time (ms)}  & \textbf{Amplitude (V)}\\\hline
2 & 84 & 2.13   \\
5    & 82 & 2.11  \\
10   &  68 & 4.06 \\
15  &   59 & 5.8  \\
35  &   32 & 10  \\\hline
\end{tabular}
\caption{Single FID fit parameters as a function of probe beam
  power.\label{tab:coh}}
\end{table}

The apparently worse statistical precision for higher probe beam power
can be traced in part to its effect on the individual FID signals of
each measurement.  Table~\ref{table:coh} displays the coherence time
of the FIDs for different probe power while the pump beam power was
held constand (120~$\mu$W).  For these settings, the coherence time is
seen to increase for descreasing probe power, which acts to improve
the statistical precision.  However, it is counteracted by a
corresponding decrease in the FID amplitude which acts to worsen the
statistical precision.  The trade off of these two effects must be
considered when selecting the probe power for a given pump power.  In
general it is expected that a probe power in the $\mu$W range would be
optimal.
 
The effect of pump beam power was also studied with the probe beam
power ($\sim 15~\mu$W) held constant.  The pump beam power was changed
by changing the duty cycle of the square wave modulation used for the
AOM.  When duty cycle is set to 5\% the pump beam power is roughly
40~$\mu$W and for 16\% duty cycle power is roughly 130~$\mu$W. During
this measurement the lock-in reference frequency was set to 1943.9 Hz
while the AOM frequency was 2036~Hz.

\begin{figure}
  \centering
  \includegraphics[width=\textwidth]{figures/pump_beam}
  \caption{Magnetic field vs. time for different pump beam power. (a)
    measured magnetic field with duty cycle $5 \%$. Precession width
    is about $2.3$ pT (b) measured magnetic field with Precession
    width $4.6$ pT for 16 $\%$ duty cycle.}
    \label{fig:different pump power}
\end{figure}

The histogram of measured magnetic field for two different pump beam
powers is shown in Fig.~\ref{fig:different pump power}. The standard
deviation becomes larger for higher duty cycle while it becomes
narrower for small duty cycle.  It is clear from the plot that the
data points are not statistically distributed.  

It was observed very clearly during these studies that when a pump
beam with a higher duty cycle was used, the FID amplitude increased
until a saturation point.  The data in Fig.~\ref{fig:different pump
  power}(b) are close to this saturation point.  But the problem of
non-statistical behavior is also more apparent.  So it can be
concluded that using higher duty cycle is likely statistically more
robust, revealing new systematic errors in field measurement which led
to the next studies.

\subsection{Reference frequency of lock-in amplifier}
  
The measurement technique of FID signal was discussed in
Section~\ref{sec:FID}.  In order to capture the FID signal, the
reference frequency of lock-in amplifier is normally set to $\sim
100$~Hz apart from the resonance frequency. In this study we kept all
other setting fixed only changed the Lock-in reference frequency and
observed the effect of that on field measurement study. The
measurement was conducted at $0.2~\mu$T field.
   \begin{figure}
    \centering
    \begin{subfigure}[b]{0.4\textwidth}
        \centering
        \includegraphics[width=\textwidth]{figures/reference_frequency1}
        \caption{}
        \label{fig:far from resonance}
    \end{subfigure}
    \hfill
    \begin{subfigure}[b]{0.4\textwidth}
        \centering
        \includegraphics[width=\textwidth]{figures/reference_frequency3}
        \caption{}
        \label{fig: middle range}
    \end{subfigure}
    \begin{subfigure}[b]{0.4\textwidth}
        \centering
        \includegraphics[width=\textwidth]{figures/reference_frequency2}
        \caption{}
        \label{fig:close to resonance}
    \end{subfigure}
 \caption{FID signal for different reference frequency of Lock-in amplifier while the resonance frequency was 2035 kHz.(a) reference frequency was set to 92 Hz far from resonance (b) difference between resonance and reference frequency is 40 Hz, (c) reference frequency was set to 2015 Hz while resonance frequency 2035 Hz. \label{fig:different reference signal}}
\end{figure}

  \begin{figure}[h]
\centering\includegraphics[width=0.8\linewidth]{figures/freq_1943_single_fit_300microsec.png}
\caption{Lock-in reference frequency and time constant are 1943 Hz and 300$\mu$s respectively. The resonance frequency is 2035 Hz. In this case only x data was fitted. \label{fig:freq_1943_single_fit_300_micros}}
\end{figure}


  \begin{figure}[h]
\centering\includegraphics[width=0.8\linewidth]{figures/freq_1943_simultaneous_fit_300microsec_.png}
\caption{Lock-in reference frequency and time constant are 1943 Hz and 300$\mu$s respectively. In this case both x and y data was fitted simultaneously. \label{fig:freq_1943_simultaneous_fit_300_micro_sec}}
\end{figure}

 \begin{figure}[h]
\centering\includegraphics[width=0.8\linewidth]{figures/freq_2015_simultaneous_fit_300microsec.png}
\caption{Lock-in reference frequency and time constant are 2015 Hz and 300$\mu$s respectively. In this case both x and y data was fitted simultaneously. \label{fig:freq_2015_simultaneous_fit_300_micros}}
\end{figure}

  \begin{figure}[h]
\centering\includegraphics[width=0.8\linewidth]{figures/freq_1943_simultaneous_fit_1ms.png}
\caption{Lock-in reference frequency and time constant are 1943~Hz and 1~ms respectively.  \label{fig:freq_1943_simultaneous_fit_1ms}}
\end{figure}

  \begin{figure}[h]
\centering\includegraphics[width=0.8\linewidth]{figures/freq_2015_simultaneous_fit_1ms.png}
\caption{Lock-in reference frequency and time constant are set to 2015~Hz and 1~ms respectively. \label{fig:freq_2015_1ms}}
\end{figure}


 
  The FID signal for different reference frequency of lock-in amplifier has shown in Fig \ref{fig:different reference signal}. In the case of Fig \ref{fig:far from resonance} the reference frequency of lock-in was set to 1943.9 Hz (90 Hz far from resonance frequency). On the other hand for Fig \ref{fig:close to resonance} the reference frequency of lock-in was set to 2015 Hz (20 Hz far from resonance frequency). In this study the resonance frequency was always set to 2035 Hz , pump power(~60 $\mu$w) and probe power (~30 $\mu$w) and cycle time is about 0.3~s
  
  Fig \ref{fig:freq_1943_single_fit_300_micros} display the example of fitted FID signal(Fig \ref{fig:freq_1943_single_fit_300_micros}a), measured magnetic field over time (Fig \ref{fig:freq_1943_single_fit_300_micros}b), phase as a function of time (Fig \ref{fig:freq_1943_single_fit_300_micros}c) and magnetic field vs. phase (Fig \ref{fig:freq_1943_single_fit_300_micros}d) for lock-in reference frequency 1943 Hz. During this measurement the lock-in time constant was set to 300$\mu$~s. An  indication of correlation between field and phase can be seen here.
  
   Fig \ref{fig:freq_1943_simultaneous_fit_300_micro_sec} display the example of simultaneous fitted x and y signal( Fig \ref{fig:freq_1943_simultaneous_fit_300_micro_sec}a), measured magnetic field over time ( Fig \ref{fig:freq_1943_simultaneous_fit_300_micro_sec}b), magnetic field vs. phase ( Fig \ref{fig:freq_1943_simultaneous_fit_300_micro_sec}c) and histogram of measured field ( Fig \ref{fig:freq_1943_simultaneous_fit_300_micro_sec}d) for lock-in reference frequency 1943 Hz. During this measurement the lock-in time constant was set to 300$\mu$~s. 
   
    Fig \ref{fig:freq_2015_simultaneous_fit_300_micros} display the example of simultaneous fitted x and y signal(Fig\ref{fig:freq_2015_simultaneous_fit_300_micros} a), measured magnetic field over time (Fig\ref{fig:freq_2015_simultaneous_fit_300_micros} b), magnetic field vs. phase (Fig\ref{fig:freq_2015_simultaneous_fit_300_micros}c) and histogram of measured field (Fig\ref{fig:freq_2015_simultaneous_fit_300_micros} d) for lock-in reference frequency 2015 Hz. During this measurement the lock-in time constant was set to 300$\mu$~s.
    
    Fig \ref{fig:freq_1943_simultaneous_fit_1ms} display the example of simultaneous fitted x and y signal(Fig \ref{fig:freq_1943_simultaneous_fit_1ms} a), measured magnetic field over time (Fig \ref{fig:freq_1943_simultaneous_fit_1ms} b), magnetic field vs. phase (Fig \ref{fig:freq_1943_simultaneous_fit_1ms} c) and histogram of measured field (Fig \ref{fig:freq_1943_simultaneous_fit_1ms} d) for lock-in reference frequency 1943 Hz. During this measurement the lock-in time constant was set to 1~ms. 
    
  Fig \ref{fig:freq_2015_1ms} display the magnetic field as a function of phase while only x data was fitted (Fig \ref{fig:freq_2015_1ms} a) , an example of simultaneous fitted x and y signal(Fig \ref{fig:freq_2015_1ms} b), magnetic field vs. phase (Fig \ref{fig:freq_2015_1ms} c) and histogram of measured field (Fig \ref{fig:freq_2015_1ms} d) for lock-in reference frequency 2015 Hz. During this measurement the lock-in time constant was set to 1~ms.  
  
  In Fig \ref{fig:field for different lockin ref freq} the measured magnetic field over 35 s for different lock-in reference frequency has shown. It is obvious from the plot that if we set reference frequency very close to resonance frequency magnetic field start to oscillate. The exact reason behind this observed field oscillation remains unknown. We are thinking that when lock-in reference frequency is set very close to resonance frequency the fit function might fail to fit data properly due to the less zero crossings. So it seems like a systematic effect on field measurement rather than a real fact.
   
\begin{figure}[h]
\centering\includegraphics[width=0.8\linewidth]{figures/reference_frequency}
\caption{Measured magnetic field for different lock-in reference frequency. The red curve represents measured magnetic field when reference frequency was set to 2015 Hz which is 20 Hz far from resonance frequency. The blue curve shows magnetic field for lock-in reference frequency 1943.9 Hz. In this measurement only x data was fitted. \label{fig:field for different lockin ref freq}}
\end{figure}
   

%\begin{figure}[h]
%\centering\includegraphics[width=0.8\linewidth]{figures/sigma_diff_lock-in_frequency.png}
%\caption{Histogram of the measured magnetic field for different lock-in reference frequency.\label{histogram-of-diff-lock-in freq}}
%\end{figure}

 In this study the systematic effect of changing lock-in amplifier time constant and reference frequency  on magnetic field measurement has been discussed.
 
  
  
  


\section{Study magnetometer performance with tilted field} 
\label{sec:tilted-results}

 %\begin{figure}
    %\centering
    %\begin{subfigure}[b]{0.45\textwidth}
       % \centering
       % %\includegraphics[widt%h=\textwidth,trim={0cm 0.5cm 0cm 0cm},clip]{figures/tilt1.png}
        %\caption{}
        %\label{fig:tilt_0_degree}
    %\end{subfigure}
   % \caption{Optical rotation as a function of time at $\Omega_L$ in the yz plane for different tilt angle.\label{fig:optical-rotation-different-angle}}
%\end{figure}

\begin{figure}
  \centering
%trim={<left> <lower> <right> <upper>}
\includegraphics[width=0.8\textwidth]{figures/tilted_field_scope_trace.png}
  \caption{Optical rotation as a function of time at $\Omega_L$ in the yz plane for different tilt angle.   \label{fig:optical-rotation-different-angle}}
    
\end{figure}
In this Rb NMOR magnetometry setup the rubidium atoms interacted with a $z$-directed laser light beam which is linearly polarized
along the y axis. In the FID NMOR technique, the magnetic field is generally directed along the light propagation direction and the resonance occurs at $2\Omega_L$.  This effect can be explained by considering that the polarization returns to its original state after a $180\degree$ rotation because of the two-fold symmetry of the optically pumped state. As a result, the optical rotation induced by the rotating linear dichroism is periodic at twice the Larmor frequency.

We have observed that Resonances in nonlinear magneto-optical rotation with amplitude modulated light by tilting the magnetic field at angles away from the direction of light propagation while operating the Rb magnetometer in FID mode. When the field is tilted in the plane perpendicular to the light polarization direction an resonance appears at $2\Omega_L$. In this case no additional resonance appears for modulation frequency $\Omega_L$. The amplitude of the FID NMOR signal decreases with increasing tilt angle. 

However, We also observed that by tilting the field direction toward the light polarization direction a new resonance occurs at $\Omega_L$ along with the main resonance at $2\Omega_L$. The resonance signal recorded at $\Omega_L$ contains two frequency components.
\begin{figure}[h]
\centering\includegraphics[width=0.7\linewidth]{figures/fft_amp.png}
\caption{FFT of FID  signal in the presence of transverse field at modulation frequency $\Omega_L$ while tilted the magnetic field direction toward light polarization direction  .\label{fig:fft-amplitude}}
\end{figure}


When direction of the magnetic is tilted toward the light polarization axis  the FID signal contain two frequency component at modulation frequency $\Omega_L$. Fig: \ref{fig:optical-rotation-different-angle} shows the FID signal for different tilt angle when pump beam is modulated at $\Omega_L$. In order to extract frequency components we have used two methods. One of them is the FFT of FID signal and the other one is the fit FID signal using equation \ref{eq:two_sinewave}.

\begin{equation}
     X = A_1 e^{-t /\tau_1} \sin(\omega_1 t + \phi_1) + A_2 e^{-t /\tau_2}  \sin (\omega_2 t + \phi_2) + C  
     \label{eq:two_sinewave}
\end{equation}

Where $A_1$ and $A_2$ are amplitude for two decaying sin wave, $\omega_1$ and $\omega_2$ represents oscillation frequency, $\phi_1$ and $\phi_2$ indicate phase and $\tau_1$ and $\tau_2$ represents coherence time. A 10th order Infinite Impulse Response( IIR) Butterworth filter is also used to reduce background noise from the signal.
Fig: \ref{fig:fft-amplitude} display the FFT of resonance signal for three different  tilt angle $0\degree$, $25\degree$ and $50\degree$. It can be seen from the plot that at $0\degree$ tilt angle there is only one frequency component while for $25\degree$ tilt angle two peak corresponds to two frequency components.

Fig.~\ref{fig:tilted-wrong} shows the amplitude of the FID signal for the magnetic field tilted in the $xz$ plane at various angles to the light propagation direction at modulation frequency $2\Omega_L$ and $\Omega_L$. It can be seen from the figure that the amplitude of the FID signal decreases with increasing tilt angle for both modulation frequency.
\begin{figure}
    \centering
   \begin{subfigure}[b]{0.45\textwidth}
        \centering
        \includegraphics[width=\textwidth]{figures/tilt_x_larmor.png}
        \caption{}
        \label{fig:y equals x}
    \end{subfigure}
    \hfill
     \begin{subfigure}[b]{0.45\textwidth}
        \centering
        \includegraphics[width=\textwidth]{figures/tilt_x_2larmor.png}
        \caption{}
        \label{fig:three sin x}
    \end{subfigure}
    \caption{ The amplitude of the FID NMOR signals as a function
      of tilt angle recorded at $\Omega_L$ and $2\Omega_L$ vs. the
      tilt angle of the magnetic field in the plane defined by the
      light-polarization and light propagation vectors (xz-plane). \label{fig:tilted-wrong}}
\end{figure}

\begin{figure}
    \centering
   \begin{subfigure}[b]{0.45\textwidth}
        \includegraphics[width=\textwidth]{figures/tilt_y_larmor.png}
        \caption{}
        \label{fig:tilt_y}
    \end{subfigure}
    \hfill
     \begin{subfigure}[b]{0.45\textwidth}
        \centering
        \includegraphics[width=\textwidth]{figures/tilt_y_2larmor.png}
        \caption{}
        \label{fig:tilt_x}
    \end{subfigure}
    \caption{(a) The amplitude of the FID NMOR signals as a function
      of tilt angle recorded at $\Omega_L$ and $2\Omega_L$ vs. the
      tilt angle of the magnetic field in the plane defined by the
      light-polarization and light propagation vectors(yz-plane). \label{fig:something-tilted}}
\end{figure}

Fig.~\ref{fig:something-tilted}(a) indicates the signal amplitude vs. different tilt angle at $\Omega_m=2\Omega_L$ . The amplitude of resonance signal at $2\Omega_L$  decreases as the angle between magnetic field B and the light propagation direction increases while the amplitude of new resonance  increases with increasing tilt angle at $\Omega_L$.
      
Fig.~\ref{fig:something-tilted}(b) shows the resonance amplitude at $\Omega_m=2\Omega_L$ keep decreases with increasing tilt angle in $yz$ plane while  the resonance amplitude measured at $\Omega_m=\Omega_L$ keep increase till $30\degree$ and after that amplitude start to decrease and reaches zero when the magnetic field is directed along the y axis. For both cases (Fig: \ref{fig:tilted-wrong} and Fig:\ref{fig:something-tilted}) the signal amplitude has been extracted as a fit parameter when data was fitted to the fit function \ref{eq:two_sinewave}
\begin{figure}[h]
\centering\includegraphics[width=0.9\linewidth]{figures/fitted_data_tilted_field.png}
\caption{ Fitted FID signal using Eq. \ref{eq:two_sinewave}. (b)  The zoomed version of (a). The  blue line indicates raw data while the red one indicates fitted curve}
\end{figure}
\begin{figure}[h]
\centering\includegraphics[width=0.9\linewidth]{figures/filtered_data.png}
\caption{(a)Raw and filtered FID signal. (b) zoomed version of (a)}
\end{figure}

Consider a two-level system $F=1\rightarrow F=0$ and the quantization vector is directed along the magnetic field. When the magnetic field is tilted in the yz plane, the light-polarization axis is perpendicular to the magnetic field. In this case the linearly polarized light containing two circularly polarized components can only create the coherence state between magnetic sublevels $m_F=\pm 1$ . Since the transition frequency between these two consecutive Zeeman energy sublevels is $2\Omega_L$  the resonance appears only at this frequency. 

%However, when the magnetic field is tilted  in the xy plane, the light is a linear superposition of polarizations parallel and perpendicular to the magnetic field. In this case, the light can create coherences between sublevels with $m_F=1$ and $m_F=2$, so resonances are observed at both $ \Omega_L$ and 2$\Omega_L$. 
 


% Conclusion
\chapter{Conclusion\label{ch:conclusion}}
\section{Summary of Results of the Thesis}


The study based on the adjustment of the pump and probe time discussed in section \ref{sec:optimization} allowed us to measure the field faster. Previously  each measurement cycle was about 1 s but now the optimized cycle time which include pump and probe time is about 0.3.5 s which allowed us to take the field measurement about 3 times faster.
  

In order to find out any correlation between coil current drift and magnetic field drift the stability of coil current further should be studied very carefully again now
  that degaussing system seems to have improved.  Fluctuations should
  be easily observable.  Improvements to data acquisition system are easy and should be pursued.  Larger changes in coil current were
  easily observable and seemed to agree with expectation based on the
  measured coil constant.
  
  
It can be concluded for the study discussed in \ref{sec:temperature}  that  magnetometer drift compared with drifts of temperature is $\lesssim 5$~pT/\degree{C}.

Studies regarding degaussing (\ref{sec:degaussing}), which in part tell the story of our
  degaussing development and learning.  
  
From section \ref{sec:degauss-near-zerofield} it can be concluded that we set up the degaussing system and tested mainly sample rate (related
to number of oscillations) stated to be important in Thiel et
al.\cite{doi:10.1063/1.2713433}.  We found that if the innermost shield
was already degaussed that additional poor/rapid degauss did not mar
the field quality as badly as we expected.

 The conclusion of the study discussed in section \ref{sec:three-degauss} regarding initial operations at non-zero field, and Ramp field to large values, without degaussing was ba.  After degaussing  the  the direction of the field drift changed but the field was unstable about  at the same level as before degaussing. 
    
      
The conclusion of the another degaussing study  \ref{sec:outermosts-shield-degauss}, in which degaussing additionally
      all three outershield shields in an unoptimized system was
      done can be drawn that the field quality seemed to improve after degaussing. The field drift reduced by a factor of 4.
 
  From the studies discussed in section \ref{sec:tuning} regarding laser locking and tuning, and the requirements on
  tune stability can be found that
    lock point or laser seems to drift over time.  A Drift of
     about 120 pT was observed where statistical error gets worse.  We think this might be due to PBS. In order to understand 
   whether the drift of lock points affects measured
      field another study has done by manual locking. We found that lock drift does not affect measured field
      drift very much, but does affect statistical precision of
      magnetometer.

 Studies attempting to push below 1~pT in an individual FID
  measurement by adjusting pump and probe powers and lock-in amplifier
  settings.  This includes adjustment of the pump and probe powers,
  and lock-in amplifier settings.  This study revealed problems in the
  procedures used to determine the precession frequency at high
  precision and suggests avenues for further study.

  Future work is to finalize these studies in order to further reduce
  the errors
 Finally, I showed my studies which revealed a way to use FID
  mode to measure transverse fields.  Further work is required to push
  to nT-scale transverse fields relevant for nEDM experiments.
%for typ.~unmeasurable
%  gradients that enter $\delta_T$ correction in Hg-n signals in nEDM
%  experiments.

\section{Future work}

FEMM simulation of z-coil 	Homogeneity of z-coil in this shielding configuration is not  measured with endcaps of innermost shield removed.  Coupling to second to innermost shield is also believed to be a problem and the degree of coupling could be calculated in such a simulation.

The homogeneity measurement of z-coil in shield  has been done before basic degaussing system which is slightly different than our present degaussing system. A further study could be done to measure homogeneity of this z-coil with our present degaussing system).


Study magnetometer performance with self-shielded coil inside the shield instead of our present z-coil. The idea  is to study whether decoupling the coil from the shield has any impact on long-term stability.

The data fitting procedure of field measurement shows a sensitivity to the lock-in time constant and reference frequency which has been discussed in section \ref{sec:reference-frequency}.  In the future removing the lock-in amplifier from data  acquisition system of field measurement could be a better idea for avoiding such systematic error. The signal could be collect directly from photodiode and use digital/analogue filter for further data analysis  instead of demodulating the FID signal through lock-in amplifier.          

 For nEDM experiment, the operated magnetic field is 1 $\mu$T and if our magnetometer could sense transverse field in the range of 1nT, it would be a new application of such magnetometry techniques. So another important future study could be done to find out how precisely the transverse field measurement can be done with
  this magnetometry system. 
 



\section{Farther Future}
The work presented in this thesis is based on measuring magnetic field stability and homogeneities which is a part of the TUCAN's future nEDM measurement at TRIUMF. Measuring nEDM within the sensitivity level of $10^{-27}$ e.cm is the goal for this experiment. Study magnetic field stability is very crucial in order to avoid systematic effects (discussed in chapter \ref{ch:intro}) in nEDM experiment. %because the best previous nEDM measurements showed that 






% Appendix
\section{Calculation of Allan deviation} 
   Allan variance is used to investigate long term stability.In this case, the Allan deviation was calculated in order to determine the deviation of the average field value for a given integration time as a function of the integration time\cite{doe:website2}. Consider a time series of measurements $y_i$ is acquired at times $t_i$ and  a subset of N data points are present.
 \begin{equation}
 \bar{y}_{n}= \sum_{i=(n-1)N+1}^{nN} \frac{y_i}{N} 
 \end{equation}
\begin{figure}[h]
\centering\includegraphics[width=0.6\linewidth]{figures/allan_plot}
\caption{Allan deviation vs. averaging time(s)}
\end{figure}
The Allan deviation can be express as
\begin{equation}
\sigma_y(\tau)=\sqrt{\sigma_y^2(\tau)}=Var_y(\tau)
\end{equation}
Where $\sigma_y$ is the Allan deviation, $\tau$ the time of each frequency estimate and $Var_y$ is the variance of the observed data points.
\begin{equation}
{\sigma_y^2(\tau)}=\frac{1}{2}(\bar{y}_{n+1}-\bar{y}_{n})
\end{equation}
where $\bar{y}_{n}$ is the fractional frequency average over the observation time $\tau$.
Allan deviation formula for sinusoidal waveform:
Suppose
\begin{equation}
y(t) = A\cos (\omega t + \phi)
\end{equation}
The average of this function over a time-interval  is
\begin{equation}
\bar{y}_{n}=\frac{2A}
{\tau \omega}
cos(\omega(t_n +\frac{\tau}{2})+\phi)sin(\frac{\omega \tau}{2})
\end{equation}
and
\begin{equation}
\bar{y}_{n+1}=\frac{2A}
{\tau \omega}
cos(\omega(t_{n+1} +\frac{\tau}{2})+\phi)sin(\frac{\omega \tau}{2})
\end{equation}
Now from equation(5.5) and (5.6) we can write
\begin{equation}
\bar{y}_{n+1}-\bar{y}_{n}=\frac{2A}
{\tau \omega}sin(\frac{\omega \tau}{2})[cos( \omega(t_{n+1} +\frac{\tau}{2})+\phi)-cos(\omega(t_n +\frac{\tau}{2})+\phi)]
\end{equation}
Thus Allan deviation can be written as

\begin{equation}
{\sigma_y^2(\tau)}=\frac{1}{2}(\bar{y}_{n+1}-\bar{y}_{n})=(\frac{4A}
{\tau \omega}sin^2(\frac{\omega \tau}{2}))^2 \sin^2     (\frac{\omega(t_{n+1} +t_{n}+\tau)}{2})+\phi)
\end{equation}
For a large number of randomly
distributed tk the average of the sine-squared function is
\begin{equation}
\sin^2    (\frac{\omega(t_{n+1} +t_{n}+\tau)}{2})+\phi)=\frac{1}{2}
\end{equation}
We therefore find that
\begin{equation}
{\sigma_y^2(\tau)}=(\frac{2A}
{\tau \omega}sin^2(\frac{\omega \tau}{2}))^2
\end{equation}

In FID mode, the resonance signal looks like a decaying sine wave. By fitting the NMOR signal with a damped sine wave the oscillation frequency has been extracted and finally this oscillation frequency has been translated to field.The Allan deviation of the measure field has been plotted in Figure 5.23.

  \section{study the effect of room temperature in magnetic field} 
  \begin{figure}[h]
\centering\includegraphics[width=0.8\linewidth]{figures/temp_.png}
\caption{Temperature measurement\label{temperature}}
\end{figure}
  \begin{figure}[h]
\centering\includegraphics[width=0.8\linewidth]{figures/field_.png}
\caption{Field measurement\label{field}}
\end{figure}
 \begin{figure}[h]
\centering\includegraphics[width=0.8\linewidth]{figures/field_vs_temp.png}
\caption{Field vs. temperature\label{field_vs_temp}}
\end{figure}
\newpage
\section{Study magnetometer performance with tilted field} 
\label{sec:tilted-results}
 \begin{figure}
    \centering
    \begin{subfigure}[b]{0.45\textwidth}
        \centering
        \includegraphics[width=\textwidth]{figures/tilt1.png}
        \caption{}
        \label{fig:tilt_0_degree}
    \end{subfigure}
    \hfill
    \begin{subfigure}[b]{0.45\textwidth}
        \centering
        \includegraphics[width=\textwidth]{figures/tilt2.png}
        \caption{}
        \label{fig:tilt_40_degree}
    \end{subfigure}
    \begin{subfigure}[b]{0.45\textwidth}
        \centering
        \includegraphics[width=\textwidth]{figures/tilt3.png}
        \caption{}
        \label{fig:tilt_70_degree}
    \end{subfigure}
     \hfill
    \begin{subfigure}[b]{0.45\textwidth}
        \centering
        \includegraphics[width=\textwidth]{figures/tilt4.png}
        \caption{}
        \label{fig:tilt_80_degree}
    \end{subfigure}
    \caption{Optical rotation as a function of time at $\Omega_L$ in the yz plane for different tilt angle.\label{fig:optical-rotation-different-angle}}
\end{figure}

\newpage
\printbibliography
\end{document}


