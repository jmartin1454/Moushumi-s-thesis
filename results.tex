\chapter{Experiments using the magnetometer\label{ch:results}}


\begin{itemize}
\item Measurements of magnetic fields over long timescales.
  Conclusion: fields drift over time.  Question: how much is due to
  magnetometer drift vs. other sources?
\item Improving statistical precision I: cycle time adjustment.
  Allowed us to measure faster.
\item Stability of coil current and room temperature.  Conclusion:
  neither drift seems particularly correlated with field drift over
  long time periods.  Temperature stability seems quite good,
  consistent with Michi's thesis.  Side conclusion: when field {\bf
    only} is changed, the magnetometer responds as expected.  Implies
  that magnetometer works well at sensing small changes in field, at
  least on shorter timescales.
\item Studies of degaussing, with multiple goals, telling the story of
  our degaussing
  \begin{itemize}
    \item Degaussing setup and testing at zero field: we set up the
      system, we tested mainly sample rate (related to number of
      oscillations) stated to be important in Thiel et al.  We found
      that if the innermost shield was already degaussed that
      additional poor/rapid degauss did not screw it up as badly as we
      expected, until degaussing was very rapid.
    \item Non-zero field.  Ramp field to large values, without
      degaussing was bad.  After degaussing was better.
    \item Degaussing next to innermost shield was important.
  \end{itemize}
\item Laser locking and tuning, again multiple goals:
  \begin{itemize}
    \item Lock point or laser seems to drift over time.  See ``Drift
      is about 120 pT'' where statistical error gets worse.  Also
      would lose lock sometimes.  We think this may be due to PBS.
    \item Other concern is whether drift of lock affects measured
      field.  Studied by ``manual locking'' and found not to be
      important.  Lock drift does not affect measured field drift very
      much, but does affect statistical precision of magnetometer.
  \end{itemize}
\item Pushing below 1~pT in an individual measurement.  Statistical
  precision can be improved by tuning pump and probe powers.  But when
  we improved the statistical precision significantly, we began to run
  into systematic errors in frequency measurement.  This led us to
  study additional errors related to lock-in amplifier settings.
  Future work is to finalize these studies in order to further reduce
  the errors
\item Finally, transverse fields.  We show our studies which reveal a
  way to use FID mode to measure transverse fields.  Further work is
  required to push to nT-scale transverse fields relevant for
  typ.~unmeasurable gradients that enter $\delta_T$ correction in Hg-n
  signals in nEDM experiments.
\end{itemize}


% Belongs in Magnetometer Literature Review Chapter 2?

%. In the case of AM NMOR, high-field
%resonance occur in addition to the regular zero-field resonance. The
%optical properties of the medium are being modulated at twice the
%Larmor frequency. In the case of strong external magnetic field, the
%dynamic Stark effect limits the sensitivity of NMOR based atomic
%magnetometry by reducing the accuracy of the field measurement. The
%advantage of using AM NMOR method is that it can reduce the Stark
%effect because light frequency is not affected by amplitude
%modulation\cite{gawlikoptical}. In AM NMOR, It is easily possible to
%control the number and amplitudes of the high field resonances by
%using the square wave modulation of light intensity.




  \section{long term FID measurement}  
The forced oscillation scans and the single FID shot gives information about the present magnetic field. In order to get information about the change in magnetic field over time, long term data was taken in the FID mode. During this long term process the laser frequency was locked into D1 transition line of Rb-85 using DigiLock 110 laser locking module. A stable power supply was used to run the coil system inside the shield. In order to observe the FID signal, a Tektronix DPO 2014 oscilloscope was connected to X and Y output of output of lock-in amplifier. A python script is used to setup function generator and to trigger DAQ for long term FID measurement. This same python script is also used to grab data continuously from oscilloscope screen to computer. For further data analysis  another python script is used. On the basis of the X
and Y channel recordings, a least-square fit was done for each X and Y pair. The measured oscillation frequency of decaying signal was then convert to magnetic field using equation \ref{eq:field}. The recorded B-field data vs. time was further used to calculate Allan deviations, getting  information on the deviations from the mean magnetic field vs. integration time (bin size). Measuring magnetic field over long period of time gives us information about the magnetometer performance. Toward this end,we study long term field measurement. Fig~\ref{fig:long term field} ~represents the magnetic field recorder over 4 hours on three different day. Each data points in this graph correspond to single FID scan. The observed field drift is almost same (15 pT) for three days. This measurement was conducted at 0.2 $\mu$T magnetic field.
\begin{figure}[h]
\centering\includegraphics[width=0.85\linewidth]{figures/field_3_day}
\caption{Magnetic field recorded over 4 hours on three different day. The observed field drift is almost same (15 pT) for three days.\label{fig:long term field}}
\end{figure}
   \begin{itemize}
   \item summary of statistical error, systematic error, stability via allan deviations.
   \end{itemize}
  
   \section{Optimization of cycle time} 
   \begin{figure}
    \centering
 
    \begin{subfigure}[b]{0.425\textwidth}
        \centering
        \includegraphics[width=\textwidth]{figures/Capture}
        \caption{}
        \label{fig:pump short}
    \end{subfigure}
    \hfill
    \begin{subfigure}[b]{0.42\textwidth}
        \centering
        \includegraphics[width=\textwidth]{figures/Capture2}
        \caption{}
        \label{fig:pump long}
    \end{subfigure}
    \caption{(a) FID signal for pump time~ 0.49 s and probe time~ 0.4 s. (b) FID signal for pump time  0.1 s and probe time ~0.2 s. Both measurement were conducted at $0.2~\mu$T magnetic field.}
    \label{fig:pump time}
\end{figure} 
   \begin{figure}[h]
\centering\includegraphics[width=0.75\linewidth]{figures/pump_time}
\caption{Measured B-field vs. time   for different pump time. The longest pump time is 0.49 s while the shortest pump time is 0.1 s. No obvious dependency of pump time on  measured magnetic field has been observed during this study.\label{fig:different pump time}}
\end{figure}
In FID NMOR optical pumping is done for polarizing Rb atoms and afterward the spontaneous decay of excited atom is observed. So a complete cycle of FID measurement consist of pump and probe time.Pump time represents the time atoms take to generate a polarized ground state. In this study we were trying to study how long the optical pumping of Rb atom should have continued in order to generate an alignment and optical pumping for long time does make any difference in measuring magnetic field precisely or not. FID signal with 0.49 sec pump time and 0.4 sec probe time has shown in Fig \ref{fig:pump short}. On the other hand, Fig \ref{fig:pump long} shows  the FID signal for 0.1 s pump time and 0.2 s probe time. In Fig \ref{fig:different pump time} the histogram of measured magnetic field over 300 s has shown for different pump time. The longer pump time is 0.49 s   and the shorter one is 0.1 sec. It is obvious from the plot that pump time doesn't effect in field precision. However for our Rb magnetometry setup it is not possible to make the pump time shorter than 0.1 s because this 0.1 s is the minimum time atoms need to get aligned for making a coherence state after interacting with laser beam. Since the amplitude of FID NMOR signal reach it's maximum while atoms are in coherence state. Thus It is easy to understand coherence state is ready or not by observing the signal amplitude. In Fig~\ref{fig:pump short} the optical pumping is done for 0.49 s while signal amplitude reach its maximum over 0.1 s and after that signal amplitude remains unchanged till 0.49 s. In this case there is no point to pump more than 0.1 s. Same situation happens about probe time. If signal amplitude decays so quickly there is no point to set longer probe time. By optimizing pump and probe time we can speed up data acquisition system. After optimization for a single FID scan it only takes 0.35 s  without losing any information. By using this faster data acquisition system it is possible to record multiple FID run with in a short period of time which is helpful to gather more information about magnetic field environment.
   \section{current with existing power supply}
   \begin{itemize}
   \item Analysis of current stability.\\
 An Agilent B2962A power supply was used to run the coil system which produce magnetic field inside the shield. In order to develop a highly sensitive magnetometer, it is very important to make use of a highly stable power supply. The idea was to determine the current drifts of the power supplies, mainly based on the fact that the given current supply actually acts as a voltage supply with the output voltage transformed into an output current by a resistor which is sensitive to external influences such as temperature. Fig~\ref{fig:current} shows the current stability of power supply over 10000 sec. During this time the current only change by 30 nA.  
   \begin{figure}[h]
\centering
\includegraphics[width=0.8\linewidth]{figures/current}
\caption{Recorded current over 10000 s using Agilent B2962A power source. During this measurement coil current only changed by 30 nA.\label{fig:current} }
\end{figure}
   \item Current stability is better than typical measured field changes.
    \begin{figure}[h]
\centering
\includegraphics[width=0.8\linewidth]{figures/field_current_study.png}
\caption{Magnetic field vs. time. During this measurement coil current only changed by 30 nA.\label{fig:field_current} }
\end{figure}

 \begin{figure}
    \centering
 
    \begin{subfigure}[b]{0.45\textwidth}
        \centering
        \includegraphics[width=\textwidth]{figures/field_coil_current.png}
        \caption{}
        \label{fig:field_measure_and_produced}
    \end{subfigure}
    \begin{subfigure}[b]{0.45\textwidth}
        \centering
        \includegraphics[width=\textwidth]{figures/field_current_allan_plot.png}
        \caption{}
        \label{fig:allan_plot}
    \end{subfigure}
    \caption{(a)Measured and produced magnetic field vs. time. (b) Allan deviation of field.}
    \label{fig:field_allan_deviation}
\end{figure} 
   \item Change in magnetic field due to the change in coil current
   
 A study has been conducted to determine the field change by changing the coil current by a known amount. The main objective of this study is to check the Rb magnetometer performance on magnetic field measurement. During this measurement coil current has been changed by $\pm$ 0.001 mA. As a result a field change of $\pm$ 50  $\mu$T  has been observed. The laser frequency was locked to Rb transition frequency and all other setting kept unchanged. During this study the magnetometer has been operated in FID mode. Fig~\ref{fig:field change} presents the magnetic field change over time for different coil current. For better understanding Fig~\ref{fig:field change} has been divide into five different regions. In region-1 magnetic field  has been recorded for 300 s while the applied current to the Z coil was 4.5 mA. Then in region-2 the applied current to the z coil has been increased by 0.001 mA i.e., total current 4.501 mA and measured the magnetic field for another 300 sec. It is clear from the figure that the field has been changed instantly by 50 $\mu$T due to the changed coil current. Now in region-3, the coil current has been reduced to 4.5 mA and measured field for 300 s. After that the coil current has been decreased by 0.001 mA. As a result the field also decreased by 50~$\mu$T in this region.Finally in region-5 the coil current has been set to 4.5 mA again and observed the corresponing quick change in field.

   \begin{figure}[h]
\centering\includegraphics[width=0.7\linewidth]{figures/field_change_with_current}
  
\caption{The change in magnetic field  by changing coil current has been studied over 1400 sec. The blue line on different regions of the graph displays the measured magnetic field corresponding to the change in coil current. \label{fig:field change}}
\end{figure}
   \end{itemize}
   \section{degaussing studies\label{sec:degaussing}}  
   \begin{itemize}
   \item different degaussing schemes(changing sample rate)
   
 
\begin{table}%[h]
\centering
\begin{tabular}{|l|l|}
\hline
\textbf{ SETTING}    & \textbf{ VALUE} \\
\hline
Function generator &   \\
\hline
Frequency &  10 Hz   \\
Sample rate    &  10000 sample/sec  \\
Amplitude   &   10 V \\
Offset  &       0 V  \\
\hline
\end{tabular}
\caption{Setting for degaussing system \label{table:degaussing setting}}
\end{table}
  
Degaussing process is done in order to avoid the environmental perturbation and reduce any remnant field inside the four layer $\mu$ metal magnetic shield \cite{doi:10.1063/1.2713433}. The setting for degaussing has shown in Table \ref{table:degaussing setting}. For our degaussing system we are using an envelope function which has $1\times 10^6$ points. Degaussing is complete after $5\times 10^6$ points. Sample rate affects how quickly we move through this waveform. A detailed analysis of NMOR signal by changing the degaussing parameter could give us some information about the goodness of degaussing procedure. Toward this end, a study has been carried out to determine  the dependency of the resonance width on the sample rate. 

In this study, data was acquired by sweeping the magnetic field near zero field. Before each measurement degaussing the innermost layer of shield has done. During this measurement only one degaussing parameter,sample rate, was varied in order to study the effect of degaussing parameter on magnetic field measurement.  Optical rotation as a function of magnetic field for different sample rate has shown in Fig~\ref{fig:different sample rate}. The resonance width $\Delta B$, the difference between two peak of the dispersive curve, changed for different sampling rate. It can be seen from Fig~\ref{fig:different sample rate} that, the resonance width is about 0.49 nT for sample rate 10000 sample/s and the resonance width is 0.38 nT for sample rate 80000 sample/s. Fig~\ref{fig:different sample rate} shows the resonance width as a function of sample rate. It is obvious from the graph that the resonance width becomes narrower for larger sample rate.    
   
   \begin{figure}[h]
\centering\includegraphics[width=0.65\linewidth]{figures/sample_rate}
\caption{ Optical rotation vs. measured B-field   for different sample rate. The resonance width is narrower(0.38 nT) for higher sample rate(80000 sample/s) whereas the resonance width becomes broaden (0.49 nT) for lower sample rate.\label{fig:different sample rate} }
\end{figure}
\begin{figure}[h]
\centering\includegraphics[width=0.6\linewidth]{figures/field_vs_sample_rate}
\caption{Resonance width vs. sample rate. Resonance width decreases with increasing sample rate.When sample rate is 5000 sample/s the observed resonance width is 0.55 nT. On the other hand the resonance width is 0.38 for sample rate 80000 sample/s. \label{resonance width} }
\end{figure}

   \item Ramp field up/down \\
  During this study the magnetometer has been operated in FID mode at 0.2 $\mu$T field. The recorded magnetic field over 2200 s has been showed in Fig \ref{fig:ramp up}. No degaussing has been done during this measurement. It can be seen from the figure the magnetic field increase linearly for first 300 s then showed a decrease in field. The field was pretty stable between 600 s and 1800 s and after that the field showed a rapid increase. The overall field change is about 15 pT during the measurement. Data was acquired by ramping $B_z$  from  0.2 $\mu$T to 10 $\mu$T then again set it to 0.2 $\mu$T and collect FID signal. By doing this field ramping we intentionally perturb the magnetic field environment inside the shield. After this field ramping the long term field measurement has been conducted for another 2200 s. In this case, the field showed a downward drift of about 35 pT. Thus field ramping has been changed the magnetic field environment inside the magnetic shielding.
   \begin{figure}
    \centering
 
    \begin{subfigure}[b]{0.425\textwidth}
        \centering
        \includegraphics[width=\textwidth]{figures/ramp_1}
        \caption{}
        \label{fig:ramp up}
    \end{subfigure}
    \hfill
    \begin{subfigure}[b]{0.42\textwidth}
        \centering
        \includegraphics[width=\textwidth]{figures/ramp_2}
        \caption{}
        \label{fig:ramp down}
    \end{subfigure}
     \begin{subfigure}[b]{0.42\textwidth}
        \centering
        \includegraphics[width=\textwidth]{figures/ramp3}
        \caption{}
        \label{fig:degauss}
    \end{subfigure}
    \caption{ FID signal at $B_z=0.2 \mu$T magnetic field.(a) no degaussing (b) Ramp $B_z$  from  0.2 $\mu$T to 10 $\mu$T then again set it to 0.2 $\mu$T and collect FID signal(no degaussing). (c) study magnetic field stability after degaussing.}
    \label{fig:ramp updown}
\end{figure}
   
   \item degaussing second innermost shield effect in field drift.
   
   After doing transverse field study (applying current to X and Y coil along with Bz coil) magnetic field environment was pretty unstable. Fig \ref{fig:without DG} shows about 50 pT drift in magnetic field in 4000 s. In order to cancel background field inside shield degaussing innermost layer of $\mu$ metal is done but it doesn't help to solve field drift problem. Then degaussing the 2nd innermost layer of shield is done which solve the field drift problem. Since the end cap of the innermost shield layer is not in use some background field leaked into 2nd layer which was causing the drift in field. So by degaussing the 2nd layer of shield canceled background field and field becomes so stable (Fig \ref{fig:with DG}).
   
  \begin{figure}
    \centering
 
    \begin{subfigure}[b]{0.45\textwidth}
        \centering
        \includegraphics[width=\textwidth]{figures/before_degaussing}
        \caption{}
        \label{fig:without DG}
    \end{subfigure}
    \hfill
    \begin{subfigure}[b]{0.45\textwidth}
        \centering
        \includegraphics[width=\textwidth]{figures/after_degaussing}
        \caption{}
        \label{fig:with DG}
    \end{subfigure}
    \caption{ Magnetic field recorded over 3 hours. (a) The stability of magnetic field without performing any degaussing. A downward field drift of about 45 pT  has observed in this case (b) magnetic field stability after degaussing the 2nd innermost layer of magnetic shielding. Only a couple pT field drift has observed in this case.}
    \label{fig:effect of DG}
\end{figure}
   \end{itemize}
   \newpage
   \section{Laser tuning} 
   \begin{itemize}
   \item effect of careful tuning\\
	The main purpose of this study is to understand the importance of laser tuning on coherence life time of excited atoms. During the study optical pumping is done to polarize the rubidium atom and hence produce coherence state. When the laser light  is  exactly tuned to resonance with the atomic transition, the lifetime of coherence state becomes longer and the amplitude of resonance signal reaches its maximum. On the other hand coherence state decay  quickly when laser frequency detunes from atomic transition. In this case, the signal amplitude reduces due to frequency detuning. Fig~ \ref{fig:effect of tuning} shows the effect of laser tuning on coherence lifetime. For proper laser tuning the observed coherence time is 68 ms (Fig \ref{fig:good tuning}) whereas for bad tuning coherence time reduces to 59 ms (Fig \ref{fig:bad tuning}). On a fundamental level, the magnetometer actually measures the energy splitting between the Zeeman sublevels of the atomic ground state due to the magnetic field. The linewidth of such a spectroscopic measurement is given by the coherence lifetime $T_2$ of the atomic spins:
\begin{equation}
 ΔB = \Delta\omega/\gamma  = 1/γ T_2
\end{equation}

The development of a sensitive magnetometer depends on achieving the maximum possible polarization lifetime.  For very short coherence time atomic spin depolarize quickly which limits the sensitivity of the magnetometer.
Since  longer coherence time and larger signal amplitude indicate better frequency precession and therefore precise field measurement it is very important to make sure the laser tuning has done carefully.
   \begin{figure}
    \centering
 
    \begin{subfigure}[b]{0.45\textwidth}
        \centering
        \includegraphics[width=\textwidth]{figures/perfect_tuning}
        \caption{}
        \label{fig:good tuning}
    \end{subfigure}
    \hfill
    \begin{subfigure}[b]{0.45\textwidth}
        \centering
        \includegraphics[width=\textwidth]{figures/bad_tuning}
        \caption{}
        \label{fig:bad tuning}
    \end{subfigure}
    \caption{(a) Recorded FID signal while laser is perfectly tuned to atomic transition frequency. (b) FID resonance signal when laser is slightly detuned from transition frequency.}
    \label{fig:effect of tuning}
\end{figure}
   \item problem with drift of tune
   
   For studying long term stability data was taken for about 12 hours. Laser was locked to D1 transition line of Rb using Digilock laser locking software. Fig \ref{fig:effect of tuning} shows 120 pT observed drift in magnetic field over 12 hours. Although DAVLL was in use laser frequency was not locked for 12 hours. As a result laser tuning moved due to mode hoping which causes the drift. The current laser locking system only works perfectly for maximum 4 hours. The possible reason for this mode hoping is the Polarizing beam splitting cube of our DAVLL system \cite{principles}. The disadvantage of using this type of PBS is that they show flaky optical behavior over longer times.
   \begin{figure}[h]
\centering\includegraphics[width=0.5\linewidth]{figures/field_drift}
\caption{Magnetic field recorded over 12 hours.In this measurement the observed field drift is about 120 pT.\label{digilock field drift}}
\end{figure}
\item manual tuning\\
   During this study frequency of laser light is tuned to atomic transition maximize optical rotation.The laser tuning was adjusted manually time to time for maintaining same signal amplitude during measurement. The main objective of this study is to check the observed drift in field measurement is the real field drift or the magnetometer drift due to the mode hoping. Instability of laser locking system due to the poor performance of PBS which is used in DAVLL system is the probable reason behind this mode-hoping. According to Fig \ref{fig:amplitude manual tuning} we can say that during this measurement the amplitude of FID NMOR was pretty stable except small fluctuations over 20000 s . The observed small fluctuation is due to the manual adjustment while keeping the laser frequency tuned to atomic transition over long period of time. Obtaining stable signal amplitude is a indication that laser is not drifting to much. Although laser frequency was not drifting a lot during  the measurement, still there is a  drift in magnetic field (Fig \ref{fig:field manual tuning}). The overall drift is about 20 pT over 20000 s. So it can be conclude that the drift in laser frequency is not the main reason behind this induced field drift. 
   
    
   \end{itemize}
   \begin{figure}
    \centering
 
    \begin{subfigure}[b]{0.45\textwidth}
        \centering
        \includegraphics[width=\textwidth]{figures/manual_tuning}
        \caption{}
        \label{fig:field manual tuning}
    \end{subfigure}
    \hfill
    \begin{subfigure}[b]{0.45\textwidth}
        \centering
        \includegraphics[width=\textwidth]{figures/amplitude_manual_tuning}
        \caption{}
        \label{fig:amplitude manual tuning}
    \end{subfigure}
    \caption{(a) magnetic field vs. time. (b) amplitude of recorded FID NMOR signal over 20000 sec. During this study laser tuning is maintained manually}
    \label{fig:manual tuning}
\end{figure}
\section{Systematic error check in frequency measurement in FID mode \label{sec:reference-frequency}} 
   \subsection{Optimization of pump and probe beam power} 
 A study has been performed to determine how optimization of the pump and probe beam power effect on precise field measurement.  During this study, the magnetometer has operated in FID mode.  A  linearly polarized probe beam has been used to analyze the spin response. In Fig \ref{fig:different probe power} the measured magnetic field has been displayed for two different  Probe beam power. Fig \ref{fig:power less} present the measured field over 100 s with 15 $\mu$W probe power while fig \ref{fig:power double} display the measured field over 100 s for probe power 30 $\mu$W. In both cases, the pump beam power has been kept unchanged ($5\%$ duty cycle). Each blue points in this graph represent a single FID scan. As can be seen from Fig \ref{fig:different probe power} the magnetic field is more scattered for high beam power. Precession width is about $7.2$ pT for probe beam power 30 $\mu$W while for low beam power(15 $\mu$W) the precession width is about $3.01$ pT. It can be concluded lower probe beam power is better for precession  magnetometry because the narrower the precession width more sensitive the magnetometer. Although low probe beam power is better for precise field measurement, it's hard to work with because of the dimness of light.
  \begin{figure}
    \centering
 
    \begin{subfigure}[b]{0.7\textwidth}
        \centering
        \includegraphics[width=\textwidth]{figures/beam_power_less}
        \caption{}
        \label{fig:power less}
    \end{subfigure}

    \begin{subfigure}[b]{0.7\textwidth}
        \centering
        \includegraphics[width=\textwidth]{figures/beam_power_double}
        \caption{}
        \label{fig:power double}
    \end{subfigure}
    \caption{ Magnetic field as a function of time for different power of probe beam.(a) measured magnetic field with probe beam power 15 $\mu$w. Precession width is about $3.01$ pT (b) measured magnetic field with precession width $7.2$ pT for probe beam power 30 $\mu$w.}
    \label{fig:different probe power}
\end{figure}
%data colllected 13th August 2018

Another study has been conducted to understand the influence of pump beam power on precession magnetometry. During this study, the probe beam power ($\sim$ 15 $ \mu$W) has kept unchanged while pump power has changed. In this measurement data has been acquired by running the magnetometer in FID mode. In FID mode amplitude modulation of pump beam has been done by using a AOM.  So the pump beam power can be change by changing the duty cycle of square wave modulation. When duty cycle is set to $5\%$ the pump beam power is 40 $\mu$W and for $16 \%$ duty cycle power is 128 $\mu$W. During this measurement the lock-in reference frequency was set to 1943.9 Hz while the AOM frequency was 2036 Hz. The histogram of measured magnetic field for different pump beam power has been showed in fig \ref{fig:different pump power}. The precession width (sigma) become larger for higher duty cycle while it becomes narrower for small duty cycle. In the case of 5$\%$  duty cycle precession width is about $2.3$ pT  (fig \ref{fig:different pump power}(a)) and for 16 $\%$ duty cycle the spread is about 4.6 pT (fig \ref{fig:different pump power}(b)). It is clear from the plot that, at higher duty cycle data points are not statistically distributed.  When a pump beam with a higher duty cycle is used  in precession magnetometry, the NMOR  signal amplitude increase accordingly. But it has been observed that error in frequency measurement also increases with higher signal amplitude which was not expected. A strong correlation between magnetic field and phase of the NMOR signal also has been observed for higher pump power. So it can be concluded that using higher duty cycle could induce more systematic errors in field measurement which led to the next study.
 \begin{figure}
    \centering  \includegraphics[width=\textwidth]{figures/pump_beam}
    \caption{ Magnetic field vs. time for different pump beam power. (a) measured magnetic field with duty cycle $5 \%$. Precession width is about $2.3$ pT (b) measured magnetic field with Precession width $4.6$ pT for  16 $\%$ duty cycle.}
    \label{fig:different pump power}
\end{figure}

\subsection{How far the reference frequency of lock-in amplifier should set to measure field correctly}
  
  In FID NMOR the atomic sample is polarized once by optical pumping and then observed the spontaneous decay of excited atom while the pump beam is off. In order to capture the FID signal correctly, the reference frequency of lock-in amplifier is normally set to about 100 Hz apart from the resonance frequency. In this study we kept all other setting fixed only changed the Lock-in reference frequency and observed the effect of that on field measurement study. The measurement was conducted at $0.2 \mu$T field. The FID signal for different reference frequency of Lock-in amplifier has shown in Fig \ref{fig:different reference signal}. In the case of Fig \ref{fig:far from resonance} the reference frequency of Lock-in was set to 1943.9 Hz (90 Hz far from resonance frequency). On the other hand for Fig \ref{fig:close to resonance} the reference frequency of Lock-in was set to 2015 Hz (20 Hz far from resonance frequency). In this study the resonance frequency is 2035 Hz. In Fig \ref{fig:field for different lockin ref freq} the measured magnetic field over 35 s for different lock-in reference frequency has shown. It is obvious from the plot that if we set reference frequency very close to resonance frequency magnetic field start to oscillate. The exact reason behind this observed field oscillation remains unknown. We are thinking that when lock-in reference frequency is set very close to resonance frequency the fit function might fail to fit data properly due to the less zero crossings. So it seems like a systematic effect on field measurement rather than a real fact.
 \begin{figure}
    \centering
    \begin{subfigure}[b]{0.4\textwidth}
        \centering
        \includegraphics[width=\textwidth]{figures/reference_frequency1}
        \caption{}
        \label{fig:far from resonance}
    \end{subfigure}
    \hfill
    \begin{subfigure}[b]{0.4\textwidth}
        \centering
        \includegraphics[width=\textwidth]{figures/reference_frequency3}
        \caption{}
        \label{fig: middle range}
    \end{subfigure}
    \begin{subfigure}[b]{0.4\textwidth}
        \centering
        \includegraphics[width=\textwidth]{figures/reference_frequency2}
        \caption{}
        \label{fig:close to resonance}
    \end{subfigure}
    \caption{FID signal for different reference frequency of Lock-in amplifier while the resonance frequency was 2035 kHz.(a) reference frequency was set to 100 Hz far from resonance (b) difference between resonance and reference frequency is 40 Hz, (c) reference frequency was set to 2015 Hz while resonance frequency 2035 Hz. \label{fig:different reference signal}}
\end{figure}
\begin{figure}[h]
\centering\includegraphics[width=0.8\linewidth]{figures/reference_frequency}
\caption{Measured magnetic field for different lock-in reference frequency. The red curve represents measured magnetic field when reference frequency was set to 2015 Hz which is 20 Hz far from resonance frequency. The blue curve shows magnetic field for lock-in reference frequency 1943.9 Hz.\label{fig:field for different lockin ref freq}}
\end{figure}

\begin{figure}[h]
\centering\includegraphics[width=0.8\linewidth]{figures/phase}
\caption{Measured magnetic field for different lock-in reference frequency.The red curve represents measured magnetic field when reference frequency was set to 2015 Hz and field for lock-in reference frequency 1943.9 Hz.}
\end{figure}
\newpage
   \begin{itemize}
   \item effect of lock-in time constant\\
  A SRS-830 lock-in amplifier is used to grab FID NMOR signal which was connected to the output of balanced photodiode. Table \ref{tab:FID_setting} represents all settings for FID NMOR. In this study the systematic effect of changing locking amplifier time constant on magnetic field measurement has been discussed. Fig \ref{fig:different time constant}  shows the histogram of measured magnetic field for two different time constant of lock-in amplifier while all other settings is same. In the case of Fig \ref{fig:time constant long} the lock-in time constant is 1 ms and Fig \ref{fig:time constant short} shows the histogram of magnetic field for time constant 300 $\mu$s. Here sigma is representing the precession width of field. It can be seen from the figure that the calculated sigma is larger for time constant 1 ms compared to 300 $\mu$s. This measurements was conducted at 0.2 $\mu$T field. The resonance frequency and lock-in reference frequency are 2035 Hz and 1943 Hz respectively.
   \begin{figure}
    \centering
    \begin{subfigure}[b]{0.4\textwidth}
        \centering
        \includegraphics[width=\textwidth]{figures/time_constant}
        \caption{}
        \label{fig:time constant long}
    \end{subfigure}
    \hfill
    \begin{subfigure}[b]{0.4\textwidth}
        \centering
        \includegraphics[width=\textwidth]{figures/time_constant_300micro_sec}
        \caption{}
        \label{fig:time constant short}
    \end{subfigure}
    \caption{Histogram of measured magnetic field for different time constant of lock-in amplifier. (a) lock-in time constant was set to 1 ms (b) for lock-in time constant 300 $\mu$s. \label{fig:different time constant} }
\end{figure}
    \item timing drift in Tektronix oscilloscope clock
   \end{itemize}
   
 \section{Study the effect of room temperature in magnetic field} 
 
 Room temperature is measured using precision thermometer and laser temperature is measured from the output of the laser temperature controller panel. The temperature on the magnetic shielding is measured by a T-type Thermocouple. During the measurement the laser temperature was unchanged but room temperature fluctuation is 1\degree
  \begin{figure}[h]
\centering\includegraphics[width=0.8\linewidth]{figures/temp_.png}
\caption{Temperature measurement\label{temperature}}
\end{figure}
  \begin{figure}[h]
\centering\includegraphics[width=0.8\linewidth]{figures/field_.png}
\caption{Field measurement\label{field}}
\end{figure}
 \begin{figure}[h]
\centering\includegraphics[width=0.6\linewidth]{figures/field_vs_temp.png}
\caption{Field vs. temperature\label{field_vs_temp}}
\end{figure}
\newpage
%The magnetometric method based on FID NMOR is a very sensitive
%technique of magnetic field measurements. Those measurements are
%scalar, i.e., the position of a given resonance depends only on the
%magnitude not the direction of the magnetic field. However, the
%relative magnitudes of the FID NMOR resonances could have some
%dependency on the magnetic field direction. Thus there is a
%possibility to get some information about the direction of the
%magnetic field by doing a detailed analysis of the FID NMOR signal.
%For this reason, the dependency of the FID NMOR signal on the
%magnetic field direction has studied here.
\section{Study magnetometer performance with tilted field} 
\label{sec:tilted-results}
 \begin{figure}
    \centering
    \begin{subfigure}[b]{0.45\textwidth}
        \centering
        \includegraphics[width=\textwidth]{figures/tilt1.png}
        \caption{}
        \label{fig:tilt_0_degree}
    \end{subfigure}
    \hfill
    \begin{subfigure}[b]{0.45\textwidth}
        \centering
        \includegraphics[width=\textwidth]{figures/tilt2.png}
        \caption{}
        \label{fig:tilt_40_degree}
    \end{subfigure}
    \begin{subfigure}[b]{0.45\textwidth}
        \centering
        \includegraphics[width=\textwidth]{figures/tilt3.png}
        \caption{}
        \label{fig:tilt_70_degree}
    \end{subfigure}
     \hfill
    \begin{subfigure}[b]{0.45\textwidth}
        \centering
        \includegraphics[width=\textwidth]{figures/tilt4.png}
        \caption{}
        \label{fig:tilt_80_degree}
    \end{subfigure}
    \caption{Optical rotation as a function of time at $\Omega_L$ in the yz plane for different tilt angle.\label{fig:optical-rotation-different-angle}}
\end{figure}
In this Rb NMOR magnetometry setup the rubidium atoms interacted with a $z$-directed laser light beam which is linearly polarized
along the y axis. In the FID NMOR technique, the magnetic field is generally directed along the light propagation direction and the resonance occurs at $2\Omega_L$.  This effect can be explained by considering that the polarization returns to its original state after a $180\degree$ rotation because of the two-fold symmetry of the optically pumped state. As a result, the optical rotation induced by the rotating linear dichroism is periodic at twice the Larmor frequency.

We have observed that Resonances in nonlinear magneto-optical rotation with amplitude modulated light by tilting the magnetic field at angles away from the direction of light propagation while operating the Rb magnetometer in FID mode. When the field is tilted in the plane perpendicular to the light polarization direction an resonance appears at $2\Omega_L$. In this case no additional resonance appears for modulation frequency $\Omega_L$. The amplitude of the FID NMOR signal decreases with increasing tilt angle. 

However, We also observed that by tilting the field direction toward the light polarization direction a new resonance occurs at $\Omega_L$ along with the main resonance at $2\Omega_L$. The resonance signal recorded at $\Omega_L$ contains two frequency components.
\begin{figure}[h]
\centering\includegraphics[width=0.7\linewidth]{figures/fft_amp.png}
\caption{FFT of FID  signal in the presence of transverse field at modulation frequency $\Omega_L$ while tilted the magnetic field direction toward light polarization direction  .\label{fig:fft-amplitude}}
\end{figure}


When direction of the magnetic is tilted toward the light polarization axis  the FID signal contain two frequency component at modulation frequency $\Omega_L$. Fig: \ref{fig:optical-rotation-different-angle} shows the FID signal for different tilt angle when pump beam is modulated at $\Omega_L$. In order to extract frequency components we have used two methods. One of them is the FFT of FID signal and the other one is the fit FID signal using equation \ref{eq:two_sinewave}.

\begin{equation}
     X = A e^{-t /\tau_1} \sin(\omega_1 t + \phi_1) + A_1 e^{-t /\tau_2}  \sin (\omega_2 t + \phi_2) + C  
     \label{eq:two_sinewave}
\end{equation}

Where A and $A_1$ are amplitude for two decaying sin wave, $\omega_1$ and $\omega_2$ represents oscillation frequency, $\phi_1$ and $\phi_2$ indicate phase and $\tau_1$ and $\tau_2$ represents coherence time. A 10th order Infinite Impulse Response( IIR) Butterworth filter is also used to reduce background noise from the signal.
Fig: \ref{fig:fft-amplitude} display the FFT of resonance signal for three different  tilt angle $0\degree$, $25\degree$ and $50\degree$. It can be seen from the plot that at $0\degree$ tilt angle there is only one frequency component while for $25\degree$ tilt angle two peak corresponds to two frequency components.

Fig.~\ref{fig:tilted-wrong} shows the amplitude of the FID signal for the magnetic field tilted in the $xz$ plane at various angles to the light propagation direction at modulation frequency $2\Omega_L$ and $\Omega_L$. It can be seen from the figure that the amplitude of the FID signal decreases with increasing tilt angle for both modulation frequency. 

\begin{figure}
    \centering
   \begin{subfigure}[b]{0.45\textwidth}
        \centering
        \includegraphics[width=\textwidth]{figures/tilt_x_larmor.png}
        \caption{}
        \label{fig:y equals x}
    \end{subfigure}
    \hfill
     \begin{subfigure}[b]{0.45\textwidth}
        \centering
        \includegraphics[width=\textwidth]{figures/tilt_x_2larmor.png}
        \caption{}
        \label{fig:three sin x}
    \end{subfigure}
    \caption{ The amplitude of the FID NMOR signals as a function
      of tilt angle recorded at $\Omega_L$ and $2\Omega_L$ vs. the
      tilt angle of the magnetic field in the plane defined by the
      light-polarization and light propagation vectors(xz-plane). \label{fig:tilted-wrong}}
\end{figure}

\begin{figure}
    \centering
   \begin{subfigure}[b]{0.45\textwidth}
        \centering
        \includegraphics[width=\textwidth]{figures/tilt_y_larmor.png}
        \caption{}
        \label{fig:tilt_y}
    \end{subfigure}
    \hfill
     \begin{subfigure}[b]{0.45\textwidth}
        \centering
        \includegraphics[width=\textwidth]{figures/tilt_y_2larmor.png}
        \caption{}
        \label{fig:tilt_x}
    \end{subfigure}
    \caption{(a) The amplitude of the FID NMOR signals as a function
      of tilt angle recorded at $\Omega_L$ and $2\Omega_L$ vs. the
      tilt angle of the magnetic field in the plane defined by the
      light-polarization and light propagation vectors(yz-plane). \label{fig:something-tilted}}
\end{figure}

Fig.~\ref{fig:something-tilted}(a) indicates the signal amplitude vs. different tilt angle at $\Omega_m=2\Omega_L$ . The amplitude of resonance signal at $2\Omega_L$  decreases as the angle between magnetic field $B$ and the light propagation direction increases while the amplitude of new resonance  increases with increasing tilt angle at $\Omega_L$.
      
Fig.~\ref{fig:something-tilted}(b) shows the resonance amplitude at $\Omega_m=2\Omega_L$ keep decreases with increasing tilt angle in $yz$ plane while  the resonance amplitude measured at $\Omega_m=\Omega_L$ keep increase till $30\degree$ and after that amplitude start to decrease and reaches zero when the magnetic field is directed along the y axis. Consider a two-level system $F=1\rightarrow F=0$ and the quantization vector is directed along the magnetic field. When the magnetic field is tilted in the yz plane, the light-polarization axis is perpendicular to the magnetic field. In this case the linearly polarized light containing two circularly polarized components can only create the coherence state between magnetic sublevels $m_F=\pm 1$ . Since the transition frequency between these two consecutive Zeeman energy sublevels is $2\Omega_L$  the resonance appears only at this frequency. However, when the magnetic field is tilted  in the xy plane, the light is a linear superposition of polarizations parallel and perpendicular to the magnetic field. In this case, the light can create coherences between sublevels with $m_F=1$ and $m_F=2$, so resonances are observed at both $\Omega_L $ and $2\Omega_L$. 
 \begin{figure}[h]
\centering\includegraphics[width=0.9\linewidth]{figures/filtered_data.png}
\caption{Raw and filtered FID signal.}
\end{figure}
 

