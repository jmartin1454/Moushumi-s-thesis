\chapter{Result\label{ch:results}}


\section{Study NMOR }
   \begin{itemize}
     \item NMOR near zero field \\
 During this measurement the pump beam was not been modulated. In this case,  at first degaussing the shields that surround the vapour cell is done in order to cancel background B-field.A function generator is used to drive the degaussing coil. After completing a degaussing sequence switch is opened to electrically disconnect the degaussing coil from experimental setup. Then B-field ramp is started and NMOR signal is observed through a oscilloscope which  connected to balance photodiode output. The magnetic field sweeping is done in a triangle wave. In Fig \ref{fig:TUNE} the whole sequence of measurement events has shown.The First secton of the scope trace has indicated to degaussing procedure, in 3rd section  switch was opened to disconnect the degaussing circuit from rest of the experimental setup, the 4th section describes the magnetic field~($B_z$) sweeping. 
In this measurement NMOR signal is used to determine effectiveness of degaussing procedure. After the degaussing procedure the observed field inside the shield is non-zero while the applied field $B_z$ was zero which might indicate the existence of some remnant field. Durig this measurement the magnetic field ($B_z$) was ramped about 2.5 nT peak to peak which corresponds to resonance width 0.4 nT. The effect of degaussing parameter on resonance width will describe on section (5.6).
\begin{table}[h]
\centering
\begin{tabular}{|l |l|}
\hline

\textbf{ SETTING}    & \textbf{VALUE} \\
\hline
Function generator &   \\
\hline
Frequency & ~ 10 Hz   \\

Sample rate    &  10000 sample/sec  \\

Amplitude   &  ~ 10 V \\
Offset  &      ~ 0 V  \\
\hline
\end{tabular}
\caption{Setting for degaussing }
\end{table}
\begin{figure}[h]
 \centering\includegraphics[width=0.7\linewidth]{figures/scope_trace_of_field_sweeping}
 
\caption{Scope trace of total measurement scheme of NMOR near zero field. The purple curve shows that the magnetic field was ramped about 2.5 nT peak to peak with a triangular wave while the blue colored dispersive lorenzian shape curve indicate induced optical rotation due to field sweeping.\label{fig:TUNE} }
\end{figure}
\newpage
In Fig \ref{fig:near zero field}    NMOR data were acquired by sweeping the magnetic field in a triangle wave; the peaks in the curve show a slight dependence on the direction of the sweep, due to the time response of the atoms. The resonance width is about 0.49 nT.

 \begin{figure}[h]
\centering\includegraphics[width=0.7\linewidth]{figures/near_zero_field}
\caption{Optical rotation as a function of magnetic field applied along the direction of the laser beam. The signal looks like a pure dispersive lorenzian curve. The measured resonance width is 0.49 nT.\label{fig:near zero field}}
\end{figure}
     \item Amplitude modulated magneto optical rotation(AM NMOR)\\
     To extent the magnetometer sensitivity to magnetic fields where Larmor precession is much faster than the ground state relaxation rate, it is necessary to synchronize the optical pumping rate with Larmor precession which can be achieved by modulating the light \cite{doi:10.1063/1.3225917}. In the case of AM NMOR, high-field resonance occur in addition to the regular zero-field resonance. The optical properties of the medium are being modulated at twice the Larmor frequency. In the case of strong external magnetic field, the dynamic Stark effect limits the sensitivity of NMOR based atomic magnetometry by reducing the accuracy of the field measurement. The advantage of using AM NMOR method is that it can reduce the Stark effect because light frequency is not affected by amplitude modulation\cite{gawlikoptical}. In AM NMOR, It is easily possible to control the number and amplitudes of the high field resonances by using the square wave modulation of light intensity.\\
In AM NMOR, we need to pump the atoms repeatedly, and a lock-in amplifier is used to observe the optical rotation at the same frequency. When this modulation frequency is the same as the harmonic frequency of the atoms we observe NMOR signal. During this study the reference frequency of lock-in amplifier is set to the modulation frequency. The x and y output signal of lock-in amplifier are shown in Fig~\ref{fig:AMOR}. The
black curve corresponds to the in-phase component(x), the red curve to the out of phase component(y). The dispersion structure(y component) is the resonance that can be observed in a regular rotation experiment with no modulation.  The modulation frequency of the pump beam was set to its resonance while the X and Y outputs recorded. In this case the observed resonance width is about 2.5 nT which has shown in Fig~\ref{fig:AMOR}.
\begin{table}[h]
\centering
\begin{tabular}{|l |l|}
\hline

\textbf{ SETTING}    & \textbf{VALUE} \\
\hline
Function generator &   \\
\hline
Frequency & 9.37kHz   \\

Waveform    &  Square  \\

Amplitude   &  $1V_{pp}$  \\
Offset  &       500 mV  \\
Duty cycle       &    $1\%$ \\
Frequency Deviation     &   40 Hz  \\
FM Frequency     &   100 mHz  \\
modulation waveform      &    Triangle \\
Amplitude modulation & On \\
\hline
Lock-in amplifier &     \\
\hline
Lock in frequency     & 9.37 KHz \\
Time constant     &  $300\mu s$ \\
Sensitivity      &  100mV  \\
\hline
\end{tabular}
\caption{Setting for AM NMOR at $1\mu T$ field}
\end{table}

\begin{figure}[h]
\centering\includegraphics[width=0.5\linewidth]{figures/AM_NMOR}
\caption{AMOR resonance signal with a 5 cm cell containing natural rubidium.Data was acquired by using  a balanced photodiode which demodulated through lock-in amplifier at 9.335 KHz.The observed resonance width is about 2.5 nT.\label{fig:AMOR}}
\end{figure} 
\newpage
\item Force Oscillation Scan\\
Study NMOR signal by sweeping resonance frequency with amplitude modulated light.\\
The magnetometer can be run as a forced oscillator, where a
frequency generator is used to sweep the frequency of the laser amplitude modulator through the NMOR resonance. In this case the applied magnetic field~($B_z$) remains fixed during a measurement cycle (pumping and probing).  The experimental setup for this measurement scheme remains almost same as for AM NMOR. The applied magnetic field was about 1  $\mu$T directed along light propagation direction. 
The measurement was taken by driving an Agilent 33522A function generator to different modulation frequencies in order to find the resonance frequency of the given Rb oscillator. The modulation waveform is a square wave with a duty cycle of 1\%. Since our pump beam is linearly polarized , a modulation at 2$\Omega_{L}$ is necessary because of the two fold symmetry of alignment state. The typical range of drive frequency is 9.31 KHz to 9.409 kHz for 1 $\mu$T field. An unmodulated linearly polarized probe beam is used to analyze the spin response. While the magnetometer response was recorded with a lock-in amplifier, which is connected to balanced photodiode output, demodulating the signal at the drive frequency of the atomic oscillator. 
The center of the resonance is used to determine the Larmor frequency and hence the magnetic field. Fig \ref{fig:FMOR} shows the resonance scan where data was taken by setting a function generator to different drive frequencies for the given atomic oscillator. During the scan drive frequencies was working as reference frequencies of the lock-in amplifier. The red and blue data points indicate x and y output respectively. When the modulation frequency is twice the Larmor frequency the x component reaches its maximum and the y component has its zero crossing. In this force oscillation scan the measured resonance width  is 1.67 nT. 
\begin{figure}[h]
\centering\includegraphics[width=0.5\linewidth]{figures/FM_modulation}
\caption{Optical rotation vs. forced oscillation frequency. NMOR resonance recorded with the 5 cm natural Rubidium cell, square-wave $100\%$ modulation of 1 duty cycle. The x component reaches its maximum and the y component has its zero crossing at $\Omega_m$ = $2\Omega_{L}$.In this case the measured resonance width is 1.67 nT.\label{fig:FMOR}} 
\end{figure} 
   \end{itemize}
  \newpage
   \section{FID at 0.2 $\mu$T and FID at 
  1 $\mu$T (need to edit) }
  Coherence time indicate the time atoms take to decay spontaneously from excited ground state after removing the pump beam. To measure the coherence time, a pump beam of linearly polarized light is
used to align the atoms along the magnetic field by optical pumping. The pump
beam is then blocked allowing the atoms to relax over time. Optical rotation of the linearly polarized probe beam is used to measure the coherence time. In this measurement optical pumping is done for 0.5 sec. Since at higher field coherence state decays quickly so coherence time get smaller. Fig~\ref{fig:graphs} shows the FID signal for  two different field.  At  0.2 $\mu$T field the coherence life time is about 75 ms Fig~\ref{fig:small field} and when the applied magnetic field is 1 $\mu$T  the coherence lifetime is about 23 ms Fig~\ref{fig:1 micro tesla}. Since at higher field, FID signal contain less no of oscillation it becomes challenging to extract oscillation frequency precisely and therefore magnetic field. In order to avoid this systematic effect most of the studies reported in this thesis has done at 0.2 $\mu$T.
   \begin{figure}
    \centering
 
    \begin{subfigure}[b]{0.45\textwidth}
        \centering
        \includegraphics[width=\textwidth]{figures/FID_0_2_micro_tesla}
        \caption{}
        \label{fig:small field}
    \end{subfigure}
    \hfill
    \begin{subfigure}[b]{0.45\textwidth}
        \centering
        \includegraphics[width=\textwidth]{figures/FID_1_micro_tesla}
        \caption{}
        \label{fig:1 micro tesla}
    \end{subfigure}
    \caption{FID NMOR resonances 
obtained with a probe light power of 22 $\mu W$, pump power of 40 $\mu$W. The pump and probe beam both are linearly polarized. FID signal was acquired at 0.2 $\mu$T (a).  FID signal at applied magnetic field 1 $\mu$T (b). The coherence time is larger for smaller field whereas for larger magnetic field the coherence time becomes small. }
    \label{fig:graphs}
\end{figure}
  \section{long term FID measurement}  
The forced oscillation scans and the single FID shot gives information about the present magnetic field. In order to get information about the change in magnetic field over time, long term data was taken in the FID mode. During this long term process the laser frequency was locked into D1 transition line of Rb-85 using DigiLock 110 laser locking module. A stable power supply was used to run the coil system inside the shield. In order to observe the FID signal, a Tektronix DPO 2014 oscilloscope was connected to X and Y output of output of lock-in amplifier. A python script is used to setup function generator and to trigger DAQ for long term FID measurement. This same python script is also used to grab data continuously from oscilloscope screen to computer. For further data analysis  another python script is used. On the basis of the X
and Y channel recordings, a least-square fit was done for each X and Y pair. The measured oscillation frequency of decaying signal was then convert to magnetic field using equation (4.7). The recorded B-field data vs. time was further used to calculate Allan deviations, getting  information on the deviations from the mean magnetic field vs. integration time (bin size). Measuring magnetic field over long period of time gives us information about the magnetometer performance. Toward this end,we study long term field measurement. Fig~\ref{fig:long term field} ~represents the magnetic field recorder over 4 hours on three different day. Each data points in this graph correspond to single FID scan. The observed field drift is almost same (15 pT) for three days. This measurement was conducted at 0.2 $\mu$T magnetic field.
\begin{figure}[h]
\centering\includegraphics[width=0.85\linewidth]{figures/field_3_day}
\caption{Magnetic field recorded over 4 hours on three different day. The observed field drift is almost same(15 pT) for three days.\label{fig:long term field}}
\end{figure}
   \begin{itemize}
   \item summary of statistical error, systematic error, stability via allan deviations.
   \end{itemize}
   \newpage
   \section{Optimization of cycle time} 
   \begin{figure}
    \centering
 
    \begin{subfigure}[b]{0.425\textwidth}
        \centering
        \includegraphics[width=\textwidth]{figures/Capture}
        \caption{}
        \label{fig:pump short}
    \end{subfigure}
    \hfill
    \begin{subfigure}[b]{0.42\textwidth}
        \centering
        \includegraphics[width=\textwidth]{figures/Capture2}
        \caption{}
        \label{fig:pump long}
    \end{subfigure}
    \caption{(a) FID signal for pump time~ 0.49 s and probe time~ 0.4 s. (b) FID signal for pump time  0.1 s and probe time ~0.2 s. Both measurement were conducted at $0.2~\mu$T magnetic field.}
    \label{fig:three graphs}
\end{figure} 
   \begin{figure}[h]
\centering\includegraphics[width=0.75\linewidth]{figures/pump_time}
\caption{Measured B-field vs. time   for different pump time. The longest pump time is 0.49 s while the shortest pump time is 0.1 s. No obvious dependency of pump time on  measured magnetic field has been observed during this study}
\end{figure}
In FID NMOR optical pumping is done for polarizing Rb atoms and afterward the spontaneous decay of excited atom is observed. So a complete cycle of FID measurement consist of pump and probe time.Pump time represents the time atoms take to generate a polarized ground state. In this study we were trying to study how long the optical pumping of Rb atom should have continued in order to generate an alignment and optical pumping for long time does make any difference in measuring magnetic field precisely or not. FID signal with 0.49 sec pump time and 0.4 sec probe time has shown in Fig \ref{fig:pump short}. On the other hand, Fig \ref{fig:pump long} shows  the FID signal for 0.1 s pump time and 0.2 sec probe time. In Figure 5.8 the measured magnetic field in about 300 sec has shown for different pump time. The longer pump time is 0.49 sec and the shorter one is 0.1 sec. It is obvious from the plot that pump time doesn't effect in field precision. However for our Rb magnetometry setup it is not possible to make the pump time shorter than 0.1 sec because this 0.1 sec is the minimum time atoms need to get aligned for making a coherence state after interacting with laser beam. Since the amplitude of FID NMOR signal reach it's maximum while atoms are in coherence state. Thus It is easy to understand coherence state is ready or not by observing the signal amplitude. In Fig~\ref{fig:pump short} the optical pumping is done for 0.49 s while signal amplitude reach its maximum over 0.1 s and after that signal amplitude remains unchanged till 0.49 s. In this case there is no point to pump more than 0.1 s. Same situation happens about probe time. If signal amplitude decays so quickly there is no point to set longer probe time. By optimizing pump and probe time we can speed up data acquisition system. After optimization for a single FID scan it only takes 0.35 s  without losing any information. By using this faster data acquisition system it is possible to record multiple FID run with in a short period of time which is helpful to gather more information about magnetic field environment.
\newpage
   \section{current with existing power supply}
   \begin{itemize}
   \item Analysis of current stability.\\
 An Agilent B2962A power supply was used to run the coil system which produce magnetic field inside the shield. In order to develop a highly sensitive magnetometer, it is very important to make use of a highly stable power supply. The idea was to determine the current drifts of the power supplies, mainly based on the fact that the given current supply actually acts as a voltage supply with the output voltage transformed into an output current by a resistor which is sensitive to external influences such as temperature. Fig~\ref{fig:current} shows the current stability of power supply over 10000 sec. During this time the current only change by 30 nA.  
   \begin{figure}[h]
\centering
\includegraphics[width=0.8\linewidth]{figures/current}
\caption{Recorded current over 10000 sec using Agilent B2962A power source. During this measurement coil current only changed by 30 nA.\label{fig:current} }
\end{figure}
   \item Current stability is better than typical measured field changes.
   \item Change in magnetic field due to the change in coil current\\
 A study has been conducted to determine the field change by changing the coil current by a known amount. The main objective of this study is to check the Rb magnetometer performance on magnetic field measurement. During this measurement coil current has been changed by $\pm$ 0.001 mA. As a result a field change of $\pm$ 50  $\mu$T  has been observed. The laser frequency was locked to Rb transition frequency and all other setting kept unchanged. During this study the magnetometer has been operated in FID mode. Fig~\ref{fig:field change} presents the magnetic field change over time for different coil current. For better understanding Fig~\ref{fig:field change} has been divide into five different regions. In region-1 magnetic field  has been recorded for 300 s while the applied current to the Z coil was 4.5 mA. Then in region-2 the applied current to the z coil has been increased by 0.001 mA i.e., total current 4.501 mA and measured the magnetic field for another 300 sec. It is clear from the figure that the field has been changed instantly by 50 $\mu$T due to the changed coil current. Now in region-3, the coil current has been reduced to 4.5 mA and measured field for 300 sec. After that the coil current has been decreased by 0.001 mA. As a result the field also decreased by 50~$\mu$T in this region.Finally in region-5 the coil current has been set to 4.5 mA again and observed the corresponing quick change in field.

   \begin{figure}[h]
\centering\includegraphics[width=0.7\linewidth]{figures/field_change_with_current}
  
\caption{The change in magnetic field  by changing coil current has been studied over 1400 sec. The blue line on different regions of the graph displays the measured magnetic field corresponding to the change in coil current. \label{fig:different sample rate}}
\end{figure}
   \end{itemize}
   \section{degaussing studies}  
   \begin{itemize}
   \item different degaussing schemes(changing sample rate)\\
Degaussing process is done in order to avoid the environmental perturbation and reduce any remnant field inside the four layer $\mu$ metal magnetic shield \cite{doi:10.1063/1.2713433}. For our degaussing system we are using an envelope function which has $1\times 10^6$ points.Degaussing is complete after $5\times 10^6$ points.Sample rate affects how quickly we move through this waveform. A detailed analysis of NMOR signal by changing the degaussing parameter could give us some information about the goodness of degaussing procedure. Toward this end, a study has been carried out to determine  the dependency of the resonance width on the sample rate. In this case, data was acquired by sweeping the magnetic field near zero field. Before each measurement degaussing the innermost layer of shield has done. During this measurement only one degaussing parameter,sample rate, was varied in order to study the effect of degaussing parameter on magnetic field measurement.  Optical rotation as a function of magnetic field for different sample rate has shown in Fig~\ref{fig:different sample rate}. The resonance width $\Delta B$, the difference between two peak of the dispersive curve, changed for different sampling rate. It can be seen from Fig~\ref{fig:different sample rate} that, the resonance width is about 0.49 nT for sample rate 10000 sample/sec and the resonance width is 0.38 nT for sample rate 80000 sample/sec. Fig~\ref{fi} shows the resonance width as a function of sample rate. It is obvious from the graph that the resonance width becomes narrower for larger sample rate.    
   
   \begin{figure}[h]
\centering\includegraphics[width=0.65\linewidth]{figures/sample_rate}
\caption{ Optical rotation vs. measured B-field   for different sample rate. The resonance width is narrower(0.38 nT) for higher sample rate(80000 sample/sec) whereas the resonance width becomes broaden (0.49 nT) for lower sample rate.\label{fig:different sample rate} }
\end{figure}
\begin{figure}[h]
\centering\includegraphics[width=0.6\linewidth]{figures/field_vs_sample_rate}
\caption{Resonance width vs. sample rate. Resonance width decreases with increasing sample rate.When sample rate is 5000 sample/sec the observed resonance width is 0.55 nT. On the other hand the resonance width is 0.38 for sample rate 80000 sample/sec. \label{resonance width} }
\end{figure}

   \item Ramp field up/down \\
  During this study the magnetometer has been operated in FID mode at 0.2 $\mu$T field. The recorded magnetic field over 2200 sec has been showed in Figure 5.13(a). No degaussing has been done during this measurement. It can be seen from the figure the magnetic field increase linearly for first 300 sec then showed a decrease in field. The field was pretty stable between 600 s and 1800 s and after that the field showed a rapid increase. The overall field change is about 15 pT during the measurement. Data was acquired by ramping $B_z$  from  0.2 $\mu$T to 10 $\mu$T then again set it to 0.2 $\mu$T and collect FID signal. By doing this field ramping we intentionally perturb the magnetic field environment inside the shield. After this field ramping the long term field measurement has been conducted for another 2200 sec. In this case, the field showed a downward drift of about 35 pT. Thus field ramping has been changed the magnetic field environment inside the magnetic shielding.
   \begin{figure}
    \centering
 
    \begin{subfigure}[b]{0.425\textwidth}
        \centering
        \includegraphics[width=\textwidth]{figures/ramp_1}
        \caption{}
        \label{fig:three sin x}
    \end{subfigure}
    \hfill
    \begin{subfigure}[b]{0.42\textwidth}
        \centering
        \includegraphics[width=\textwidth]{figures/ramp_2}
        \caption{}
        \label{fig:five over x}
    \end{subfigure}
    \caption{ FID signal at $B_z=0.2 \mu$T magnetic field.(a) no degaussing (b) ) Ramp $B_z$  from  0.2 $\mu$T to 10 $\mu$T then again set it to 0.2 $\mu$T and collect FID signal.}
    \label{fig:three graphs}
\end{figure}
   
   \item degaussing second innermost shield effect in field drift.\\
   After doing transverse field study (applying current to X and Y coil along with Bz coil) magnetic field environment was pretty unstable. Figure 5.14(a) shows about 50 pT drift in magnetic field in 4000 sec. In order to cancel background field inside shield degaussing innermost layer of $\mu$ metal is done but it doesn't help to solve field drift problem. Then degaussing the 2nd innermost layer of shield is done which solve the field drift problem. Since the end cap of the innermost shield layer is not in use some background field leaked into 2nd layer which was causing the drift in field. So by degaussing the 2nd layer of shield canceled background field and field becomes so stable (Figure 5.14 (b)).
   
  \begin{figure}
    \centering
 
    \begin{subfigure}[b]{0.45\textwidth}
        \centering
        \includegraphics[width=\textwidth]{figures/before_degaussing}
        \caption{}
        \label{fig:three sin x}
    \end{subfigure}
    \hfill
    \begin{subfigure}[b]{0.45\textwidth}
        \centering
        \includegraphics[width=\textwidth]{figures/after_degaussing}
        \caption{}
        \label{fig:five over x}
    \end{subfigure}
    \caption{ Magnetic field recorded over 3 hours. (a) The stability of magnetic field without performing any degaussing. A downward field drift of about 45 pT  has observed in this case (b) magnetic field stability after degaussing the 2nd innermost layer of magnetic shielding. Only a couple pT field drift has observed in this case.}
    \label{fig:three graphs}
\end{figure}
   \end{itemize}
   \newpage
   \section{Laser tuning} 
   \begin{itemize}
   \item effect of careful tuning\\
	The main purpose of this study is to understand the importance of laser tuning on coherence life time of excited atoms. During the study optical pumping is done to polarize the rubidium atom and hence produce coherence state. When the laser light  is  exactly tuned to resonance with the atomic transition, the lifetime of coherence state becomes longer and the amplitude of resonance signal reaches its maximum. On the other hand coherence state decay  quickly when laser frequency detunes from atomic transition. In this case, the signal amplitude reduces due to frequency detuning. Figure 5.15 shows the effect of laser tuning on coherence lifetime. For proper laser tuning the observed coherence time is 68 ms (a) whereas for bad tuning coherence time reduces to 59 ms (b). On a fundamental level, the magnetometer actually measures the energy splitting between the Zeeman sublevels of the atomic ground state due to the magnetic field. The linewidth of such a spectroscopic measurement is given by the coherence lifetime $T_2$ of the atomic spins:
\begin{equation}
 ΔB = \Delta\omega/\gamma  = 1/γ T_2
\end{equation}

The development of a sensitive magnetometer depends on achieving the maximum possible polarization lifetime.  For very short coherence time atomic spin depolarize quickly which limits the sensitivity of the magnetometer.
Since  longer coherence time and larger signal amplitude indicate better frequency precession and therefore precise field measurement it is very important to make sure the laser tuning has done carefully.
   \begin{figure}
    \centering
 
    \begin{subfigure}[b]{0.45\textwidth}
        \centering
        \includegraphics[width=\textwidth]{figures/perfect_tuning}
        \caption{}
        \label{fig:good tuning}
    \end{subfigure}
    \hfill
    \begin{subfigure}[b]{0.45\textwidth}
        \centering
        \includegraphics[width=\textwidth]{figures/bad_tuning}
        \caption{}
        \label{fig:bad tuning}
    \end{subfigure}
    \caption{(a) Recorded FID signal while laser is perfectly tuned to atomic transition frequency. (b) FID resonance signal when laser is slightly detuned from transition frequency.}
    \label{fig:effect of tuning on field}
\end{figure}
   \item problem with drift of tune\\
   For studying long term stability data was taken for about 12 hours.Laser was locked to D1 transition line of Rb using Digilock laser locking software. Figure 5.15 shows 120 pT observed drift in magnetic field over 12 hours. Although DAVLL was in use laser frequency was not locked for 12 hours. As a result laser tuning moved due to mode hoping which causes the drift. The current laser locking system only works perfectly for maximum 4 hours. The possible reason for this mode hoping is the Polarizing beam splitting cube of our DAVLL system \cite{principles}. The disadvantage of using this type of PBS is that they show flaky optical behavior over longer times.
   \begin{figure}[h]
\centering\includegraphics[width=0.5\linewidth]{figures/field_drift}
\caption{Magnetic field recorded over 12 hours.In this measurement the observed field drift is about 120 pT.\label{digilock field drift}}
\end{figure}
\item manual tuning\\
   During this study frequency of laser light is tuned to atomic transition maximize optical rotation.The laser tuning was adjusted manually time to time for maintaining same signal amplitude during measurement .The main objective of this study is to check the observed drift in field measurement is the real field drift or the magnetometer drift due to the mode hoping. Instability of laser locking system due to the poor performance of PBS which is used in DAVLL system is the probable reason behind this mode-hoping. According to figure 5.17~ (b) we can say that during this measurement the amplitude of FID NMOR was pretty stable except small fluctuations over 20000 sec . The observed small fluctuation is due to the manual adjustment while keeping the laser frequency tuned to atomic transition over long period of time. Obtaining stable signal amplitude is a indication that laser is not drifting to much. Although laser frequency was not drifting a lot during  the measurement, still there is a  drift in magnetic field (Figure 5.17 (a)). The overall drift is about 20 pT over 20000 sec. So it can be conclude that the drift in laser frequency is not the main reason behind this induced field drift. 
   
    
   \end{itemize}
   \begin{figure}
    \centering
 
    \begin{subfigure}[b]{0.45\textwidth}
        \centering
        \includegraphics[width=\textwidth]{figures/manual_tuning}
        \caption{}
        \label{fig:field manual tuning}
    \end{subfigure}
    \hfill
    \begin{subfigure}[b]{0.45\textwidth}
        \centering
        \includegraphics[width=\textwidth]{figures/amplitude_manual_tuning}
        \caption{}
        \label{fig:amplitude manual tuning}
    \end{subfigure}
    \caption{(a) magnetic field vs. time. (b) amplitude of recorded FID NMOR signal over 20000 sec. During this study laser tuning is maintained manually}
    \label{fig:manual tuning}
\end{figure}
   \section{Optimization of pump and probe beam power} 
 A study has been performed to determine how optimization of the pump and probe beam power effect on precise field measurement.  During this study, the magnetometer has operated in FID mode.  A  linearly polarized probe beam has been used to analyze the spin response. In figure 5.18 the measured magnetic field has been displayed for two different  Probe beam power. Figure 5.18 (a) present the measured field over 100 sec with 15 $\mu$ W probe power while figure 5.18(b) display the measured field over 100 sec for probe power 30 $\mu$W. In both cases, the pump beam power has been kept unchanged ($5\%$ duty cycle). Each blue points in this graph represent a single FID scan. As can be seen from figure 5.18 the magnetic field is more scattered for high beam power. Precession width is about $15$ pT for probe beam power 30 $\mu$W while for low beam power(15 $\mu$W) the precession width is about $7$ pT. It can be concluded lower probe beam power is better for precession  magnetometry because the narrower the precession width more sensitive the magnetometer is . Although low probe beam power is better for precise field measurement, it's hard to work with because of the dimness of light.\\
  \begin{figure}
    \centering
 
    \begin{subfigure}[b]{0.45\textwidth}
        \centering
        \includegraphics[width=\textwidth]{figures/beam_power_less}
        \caption{}
        \label{fig:power less}
    \end{subfigure}
    \hfill
    \begin{subfigure}[b]{0.45\textwidth}
        \centering
        \includegraphics[width=\textwidth]{figures/beam_power_double}
        \caption{}
        \label{fig:power double}
    \end{subfigure}
    \caption{ Magnetic field as a function of time for different power of probe beam.(a) measured magnetic field with probe beam power 15 $\mu$w. Precession width is about $7$ pT (b) measured magnetic field with precession width $15$ pT for probe beam power 30 $\mu$w.}
    \label{fig:different probe power}
\end{figure}
Another study has been conducted to understand the influence of pump beam power on precession magnetometry. During this study, the probe beam power has kept unchanged while pump power has changed. In this measurement data has been acquired by running the magnetometer in FID mode. In FID mode amplitude modulation of pump beam has been done by using a AOM.  So the pump beam power can be change by changing the duty cycle of square wave modulation. When duty cycle is set to $5\%$ the pump beam power is 40 $\mu$W and for $16 \%$ duty cycle power is 128 $\mu$W. The histogram of measured magnetic field for different pump beam power has been showed in figure 5.19. The precession width (sigma) become larger for higher duty cycle while it becomes narrower for small duty cycle. In the case of 5$\%$  duty cycle precession width is about 2.3 pT  5.19 (a) and for 16 $\%$ duty cycle the spread is about 4.6 pT. It is clear from the plot that, at higher duty cycle data points are not statistically distributed.  When a pump beam with a higher duty cycle is used  in precession magnetometry, the NMOR  signal amplitude increase accordingly. But it has been observed that error in frequency measurement also increases with higher signal amplitude which was not expected. A strong correlation between magnetic field and phase of the NMOR signal also has been observed for higher pump power. So it can be concluded that using higher duty cycle could induce more systematic errors in field measurement which led to the next study.
 \begin{figure}
    \centering  \includegraphics[width=\textwidth]{figures/pump_beam}
    \caption{ Magnetic field vs. time for different pump beam power.(a) measured magnetic field with duty cycle $5 \%$. Precession width is about $2.3$ pT (b) measured magnetic field with Precession width $4.6$ pT for  16 $\%$ duty cycle.}
    \label{fig:different pump power}
\end{figure}
   \section{Systematic error check in frequency measurement in FID mode} 
   \subsection{How far the reference frequency of lock-in amplifier should set to measure field correctly}
  
  In FID NMOR the atomic sample is polarized once by optical pumping and then observed the spontaneous decay of excited atom while the pump beam is off.In order to capture the FID signal correctly, the reference frequency of lock-in amplifier is normally set to about 100 Hz apart from the resonance frequency. In this study we kept all other setting fixed only changed the Lock-in reference frequency and observed the effect of that on field measurement study. The measurement was conducted at $0.2\mu$ T field. The FID signal for different reference frequency of Lock-in amplifier has shown in Figure 5.20. In the case of Figure 5.20 (a) the reference frequency of Lock-in was set to 1943.9 Hz(90 Hz far from resonance frequency). On the other hand for Figure 5.20 (c) the reference frequency of Lock-in was set to 2015 Hz(20 Hz far from resonance frequency). In this study the resonance frequency is 2035 Hz. In Figure 5.21 the measured magnetic field over 35 sec for different lock-in reference frequency has shown. It is obvious from the plot that if we set reference frequency very close to resonance frequency magnetic field start to oscillate. The exact reason behind this observed field oscillation remains unknown. We are thinking that when lock-in reference frequency is set very close to resonance frequency the fit function might fail to fit data properly due to the less zero crossings. So it seems like a systematic effect on field measurement rather than a real fact.
 \begin{figure}
    \centering
    \begin{subfigure}[b]{0.4\textwidth}
        \centering
        \includegraphics[width=\textwidth]{figures/reference_frequency1}
        \caption{}
        \label{fig:three sin x}
    \end{subfigure}
    \hfill
    \begin{subfigure}[b]{0.4\textwidth}
        \centering
        \includegraphics[width=\textwidth]{figures/reference_frequency3}
        \caption{}
        \label{fig:five over x}
    \end{subfigure}
    \begin{subfigure}[b]{0.4\textwidth}
        \centering
        \includegraphics[width=\textwidth]{figures/reference_frequency2}
        \caption{}
        \label{fig:five over x}
    \end{subfigure}
    \caption{FID signal for different reference frequency of Lock-in amplifier while the resonance frequency was 2035 kHz.(a) reference frequency was set to 100 Hz far from resonance (b) difference between resonance and reference frequency is 40 Hz,(c) reference frequency was set to 2015 Hz while resonance frequency 2035 Hz. }
\end{figure}
\begin{figure}[h]
\centering\includegraphics[width=0.8\linewidth]{figures/reference_frequency}
\caption{Measured magnetic field for different lock-in reference frequency.The red curve represents measured magnetic field when reference frequency was set to 2015 Hz which is 20 Hz far from resonance frequency. The blue curve shows magnetic field for lock-in reference frequency 1943.9 Hz.}
\end{figure}
\begin{figure}[h]
\centering\includegraphics[width=0.8\linewidth]{figures/reference_frequency}
\caption{Measured magnetic field for different lock-in reference frequency. The red curve represents measured magnetic field when reference frequency was set to 2015 Hz which is 20 Hz far from resonance frequency. The blue curve shows magnetic field for lock-in reference frequency 1943.9 Hz.}
\end{figure}
\begin{figure}[h]
\centering\includegraphics[width=0.8\linewidth]{figures/phase}
\caption{Measured magnetic field for different lock-in reference frequency.The red curve represents measured magnetic field when reference frequency was set to 2015 Hz and field for lock-in reference frequency 1943.9 Hz.}
\end{figure}
\newpage
   \begin{itemize}
   \item effect of lock-in time constant\\
  A SRS-830 lock-in amplifier is used to grab FID NMOR signal which was connected to the output of balanced photodiode. Table 4.1 represents all settings for FID NMOR. In this study the systematic effect of changing locking amplifier time constant on magnetic field measurement has been discussed. Figure 5.22  shows the histogram of measured magnetic field for two different time constant of lock-in amplifier while all other settings is same. In the case of Figure 5.22 (a) the lock-in time constant is 1 ms and figure 5.22 (b) shows the histogram of magnetic field for time constant 300 $\mu$s. Here sigma is representing the precession width of field. It can be seen from the figure that the calculated sigma is larger for time constant 1 ms compared to 300 $\mu$s. This measurements was conducted at 0.2 $\mu$T field. The resonance frequency and lock-in reference frequency are 2035 Hz and 1943 Hz respectively.
   \begin{figure}
    \centering
    \begin{subfigure}[b]{0.4\textwidth}
        \centering
        \includegraphics[width=\textwidth]{figures/time_constant}
        \caption{}
        \label{fig:time constant long}
    \end{subfigure}
    \hfill
    \begin{subfigure}[b]{0.4\textwidth}
        \centering
        \includegraphics[width=\textwidth]{figures/time_constant_300micro_sec}
        \caption{}
        \label{fig:time constant short}
    \end{subfigure}
    \caption{Histogram of measured magnetic field for different time constant of lock-in amplifier. (a) lock-in time constant was set to 1 ms (b) for lock-in time constant 300 $\mu$s. }
\end{figure}
    \item timing drift in Tektronix oscilloscope clock
   \end{itemize}
   \newpage
   \section{Calculation of Allan deviation} 
   Allan variance is used to investigate long term stability.In this case, the Allan deviation was calculated in order to determine the deviation of the average field value for a given integration time as a function of the integration time\cite{doe:website2}. Consider a time series of measurements $y_i$ is acquired at times $t_i$ and  a subset of N data points are present.
 \begin{equation}
 \bar{y}_{n}= \sum_{i=(n-1)N+1}^{nN} \frac{y_i}{N} 
 \end{equation}

The Allan deviation can be express as
\begin{equation}
\sigma_y(\tau)=\sqrt{\sigma_y^2(\tau)}=Var_y(\tau)
\end{equation}
Where $\sigma_y$ is the Allan deviation, $\tau$ the time of each frequency estimate and $Var_y$ is the variance of the observed data points.
\begin{equation}
{\sigma_y^2(\tau)}=\frac{1}{2}(\bar{y}_{n+1}-\bar{y}_{n})
\end{equation}
where $\bar{y}_{n}$ is the fractional frequency average over the observation time $\tau$.
Allan deviation formula for sinusoidal waveform:
Suppose
\begin{equation}
y(t) = A\cos (\omega t + \phi)
\end{equation}
The average of this function over a time-interval  is
\begin{equation}
\bar{y}_{n}=\frac{2A}
{\tau \omega}
cos(\omega(t_n +\frac{\tau}{2})+\phi)sin(\frac{\omega \tau}{2})
\end{equation}
and
\begin{equation}
\bar{y}_{n+1}=\frac{2A}
{\tau \omega}
cos(\omega(t_{n+1} +\frac{\tau}{2})+\phi)sin(\frac{\omega \tau}{2})
\end{equation}
Now from equation(5.5) and (5.6) we can write
\begin{equation}
\bar{y}_{n+1}-\bar{y}_{n}=\frac{2A}
{\tau \omega}sin(\frac{\omega \tau}{2})[cos( \omega(t_{n+1} +\frac{\tau}{2})+\phi)-cos(\omega(t_n +\frac{\tau}{2})+\phi)]
\end{equation}
Thus Allan deviation can be written as

\begin{equation}
{\sigma_y^2(\tau)}=\frac{1}{2}(\bar{y}_{n+1}-\bar{y}_{n})=(\frac{4A}
{\tau \omega}sin^2(\frac{\omega \tau}{2}))^2 \sin^2     (\frac{\omega(t_{n+1} +t_{n}+\tau)}{2})+\phi)
\end{equation}
For a large number of randomly
distributed tk the average of the sine-squared function is
\begin{equation}
\sin^2    (\frac{\omega(t_{n+1} +t_{n}+\tau)}{2})+\phi)=\frac{1}{2}
\end{equation}
We therefore find that
\begin{equation}
{\sigma_y^2(\tau)}=(\frac{2A}
{\tau \omega}sin^2(\frac{\omega \tau}{2}))^2
\end{equation}
\begin{figure}[h]
\centering\includegraphics[width=0.6\linewidth]{figures/allan_plot}
\caption{Allan deviation vs. averaging time(s)}
\end{figure}
In FID mode, the resonance signal looks like a decaying sine wave. By fitting the NMOR signal with a damped sine wave the oscillation frequency has been extracted and finally this oscillation frequency has been translated to field.The Allan deviation of the measure field has been plotted in Figure 5.23.
\newpage
  \section{study the effect of room temperature in magnetic field} 
  \begin{figure}[h]
\centering\includegraphics[width=0.6\linewidth]{figures/temp}
\caption{Allan deviation vs. averaging time(s)}
\end{figure}
\newpage
   \section{Study magnetometer performance with transverse field}  
   
\begin{figure}[h]
\centering\includegraphics[width=0.5\linewidth]{figures/tilted_field_amp_vs_angle}
\caption{The amplitude of the FID NMOR signals recorded at $2\Omega_L$ vs. the tilt angle of the magnetic field in xz-plane}
\end{figure}
 The magnetometric method based on FID NMOR is a very sensitive technique of magnetic field measurements. Those measurements are scalar, i.e., the position of a given resonance depends only on the magnitude not the direction of the magnetic field. However, the relative magnitudes of the FID NMOR resonances could have some dependency on the magnetic field direction. Thus there is a possibility to get some information about the direction of the magnetic field by doing a detailed analysis of the FID NMOR signal.  For this reason, the dependency of the FID NMOR signal on the magnetic field direction has studied here.  In this Rb NMOR magnetometry setup the rubidium atoms interacted with a z-directed laser light beam which is linearly polarized
along the y axis. In the FID NMOR technique, the magnetic field is generally directed along the light propagation direction and the resonance occurs at $2\Omega_L$.  This effect can be explained by considering that the polarization returns to its original state after a 180° rotation because of the two-fold symmetry of the optically pumped state. As a result, the optical rotation induced by the rotating linear dichroism is periodic at twice the Larmor frequency.
We have observed that Resonances in nonlinear magneto-optical rotation with amplitude modulated light by tilting the magnetic field at angles away from the direction of light propagation while operating the Rb magnetometer in FID mode. When the field is tilted in the plane perpendicular to the light polarization direction an resonance appears at $2\Omega_L$. In this case no additional resonance appears for modulation frequency $\omega_L$. The amplitude of the resonance signal decreases with increasing tilt angle. Figure 5.24 shows the amplitude of the FID signal for the magnetic field tilted in the xz plane at various angles to the light propagation direction at modulation frequency $2\Omega_L$.
However, We also observed that by tilting the field direction toward the light polarization direction a new resonance occurs at $\Omega_L$ along with the main resonance at $2\Omega_L$. The resonance signal recorded at $\Omega_L$ contains two frequency components. Figure 5.25 (b) display the FFT of resonance signal for tilt angle $15\degree$ where two peak corresponds to two frequency components . The amplitude of  resonance signal at $2\Omega_L$  decreases as the angle between magnetic field (B) and the light propagation direction increases while the amplitude of new resonance  increases with increasing tilt angle at $\Omega_L$. Figure 5.25(a) shows the resonance amplitude at $\Omega_m=2\Omega_L$ keep decreases with increasing tilt angle in yz plane while  the resonance amplitude measured at $\Omega_m=\Omega_L$ keep increase till $30\degree$ and after that amplitude start to decrease and reaches zero when the magnetic field is directed along the y axis.Consider a two-level system F=1→F=0 and the quantization vector is directed along the magnetic field. When the magnetic field is tilted in the yz plane, the light-polarization axis is perpendicular to the magnetic field. In this case the linearly polarized light containing two circularly polarized components can only create the coherence state between magnetic sublevels $m = ±1$ . Since the transition frequency between these two consecutive Zeeman energy sublevels is $2\Omega_L$  the resonance appears only at this frequency. However, when the magnetic field is tilted  in the xy plane, the light is a linear superposition of polarizations parallel and perpendicular to the magnetic field. In this case, the light can create coherences between sublevels with $m=1$ and $m=2$, so resonances are observed at both $\Omega_L $ and $2\Omega_L$.

\begin{figure}
    \centering
   \begin{subfigure}[b]{0.45\textwidth}
        \centering
        \includegraphics[width=\textwidth]{figures/amp_tiltangle}
        \caption{}
        \label{fig:y equals x}
    \end{subfigure}
    \hfill
     \begin{subfigure}[b]{0.45\textwidth}
        \centering
        \includegraphics[width=\textwidth]{figures/tansverse_field_fft}
        \caption{}
        \label{fig:three sin x}
    \end{subfigure}
    \caption{(a) The amplitude of the FID NMOR signals as a function of tilt angle recorded at $\Omega_L$ and $2\Omega_L$ vs. the tilt angle of the magnetic field in the plane defined by the light-polarization and light propagation vectors(yz-plane). (b) FFT of FID NMOR signal in the presence of trensverse field.}
\end{figure}

